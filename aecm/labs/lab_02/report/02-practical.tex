\chapter{Практическая часть}

\section{Задание №1}

\includelisting {ex_asm.txt} {Программа для примера}

\begin{enumerate}[label=\arabic*)]
	\item Объявление секции $.text$, содержащей исполняемый код.
	\item Объявление символа $\_start$, имеющего глобальную видимость. Символ $\_start$ это специальный символ, обозначающий точку входа в программу.
	\item Метка.
	\item Объявление констант.
	\item Арифметические выражения над константами могут использоваться в командах
	на месте непосредственного операнда.
	\item Загрузка в $x1$ адреса символа $\_x$ (то есть, начала массива).
	\item Загрузка в $x2$ числа по адресу, содержащемуся в $x1$ по смещению 0.
	\item Добавление к $x31$ (который хранит результат) значения $x2$.
	\item Смещение указателя $x1$.
	\item Уменьшение счетчика цикла.
	\item Условный переход на метку $loop$.
	\item Бесконечный цикл.
	\item Объявление секции данных.
	\item Начало описания массива.
\end{enumerate}

Программа, представленная на листинге \ref{lst:ex_asm.txt} выполняет суммирование значений элементов масcива слов и увеличивает это значение на 1. Примером данной небольшой программы для RV32I мы будем пользоваться далее для исследования процесса выполнения команд.

\clearpage

Дизассемблированный код представлен на листинге \ref{lst:ex_disasm.txt}.
\includelisting {ex_disasm.txt} {Дизассемблированный код программы для примера}

\clearpage

Можно сказать, что данная программа эквивалентна следующему псевдокоду на языке $C$, представленному на листинге \ref{lst:ex_c.txt}.

\includelisting {ex_c.txt} {Псевдокод программы для примера на языке $C$}

\clearpage

\subsection*{Условие задания}

В процессе выполнения задания необходимо выполнить следующие действия:

\begin{enumerate}[label=\arabic*)]
	\item Ознакомиться с теоретической частью, внимательно изучить примеры.
	\item Перейти в подкаталог src командой cd riscv-lab/src.
	\item Выполнить сборку, запустив команду make. Убедиться, что был создан файл test.hex,
	содержащий шестнадцатеричное представление программы, а в окне терминала отобразился
	дизассемблерный листинг. Сравнить дизассемблерный листинг с тем, который приведен в примере.
	\item Создать новый файл, содержащий текст программы по индивидуальному варианту. Поместить его
	в каталог src. Текст программы сохранить в файле с расширением .s.
	\item Изучить текст программы по индивидуальному варианту. Поместить в отчете псевдокод,
	соответствующий данной программе.
	\item Анализируя исходный текст программы, ответьте на вопрос: какое значение должно
	содержаться в регистре x31 в конце выполнения программы?
	\item Изменить в Makefile строку SRC= так, чтобы ее содержимое соответствовало
	имени файла с текстом программы без расширения .s.
	\item Выполнить компиляцию командой make. В процессе будет создан файл с расширением .hex,
	хранящий содержимое памяти команд и данных, а в окне терминала отобразится дизассемблерный листинг, который необходимо поместить в отчет вместе с исходным текстом.
\end{enumerate}

\clearpage

\subsection*{Результаты выполнения}

При переходе в подкаталог $riscv-lab/src$ и запуске сборки командой $make$, был получен файл $test.hex$ и в окне терминала отобразился следующий листинг:

\includeimage
{test_disasm} % Имя файла без расширения (файл должен быть расположен в директории inc/img/)
{f} % Обтекание (без обтекания)
{h} % Положение рисунка (см. figure из пакета float)
{0.5\textwidth} % Ширина рисунка
{Выполнение команды $make$ к программе для примера} % Подпись рисунка

Полученный дизасемблерный листинг в точности повторяет тот, что был приведен в примере (листинг \ref{lst:ex_disasm.txt})

\clearpage

Был создан файл, содержащий текст программы по индивидуальному варианту, с названием $my14var.s$ и помещен в каталог $src$.

Код программы для 14-го варианта представлен на листинге \ref{lst:var14_asm.txt}.
\includelisting {var14_asm.txt} {Код программы для 14-го варианта}

\clearpage

Псевдокод на языке $C$, соответствующий данной программе представлен на листинге \ref{lst:var14_c.c}.

\includelisting {var14_c.c} {Псевдокод программы для 14-го варианта на языке $C$}

В ходе анализа исходного текста программы, соответствующего 14-му варианту, было установлено, что ее задачей является поиск максимального элемента в массиве. Следовательно, в регистре $x31$ в конце работы программы будет лежать значение максимального элемента массива (в данном случае -- 8).

После присвоения в Makefile строки SRC имени файла программы без расширения, соответствующее 14-му варианту, была выполнена команда $make$ ( рисунок \ref{img:make_my14var}):

\includeimage
{make_my14var} % Имя файла без расширения (файл должен быть расположен в директории inc/img/)
{f} % Обтекание (без обтекания)
{h} % Положение рисунка (см. figure из пакета float)
{0.5\textwidth} % Ширина рисунка
{Выполнение команды $make$ к программе 14-го варианта} % Подпись рисунка

\clearpage

Дизассемблированный код программы 14-го варианта представлен на листинге \ref{lst:var14_disasm.txt}

\includelisting {var14_disasm.txt} {Дизассемблированный код программы для 14-го варианта}

\clearpage
\section{Задание №2}

\subsection*{Условие задания}

В ходе выполнения данного задания необходимо выполнить следующие действия:

\begin{enumerate}[label=\arabic*)]
\item Запустить симуляцию в среде Modelsim. Для этого найти в каталоге taiga файл run.sh (если лабораторная работа выполняется в среде ОС Linux) или run.bat (для ОС Windows) и запустить его двойным щелчком мыши.
\item Запустить симуляцию, набрав в командной строке Modelsim команду run 460us.
\item Изучить список сигналов, приведенных в окне Wave.
\item В соответствии с таблицей, приведенной ниже, получить снимок экрана, содержащий
временную диаграмму выполнения стадий выборки и диспетчеризации команды с
указанным адресом. Для команд, входящих в тело цикла, приведен номер итерации.
\end{enumerate}

Мой вариант: команда с адресом 80000014, 2-я итерация.

\clearpage

\subsection*{Результаты выполнения}

После выполнения пунктов 1-2 запускается симуляция в среде Modelsim, как видно на рисунке \ref{img:win_modelsim}.

\includeimage
{win_modelsim} % Имя файла без расширения (файл должен быть расположен в директории inc/img/)
{f} % Обтекание (без обтекания)
{h} % Положение рисунка (см. figure из пакета float)
{1\textwidth} % Ширина рисунка
{Скриншот запуска симуляции в среде Modelsim} % Подпись рисунка

\clearpage

На рисунке \ref{img:f_id_modelsim}  показан снимок экрана симуляции в среде Modelsim на стадии выборки (29-й такт) и диспетчеризации (30-й такт) команды с адресом 80000014 на 2-й итерации.

\includeimage
{f_id_modelsim} % Имя файла без расширения (файл должен быть расположен в директории inc/img/)
{f} % Обтекание (без обтекания)
{h} % Положение рисунка (см. figure из пакета float)
{1\textwidth} % Ширина рисунка
{Стадии выборки и диспетчеризаци команды с адресом 80000014 на 2-й итерации.} % Подпись рисунка

\clearpage

\section{Задание №3}

\subsection*{Условие задания}
Получить снимок экрана, содержащий
временную диаграмму выполнения стадии декодирования и планирования на выполнение
команды с указанным адресом. Для команд, входящих в тело цикла, приведен номер итерации.

Мой вариант: команда  с адресом 80000020, 2-я итерация.

\subsection*{Результаты выполнения}

На рисунке \ref{img:dec_modelsim}  показан снимок экрана симуляции в среде Modelsim на стадии декодирования (38-й такт) команды с адресом 80000020 на 2-й итерации.

\includeimage
{dec_modelsim} % Имя файла без расширения (файл должен быть расположен в директории inc/img/)
{f} % Обтекание (без обтекания)
{h} % Положение рисунка (см. figure из пакета float)
{1\textwidth} % Ширина рисунка
{Стадии декодирования и планирования на выполнение команды с адресом 80000020 на 2-й итерации.} % Подпись рисунка

\clearpage 

\section{Задание №4}

\subsection*{Условие задания}

Получить снимок экрана, содержащий временную диаграмму выполнения стадии выполнения команды с указанным адресом. Для команд, входящих в тело цикла, приведен номер итерации.

Мой вариант: команда  с адресом 8000000с, 2-я итерация.

\subsection*{Результаты выполнения}

На рисунке \ref{img:impl_modelsim}  показан снимок экрана симуляции в среде Modelsim на стадии выполнения (30-й, 31-й и 32-й такты) команды с адресом 8000000с на 2-й итерации.

\includeimage
{impl_modelsim} % Имя файла без расширения (файл должен быть расположен в директории inc/img/)
{f} % Обтекание (без обтекания)
{h} % Положение рисунка (см. figure из пакета float)
{1\textwidth} % Ширина рисунка
{Стадии декодирования и планирования на выполнение команды с адресом 8000000с на 2-й итерации.} % Подпись рисунка

\clearpage 

\section{Задание №5}

\subsection*{Условие задания}

В процесссе выполнения этого задания необходимо выполнить следующие действия:

\begin{enumerate}[label=\arabic*)]
\item Исправить файл taiga/run.sh или taiga/run.bat так, чтобы там был указан путь к файлу .hex, соответствующему программе по индивидуальному варианту. Сохранить файл.
\item Закрыть Modelsim.
\item Запустить симуляция заново.
\item Получить временную диаграмму сигналов выполнения программы индивидуального варианта.
\item Сравнить значение регистра $x31$ (сигнал $/tb/register\_file[31]$) на момент окончания выполнения программы с тем, который был получен в Задании №1.
\item Получить снимок экрана, содержащий временные диаграммы сигналов, соответствующих всем стадиям выполнения команды, обозначенной в тексте программы символом $\#!$.
\item Анализируя диаграмму заполнить трассу выполнения программы. Рекомендуется использовать для этого файл pipeline.ods, содержащий трассу тестового примера.
\item Сделать вывод об эффективности выполнения программы и о путях оптимизации.
\item Провести оптимизацию программы путем перестановки команд для устранения конфликтов.
\item Перекомпилировать программу и перезапустить симуляцию.
\item Заполнить трассу выполнения оптимизированной программы.
\item Сравнить трассы выполнения неоптимизированной и оптимизированной версии, сделать выводы.
\end{enumerate}

\subsection*{Результаты выполнения}

Узнаем значение, хранящееся в регистре $x31$ по окончании выполнения программы. На рисунке \ref{img:task5_x31_after_execution} видно, что вычисленное в первом задании значение совпадает с хранящимся там.

\includeimage
{task5_x31_after_execution} % Имя файла без расширения (файл должен быть расположен в директории inc/img/)
{f} % Обтекание (без обтекания)
{h} % Положение рисунка (см. figure из пакета float)
{1\textwidth} % Ширина рисунка
{Значение регистра $x31$ после выполнения программы} % Подпись рисунка


Символом $\#!$ помечена команда $add \hspace{0.25cm} x31, x0, x2$ с адресом $8000001c$. На рисунке \ref{img:task5_f_id} представлены стадии выборки (8-й такт) и диспетчеризации (9-й такт), а на рисунке \ref{img:task5_dec_execute} показаны стадии декодирования и планирования на выполнение (12-й такт) и выполнения (13-й такт).

\includeimage
{task5_f_id} % Имя файла без расширения (файл должен быть расположен в директории inc/img/)
{f} % Обтекание (без обтекания)
{h} % Положение рисунка (см. figure из пакета float)
{1\textwidth} % Ширина рисунка
{Стадии выборки и диспетчеризации команды с адресом $8000001c$} % Подпись рисунка

\clearpage

\includeimage
{task5_dec_execute} % Имя файла без расширения (файл должен быть расположен в директории inc/img/)
{f} % Обтекание (без обтекания)
{h} % Положение рисунка (см. figure из пакета float)
{1\textwidth} % Ширина рисунка
{Стадии декодирования и выполнения команды с адресом $8000001c$} % Подпись рисунка

\clearpage 

Трасса выполнения программы представлена на рисунке \ref{img:task5_track}.

\includeimage
{task5_track} % Имя файла без расширения (файл должен быть расположен в директории inc/img/)
{f} % Обтекание (без обтекания)
{h} % Положение рисунка (см. figure из пакета float)
{1\textwidth} % Ширина рисунка
{Трасса выполнения программы 14-го варианта} % Подпись рисунка

Из трассы видно возникновение конфликтов, которые замедляют работу программы. Для оптимизации можно перенести команду $addi \hspace{0.25cm} x1,x1,4$ в место между конфликтующими командами.

Ниже приведены  ассемблерный и дизассемблерный коды оптимизированной программы.

\clearpage 

\includelisting {var14opt_asm.txt} {Исходный код оптимизированной программы для 14-го варианта}

\clearpage 

\includelisting {var14opt_disasm.txt} {Дизассемблированный код оптимизированной программы для 14-го варианта}

\clearpage

Трасса выполнения оптимизированной программы представлена на рисунке \ref{img:task5_track_opt}.

\includeimage
{task5_track_opt} % Имя файла без расширения (файл должен быть расположен в директории inc/img/)
{f} % Обтекание (без обтекания)
{h} % Положение рисунка (см. figure из пакета float)
{1\textwidth} % Ширина рисунка
{Трасса выполнения оптимизированной программы 14-го варианта} % Подпись рисунка

Проанализировав обе трассы, можно увидеть, что после оптимизаций программа стала работать на 8 тактов быстрее.