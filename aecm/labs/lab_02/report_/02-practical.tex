\chapter{Практическая часть}

\section{Эксперимент №1: Исследования расслоения динамической памяти}

\subsection*{Цель эксперимента}

Цель эксперимента -- определение способа трансляции физического адреса,
используемого при обращении к динамической памяти.

\subsection*{Описание проблемы}

В связи с конструктивной неоднородностью оперативной памяти, обращение к последовательно расположенным данным требует различного времени. В связи с этим, для создания эффективных программ необходимо учитывать расслоение памяти, характеризуемое способом трансляции физического адреса. 

\subsection*{Суть эксперимента}

Для определения способа трансляции физического адреса при формировании сигналов выборки банка, выборки строки и столбца запоминающего массива применяется процедура замера времени обращения к динамической памяти по последовательным адресам с изменяющимся шагом чтения. Для сравнения времен
используется обращение к одинаковому количеству различных ячеек, отстоящих друг от друга на определенный шаг. Результат эксперимента представляется зависимостью времени (или количества тактов процессора), потраченного на чтение ячеек, от шага чтения. Для проведения эксперимента необходимо задать изменяемые параметры:

\subsection*{Исходные данные}

\begin{itemize}[label*=--]
	\item размер линейки кэш-памяти верхнего уровня;
	\item объем физической памяти.
\end{itemize}


\subsection*{Результаты эксперимента}

\begin{itemize}[label*=--]
	\item количество банков динамической памяти;
	\item размер одной страницы динамической памяти;
	\item количество страниц в динамической памяти.
\end{itemize}

\subsection*{Проведение эксперимента}

Изменяемые параметры:

\begin{enumerate}[label*=--]
	\item Максимальное расстояние между читаемыми блоками -- 128;
	\item Шаг увеличения расстояния между читаемыми 4-х байтовыми ячейками -- 64;
	\item Размер массива -- 4;
\end{enumerate}

%\includeimage
%{exp1-lystsev} % Имя файла без расширения (файл должен быть расположен в директории inc/img/)
%{f} % Обтекание (без обтекания)
%{h} % Положение рисунка (см. figure из пакета float)
%{1\textwidth} % Ширина рисунка
%{sdfdsfdsdfssfd} % Подпись рисунка

Получим значения следующих величин:
\begin{equation}
	B = \frac{T_1}{M},
\end{equation}
где \textit{B} -- количество банков; $T_1$ -- минимальный шаг чтения
динамической памяти, при котором происходит постоянное обращение к одному и
тому же банку; \textit{M} -- объем данных, являющийся минимальной порцией
обмена кэш-памяти верхнего уровня с оперативной памятью и соответствует размеру линейки кэш-памяти верхнего уровня

А также

\begin{equation}
	S = \frac{T_2}{B},
\end{equation}

где $B$ -- количество банков; $T_2$ -- соответствует расстоянию (в байтах) между началом двух последовательных страниц одного банка; $S$ -- количество секторов.

В результате эксперимента было получено, что $T_1 = 128$ и $T_2 = 65536$, тогда:

\begin{equation}
	B = \frac{T_1}{M} = \frac{128}{128} = 1;
\end{equation}

\begin{equation}
	S = \frac{T_2}{B} = \frac{65536}{1} = 65536.
\end{equation}


