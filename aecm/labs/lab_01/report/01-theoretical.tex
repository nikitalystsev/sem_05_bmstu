\chapter{Основные теоретические сведения}

RISC-V является открытым современным набором команд, который может использоваться для построения как микроконтроллеров, так и высокопроизводительных микропроцессоров. Таким образом, термин RISC-V фактически является названием для семейства различных систем команд, которые строятся вокруг базового набора команд, путем внесения в него различных расширений.

В данной работе исследуется набор команд RV32I, который включает в себя основные команды 32-битной целочисленной арифметики кроме умножения и деления. 

\section{Модель памяти}

Архитектура RV32I предполагает плоское линейное 32-х битное адресное пространство. Минимальной адресуемой единицей информации является 1 байт. Используется порядок байтов от младшего к старшему (Little Endian), то есть, младший байт 32-х битного слова находится по младшему адресу (по смещению 0). Отсутствует разделение на адресные пространства команд, данных и ввода-вывода. Распределение областей памяти между различными устройствами (ОЗУ, ПЗУ, устройства ввода-вывода) определяется реализацией.

\section{Система команд}

Большая часть команд RV32I является трехадресными, выполняющими операции над двумя заданными явно операндами, и сохраняющими результат в регистре. Операндами могут являться регистры или константы, явно заданные в коде команды. Операнды всех команд задаются явно. 

Архитектура RV32I, как и большая часть RISC-архитектур, предполагает разделение команд на команды доступа к памяти (чтение данных из памяти в регистр или запись данных из регистра в память) и команды обработки данных в регистрах.

