\chapter{Функции обработчика прерывания системного \\ таймера в системах разделения времени}

\section{Функции обработчика прерывания системного \\ таймера в семействе ОС Windows}

\subsection*{По кванту}

%Обработчик прерывания системного таймера ставит DPC-вызов в очередь, чтобы инициировать диспетчеризацию потоков, а затем завершить свою работу и понизить IROL-уровень процессора. 

\subsection*{По тику}

\begin{itemize}[label*=--]
	\item декремент кванта;
	\item  инкремент счетчика реального времени;
	\item декремент счетчиков времени до выполнения отложенных задач.
\end{itemize}

\subsection*{По главному тику}

\begin{itemize}[label*=--]
	\item освобождение объекта «событие», которое ожидает диспетчер настройки баланса. Диспетчер настройки баланса сканирует очередь готовых процессов и повышает приоритет процессов, которые находились в состоянии ожидания дольше 4 секунд;
	\item постановка в очередь DPC отложенного вызова обработчика ловушки профилирования ядра.
\end{itemize}

\clearpage

\section{Функции обработчика прерывания системного \\ таймера в семействе OC UNIX/Linux}

\subsection*{По кванту}

\begin{itemize}[label*=--]
	\item посылка текущему процессу сигнала SIGXCPU, если он превысил выделенный для него квант процессорного времени.
\end{itemize}

\subsection*{По тику}

\begin{itemize}[label*=--]
	\item декремент кванта;
	\item инкремент счётчика тиков;
	\item обновление статистики использования процессора текущим процессом; % за счёт изменения значения полей, предназначенных для учёта и оценки времени работы процесса (например, в структуре task\_struct присутствуют поля utime, stime --- время выполнения процесса в режиме задачи и режиме ядра соответственно, prev\_cputime --- структура, используемая для отслеживания предыдущего процессорного времени);
	\item декремент счетчика времени до установки флага для обработчиков отложенных вызовов.
\end{itemize}

\subsection*{По главному тику}

\begin{itemize}[label*=--]
	\item инициализация отложенных действий;
	\item инициализация отложенного вызова процедуры wake\_up, которая перемещает дескрипторы процессов из очереди <<спящих>> в очередь готовых к выполнению;
	\item декремент счетчика времени до отправки дного из сигналов:
	\begin{itemize}[label=•]
		\item SIGALRM (посылается процессу после истечения времени, установленного с помощью функции alarm());
		\item SIGPROF (посылается процессу после истечения времени, установленного в таймере профилирования);
		\item SIGVTALRM (посылается процессу после истечения времени, установленного в  <<виртуальном таймере>>).
	\end{itemize}
\end{itemize}

