\chapter{Исследовательский раздел}

В данном разделе будут проведены сравнения реализаций алгоритмов сортировки  по времени работы и по затрачиваемой памяти.

\section{Технические характеристики}

Технические характеристики устройства, на котором проводились исследования: 

\begin{itemize}[label=--]
	\item операционная система: Ubuntu 22.04.3 LTS x86\_64 \cite{info_os};
	\item оперативная память: 16 Гб;
	\item процессор: 11th Gen Intel® Core™ i7-1185G7 @ 3.00 ГГц × 8, 4 физических ядра, 8 логических ядер.
\end{itemize}

\section{Проведение первого исследования}

\section*{Цель первого исследования}

Целью первого исследования является проведение сравнительного анализа времени работы последовательной версии алгоритма и параллельной версии алгоритма с одним вспомогательным потоком

\section*{Наборы варьируемых и фиксированных параметров}

Замеры времени проводились для случайно сгенерированных файлом с числом строк, равным 10, 20, 30, 40, 50, 60, 70, 80, 90, 100.

В качестве фиксированного параметра было выбрано число $N$, равное 3.

Замеры времени для каждого размера 1000 раз. Время работы алгоритмов измерялось с использованием функции $clock\_gettime$ из встроенного модуля $ctime$.  

\clearpage

\section*{Результаты первого исследования}

\begin{table}[ht]
	\small
	\begin{center}
		\begin{threeparttable}
			\caption{Замер времени для файлов с числом строк от 10 до 100}
			\label{tbl:time}
			\begin{tabular}{|r|r|r|}
				\hline
				& \multicolumn{2}{c|}{\bfseries Время, мкс} \\ \cline{2-3}
				\bfseries \makecell{Число строк в файле, \\ единицы} & \bfseries \makecell{Последовательная \\ версия} & \bfseries \makecell{Параллельная версия \\ с 1-м впомогат. потоком} \\ \cline{2-3}
				\hline
				10 & 435.759 & 882.219  \\
				\hline
				20 & 923.062 & 1 532.931 \\
				\hline
				30 & 1 435.287 & 2 343.806  \\
				\hline
				40 & 2 000.813 & 3 070.613 \\
				\hline
				50 & 2 383.545 & 3 892.355  \\
				\hline
				60 & 3 051.124 & 4 674.078  \\
				\hline
				70 & 3 511.962 & 5 709.621  \\
				\hline
				80 & 4 054.163 & 6 309.628  \\
				\hline
				90 & 4 709.397 & 7 285.999  \\
				\hline
				100 & 5 569.732 & 8 170.710  \\
				\hline
			\end{tabular}	
		\end{threeparttable}
	\end{center}
\end{table}

\clearpage

\includesvgimage
{research1} % Имя файла без расширения (файл должен быть расположен в директории inc/img/)
{f} % Обтекание (без обтекания)
{h} % Положение рисунка (см. figure из пакета float)
{1\textwidth} % Ширина рисунка
{Результаты замеров времени работы алгоритмов для файлов с числом строк от 10 до 100} % Подпись рисунка

Из полученных данных следует, что однопоточный процесс демонстрирует более высокую эффективность по сравнению с процессом, включающим в себя создание вспомогательного потока для обработки всех строк файла. Это объясняется дополнительными временными затратами, связанными с созданием потока и передачей ему необходимых аргументов.

\clearpage

\section{Проведение второго исследования}

\section*{Цель второго исследования}

Целью второго исследования является проведение сравнительного анализа времени работы параллельной версии алгоритма с разным числом потоков.

\section*{Наборы варьируемых и фиксированных параметров}

Замеры времени проводились для числа потоков, равного 1, 2, 4, 8, 16, 34, 64.

В качестве фиксированного параметра было выбрано число $N$, равное 3 и число строк в файле с текстом, равное 100.

Замеры времени для каждого размера 1000 раз. Время работы алгоритмов измерялось с использованием функции $clock\_gettime$ из встроенного модуля $ctime$.  

\section*{Результаты второго исследования}

\begin{table}[ht]
	\small
	\begin{center}
		\begin{threeparttable}
			\caption{Замер времени для числа потоков от 1 до 64}
			\label{tbl:time2}
			\begin{tabular}{|r|r|r|}
				\hline
				& \multicolumn{1}{c|}{\bfseries Время, мкс} \\ \cline{2-2}
				\bfseries \makecell{Число потоков, \\ единицы} & \bfseries \makecell{Параллельная \\ версия} \cline{2-2}
				\hline
				1 & 7 592.842 \\
				\hline
				2 & 7 397.898 \\
				\hline
				4 & 8 306.083  \\
				\hline
				8 & 11 318.340 \\
				\hline
				16 & 11 416.410 \\
				\hline
				32 & 12 302.330 \\
				\hline
				64 & 12 837.810 \\
				\hline
			\end{tabular}	
		\end{threeparttable}
	\end{center}
\end{table}

\clearpage

\includesvgimage
{research2} % Имя файла без расширения (файл должен быть расположен в директории inc/img/)
{f} % Обтекание (без обтекания)
{h} % Положение рисунка (см. figure из пакета float)
{1\textwidth} % Ширина рисунка
{Результаты замеров времени работы алгоритмов для числа потоков от 1 до 64} % Подпись рисунка

Оптимальные результаты по времени достигаются при использовании процесса с двумя дополнительными потоками, ответственными за вычисления. Для данной архитектуры вычислительной машины рекомендуется удерживаться при числе дополнительных потоков, равном 2. При увеличении числа потоков выше этого значения, расходы на поддержание потоков превышают выгоду от использования многопоточности, и функция времени в зависимости от количества потоков начинает увеличиваться.

\clearpage

\section*{Вывод}

Для файла с текстом на русском языке размером в 100 строк:

\begin{itemize}[label*=--]
	\item однопоточный процесс --- 5569.732 мкс;
	\item параллельная версия с 1-м вспомогательным потоком --- 8170.710;
	\item параллельная версия с 2-мя вспомогательными потоками -- 7397.898;
	\item параллельная версия с 64-мя вспомогательными потоками -- 12837.810;
\end{itemize}

Таким образом, для данного алгоритма использование дополнительных потоков приводит лишь к увеличению времени работы программы (даже с 2-мя потоками время работы программы в 1.33 раза больше, чем у однопоточного процесса).Для данной архитектуры вычислительной машины рекомендуется удерживаться при числе дополнительных потоков, равном 2.

