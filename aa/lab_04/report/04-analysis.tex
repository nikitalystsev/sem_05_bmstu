\chapter{Аналитический раздел}

В данном разделе будет приведено теоретическое описание последовательной и параллельной версии алгоритма составления файла словаря с количеством употреблений каждой N-граммы букв из одного слова в тексте на русском языке.

\section{N-граммы}

Пусть задан некоторый конечный алфавит 

\begin{equation}
	\label{eq:alph}
	V = {w_i}
\end{equation} 

где $w_i$ --- символ. 

Языком $L(V)$ называют множество цепочек конечной длины из символов $w_i$.
N-граммой на алфавите $V$ (\ref{eq:alph}) называют произвольную цепочку из $L(V)$ длиной $N$, например последовательность из $N$ букв русского языка одного слова, одной фразы, одного текста или, в более интересном случае, последовательность из грамматически допустимых описаний $N$ подряд стоящих слов \cite{info_ngram}.

\section{Последовательная версия алгоритма}

Алгоритм составления файла словаря с количеством употреблений каждой N-граммы букв из одного слова в тексте на русском языке состоит из следующих шагов:

\begin{enumerate}[label={\arabic*)}]
	\item считывание текста в массив строк;
	\item преобразование считанного текста (перевод букв в нижний регистр, удаление знаков препинания);
	\item обработка каждой строки текста.
\end{enumerate}

Обработка строки текста включает следующие шаги:

\begin{enumerate}[label={\arabic*)}]
	\item обработка каждого слова из строки текста;
	\item выделение существующих в этом слове $N$-грамм;
	\item увеличение количества выделенных $N$-грамм в словаре.
\end{enumerate}

\section{Параллельная версия алгоритма}

В алгоритме составления файла словаря с количеством употреблений каждой $N$-граммы букв из одного слова в тексте на русском языке обработка строк текста происходит независимо, поэтому есть возможность произвести распараллеливание данных вычислений. 

Для этого строки текста поровну распределяются между потоками. Каждый поток получает локальную копию словаря для $N$-грамм, производит вычисления над своим набором строк и после завершения работы всех потоков словари с количеством употреблений $N$-грамм каждого потока объединяются в один. 
Так как каждая строка массива передается в монопольное использование каждому потоку, не возникает конфликтов доступа к разделяемым ячейки памяти,следовательно, в использовании средства синхронизации в виде мьютекса нет необходимости.

\section*{Вывод}

В данном разделе было приведено теоретическое описание последовательной и параллельной версии алгоритма составления файла словаря с количеством употреблений каждой N-граммы букв из одного слова в тексте на русском языке.
