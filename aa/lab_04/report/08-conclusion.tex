\chapter*{ЗАКЛЮЧЕНИЕ}
\addcontentsline{toc}{chapter}{ЗАКЛЮЧЕНИЕ}

В ходе выполнения лабораторной работы были решены следующие задачи:

\begin{enumerate}[label={\arabic*)}]
	\item Изучены и описаны три алгоритма сортировки: блочной, слиянием и поразрядной.
	\item Создано программное обеспечение, реализующее следующие алгоритмы:
	\begin{itemize}[label=--]
		\item алгоритм блочной сортировки;
		\item алгоритм сортировки слиянием;
		\item алгоритм поразрядной сортировки.
	\end{itemize}
	
	\item Проведен анализ эффективности реализаций алгоритмов по памяти и по времени.
	\item Проведена оценка трудоемкости алгоритмов сортировки.
	\item Подготовлен отчет по лабораторной работе.
\end{enumerate}

Цель данной лабораторной работы, а именно исследование трех алгоритмов сортировки: блочной сортировки, сортировки слиянием и поразрядной сортировки, также была достигнута.

Согласно теоретическим расчетам трудоемкости алгоритмов сортировки наименее трудоемким на равномерно распределенных данных оказался алгоритм блочной сортировки, наиболее трудоемким -- алгоритм поразрядной сортировки.

Результаты проведенного исследования практически подтвердили теоретические расчеты трудоемкости: наиболее эффективным по времени работы и по используемой памяти является алгоритм блочной сортировки, но наиболее трудоемким оказался алгоритм сортировки слияением.

На равномерно распределенных данных лучше всего использовать алгоритм блочной сортировки, а для сортировки любых элементов, которые можно поделить на разряды, подошел бы алгоритм поразрядной сортировки. 
