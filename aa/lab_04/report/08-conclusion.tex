\chapter*{ЗАКЛЮЧЕНИЕ}
\addcontentsline{toc}{chapter}{ЗАКЛЮЧЕНИЕ}

В ходе выполнения лабораторной работы были решены следующие задачи:

\begin{enumerate}[label={\arabic*)}]
	\item изучен и описан алгоритм составления файла словаря с количеством употреблений каждой N-граммы букв из одного слова в тексте на русском языке;
	\item разработана последовательная и параллельная версии данного алгоритма;
	\item реализована каждая версия алгоритма;
	\item проведен сравнительный анализ алгоритмов по времени работы реализаций;
	\item обоснованы полученные результаты в отчете к выполненной лабораторной работе.
\end{enumerate}

Цель данной лабораторной работы, изучение принципов и
получение навыков организации параллельного выполнения операций, также была достигнута.

В ходе выполнения лабораторной работы было выявлено, что в результате использования многопоточной реализации время выполнения процесса может как улучшиться, так и ухудшиться в зависимости от количества используемых потоков.

Для рассматриваемого алгоритма использование многопоточности приводит лишь к увеличению времени работы программы (даже с 2-мя потоками время работы программы в 1.33 раза больше, чем у однопоточного процесса). 
Для данной архитектуры вычислительной машины рекомендуется удерживаться при числе дополнительных потоков, равном 2.
