\chapter*{ВВЕДЕНИЕ}
\addcontentsline{toc}{chapter}{ВВЕДЕНИЕ}

Многопоточность --- свойство кода программы выполняться параллельно
(одновременно) на нескольких ядрах процессора или псевдопараллельно
на одном ядре (каждый поток получает в свое распоряжение некоторое время, за
которое он успевает исполнить часть своего кода на процессоре) \cite{info_multithreading}.

Поток (thread) представляет собой независимую последовательность инструкций в
программе. 
В приложениях, которые имеют пользовательский интерфейс, всегда есть
как минимум один главный поток, который отвечает за состояние компонентов
интерфейса. 
Кроме него в программе может создаваться множество независимых
дочерних потоков, которые будут выполняться независимо \cite{info_multithreading}.

Целью данной лабораторной работы является изучение принципов и
получение навыков организации параллельного выполнения операций.

Для достижения поставленной цели необходимо решить следующие задачи:

\begin{enumerate}[label={\arabic*)}]
	\item изучить и описать алгоритм составления файла словаря с количеством употреблений каждой N-граммы букв из одного слова в тексте на русском языке;
	\item разработать последовательную и параллельную версии данного алгоритма;
	\item реализовать каждую версию алгоритма;
	\item провести сравнительный анализ алгоритмов по времени работы реализаций;
	\item обосновать полученные результаты в отчете к выполненной лабораторной работе.
\end{enumerate}
