\chapter*{ВВЕДЕНИЕ}
\addcontentsline{toc}{chapter}{ВВЕДЕНИЕ}

Сортировка -- процесс перегруппировки последовательности объектов в некотором порядке. Это одна из фундаментальных операций в алгоритмике и компьютерных науках, играющая ключевую роль в эффективной обработке данных.

Целью данной лабораторной работы является исследование трех алгоритмов сортировки: блочной сортировки, сортировки слиянием и поразрядной сортировки.

Для достижения поставленной цели необходимо решить следующие задачи:

\begin{enumerate}[label={\arabic*)}]
	\item Изучить и описать три алгоритма сортировки: блочной, слиянием и поразрядной.
	\item Создать программное обеспечение, реализующее следующие алгоритмы:
	\begin{itemize}[label=--]
		\item алгоритм блочной сортировки;
		\item алгоритм сортировки слиянием;
		\item алгоритм поразрядной сортировки.
	\end{itemize}

	\item Провести анализ эффективности реализаций алгоритмов по памяти и по времени.
	\item Провести оценку трудоемкости алгоритмов сортировки.
	\item Обосновать полученные результаты в отчете к выполненной лабораторной работе.
\end{enumerate}
