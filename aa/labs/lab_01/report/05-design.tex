\chapter{Конструкторский раздел}

В данном разделе будут разработаны алгоритмы поиска расстояния Левенштейна и Дамерау-Левенштейна и приведены схемы алгоритмов различных способов их реализации.

\section{Разработка алгоритма поиска расстояния Левенштейна}

\includeimage
{levNotRecur_part1} % Имя файла без расширения (файл должен быть расположен в директории inc/img/)
{f} % Обтекание (без обтекания)
{h} % Положение рисунка (см. figure из пакета float)
{0.85\textwidth} % Ширина рисунка
{Схема нерекурсивного алгоритма нахождения расстояния Левенштейна, первая часть} % Подпись рисунка

\includeimage
{levNotRecur_part2} % Имя файла без расширения (файл должен быть расположен в директории inc/img/)
{f} % Обтекание (без обтекания)
{h} % Положение рисунка (см. figure из пакета float)
{0.3\textwidth} % Ширина рисунка
{Схема нерекурсивного алгоритма нахождения расстояния Левенштейна, вторая часть} % Подпись рисунка

\clearpage

\section{Разработка алгоритма поиска расстояния Дамерау- Левенштейна}

На рисунках \ref{img:damlevNotRecurWithMatrix_part1} и \ref{img:damlevNotRecurWithMatrix_part2}
приведена схема нерекурсивной реализации алгоритма нахождения расстояния Дамерау-Левенштейна.

\includeimage
{damlevNotRecurWithMatrix_part1} % Имя файла без расширения (файл должен быть расположен в директории inc/img/)
{f} % Обтекание (без обтекания)
{h} % Положение рисунка (см. figure из пакета float)
{0.65\textwidth} % Ширина рисунка
{Схема нерекурсивного алгоритма нахождения расстояния Дамерау-Левенштейна, первая часть} % Подпись рисунка


\includeimage
{damlevNotRecurWithMatrix_part2} % Имя файла без расширения (файл должен быть расположен в директории inc/img/)
{f} % Обтекание (без обтекания)
{h} % Положение рисунка (см. figure из пакета float)
{0.7\textwidth} % Ширина рисунка
{Схема нерекурсивного алгоритма нахождения расстояния Дамерау-Левенштейна, вторая часть} % Подпись рисунка

\clearpage

На рисунке \ref{img:damlevRecur} приведена схема рекурсивной реализации алгоритма нахождения расстояния Дамерау-Левенштейна.

\includeimage
{damlevRecur} % Имя файла без расширения (файл должен быть расположен в директории inc/img/)
{f} % Обтекание (без обтекания)
{h} % Положение рисунка (см. figure из пакета float)
{0.91\textwidth} % Ширина рисунка
{Схема рекурсивного алгоритма нахождения расстояния Дамерау-Левенштейна} % Подпись рисунка

\clearpage

На рисунке \ref{img:damLevRecurWithCaching} приведена схема рекурсивной реализации алгоритма нахождения расстояния Дамерау-Левенштейна с кешированием.

\includeimage
{damLevRecurWithCaching} % Имя файла без расширения (файл должен быть расположен в директории inc/img/)
{f} % Обтекание (без обтекания)
{h} % Положение рисунка (см. figure из пакета float)
{0.8\textwidth} % Ширина рисунка
{Схема рекурсивного алгоритма нахождения расстояния Дамерау-Левенштейна с кешированием} % Подпись рисунка

\clearpage

\section*{Вывод}

В данном разделе были перечислены основные требования к программному обеспечению, а также, на основе  данных, полученных в аналитическом разделе, были построены схемы алгоритмов различных способов реализации алгоритмов поиска расстояний Левенштейна и Дамерау-Левенштейна.

