\chapter*{ВВЕДЕНИЕ}
\addcontentsline{toc}{chapter}{ВВЕДЕНИЕ}

Матрица в математике -- таблица чисел, состоящая из определенного количества строк и столбцов.

Умножение матриц -- одна из основных операций над матрицами. Оно используется в различных областях, включая машинное обучение, обработку изображений, и многие другие.

Целью данной лабораторной работы является изучение, описание, реализация и исследование классического алгоритма умножения матриц, умножения матриц с использованием алгоритма Винограда, а также с использованием его оптимизированной версии согласно варианту.

Для достижения поставленной цели необходимо решить следующие задачи:

\begin{enumerate}[label={\arabic*)}]
	\item Изучить и описать алгоритмы классического умножения матриц и умножения матриц с использованием алгоритма Винограда.
	\item Создать программное обеспечение, реализующее следующие алгоритмы:
	\begin{itemize}[label=--]
		\item классический алгоритм умножения матриц;
		\item умножение матриц с использованием алгоритма Винограда;
		\item умножение матриц с использованием оптимизированной версии алгоритма Винограда.
	\end{itemize}

	\item Провести анализ эффективности реализаций алгоритмов по памяти и по времени.
	\item Провести оценку сложности алгоритмов и казать влияние оптимизаций на характеристики программной реализации.
	\item Обосновать полученные результаты в отчете к выполненной лабораторной работе.
\end{enumerate}
