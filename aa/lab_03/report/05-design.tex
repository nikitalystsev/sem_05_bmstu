\chapter{Конструкторский раздел}

\section{Разработка алгоритмов}

\subsection{Алгоритм блочной сортировки}

На рисунке \ref{img:bucketSort} представлена схема алгоритма блочной сортировки.

\includesvgimage
{bucketSort} % Имя файла без расширения (файл должен быть расположен в директории inc/img/)
{f} % Обтекание (без обтекания)
{h} % Положение рисунка (см. figure из пакета float)
{0.5\textwidth} % Ширина рисунка
{Схема алгоритма блочной сортировки} % Подпись рисунка

\clearpage

\subsection{Алгоритм сортировки слиянием}

На рисунке \ref{img:mergeSort} представлена схема алгоритма сортировки слиянием.

На рисунках \ref{img:mergePart1} и \ref{img:mergePart2} представлена схема алгоритма функции слияния двух отсортированных подмассивов.

\includesvgimage
{mergeSort} % Имя файла без расширения (файл должен быть расположен в директории inc/img/)
{f} % Обтекание (без обтекания)
{h} % Положение рисунка (см. figure из пакета float)
{1\textwidth} % Ширина рисунка
{Схема алгоритма сортировки слиянием} % Подпись рисунка

\includesvgimage
{mergePart1} % Имя файла без расширения (файл должен быть расположен в директории inc/img/)
{f} % Обтекание (без обтекания)
{h} % Положение рисунка (см. figure из пакета float)
{1\textwidth} % Ширина рисунка
{Схема алгоритма функции слияния двух отсортированных подмассивов (начало)} % Подпись рисунка

\includesvgimage
{mergePart2} % Имя файла без расширения (файл должен быть расположен в директории inc/img/)
{f} % Обтекание (без обтекания)
{h} % Положение рисунка (см. figure из пакета float)
{0.4\textwidth} % Ширина рисунка
{Схема алгоритма функции слияния двух отсортированных подмассивов (конец)} % Подпись рисунка

\clearpage

\subsection{Алгоритм поразрядной сортировки}

На рисунке \ref{img:radixSort} представлена схема алгоритма сортировки слиянием.

На рисунке \ref{img:countSort} представлена схема алгоритма функции сортировки по определенному разряду.

\includesvgimage
{radixSort} % Имя файла без расширения (файл должен быть расположен в директории inc/img/)
{f} % Обтекание (без обтекания)
{h} % Положение рисунка (см. figure из пакета float)
{0.8\textwidth} % Ширина рисунка
{Схема алгоритма поразрядной сортировки} % Подпись рисунка

\includesvgimage
{countSort} % Имя файла без расширения (файл должен быть расположен в директории inc/img/)
{f} % Обтекание (без обтекания)
{h} % Положение рисунка (см. figure из пакета float)
{0.7\textwidth} % Ширина рисунка
{Схема алгоритма функции сортировки по определенному разряду} % Подпись рисунка

%\section{Оценка трудоемкости алгоритмов}
%
%\subsection{Модель вычислений для проведения оценки трудоемкости алгоритмов}
%
%Была введена модель вычислений для определения трудоемкости каждого отдельного взятого алгоритма сортировки.
%
%\begin{enumerate}[label={\arabic*)}]
%	\item Трудоемкость базовых операций имеет:
%	\begin{itemize}[label=---]i := i + 1
%		\item равную 1:
%		\begin{equation}
%			\label{for:operations_1}
%			\begin{gathered}
%				+, -, =, +=, -=, ==, !=, <, >, <=, >=, [], ++, {-}-,\\
%				\&\&, >>, <<, ||, \&, |
%			\end{gathered}
%		\end{equation}
%		\item равную 2:
%		\begin{equation}
%			\label{for:operations_2}
%			*, /, \%, *=, /=, \%=
%		\end{equation}
%	\end{itemize}
%	\item Трудоемкость условного оператора:
%	\begin{equation}
%		\label{for:if}
%		f_{if} = f_{\text{условия}} + 
%		\begin{cases}
%			min(f_1, f_2), & \text{лучший случай}\\
%			max(f_1, f_2), & \text{худший случай}
%		\end{cases}
%	\end{equation}
%	\item Трудоемкость цикла:
%	\begin{equation}
%		\label{for:for}
%		\begin{gathered}
%			f_{for} = f_{\text{инициализация}} + f_{\text{сравнения}} + M_{\text{итераций}} \cdot (f_{\text{тело}} +\\
%			+ f_{\text{инкремент}} + f_{\text{сравнения}})
%		\end{gathered}
%	\end{equation}
%	\item Трудоемкость передачи параметра в функции и возврат из функции равны 0.
%\end{enumerate}
%
%\clearpage
%
%\subsection{Трудоемкость алгоритма блочной сортировки}
%
%Трудоемкость алгоритма блочной сортировки будет слагаться из:
%
%\begin{itemize}[label=---]
%	\item инициализации пяти переменных, суммарная трудоемкость которых равна 5;
%	\item цикла по $j \in [1 \ldots P]$ , трудоёмкость которого: $f = 2 + 2 + P \cdot (2 + f_{body})$;
%	\item цикла по $k \in [1 \ldots M]$ , трудоёмкость которого: $f = 2 + 2 + 14M$;
%\end{itemize}
%
%Поскольку трудоемкость стандартного алгоритма равна трудоемкости внешнего цикла, то:
%\begin{equation}
%	\label{eq:classic}
%	\begin{gathered}
%		f_{standart} = 2 + N \cdot (2 + 2 + P \cdot (2 + 2 + M \cdot (2 + 8 + 1 + 1 + 2)))= \\
%		= 2 + 4N + 4NP + 14NMP \approx 14NMP = O(N^3)
%	\end{gathered}
%\end{equation}
%
%\subsection{Трудоемкость алгоритма Винограда для умножения двух матриц}
%
%При вычислении трудоемкости алгоритма Винограда учитывается следующее:
%
%\begin{itemize}[label=---]
%	\item создание и инициализация массивов $rowFactor$ и $colFactor$, трудоёмкость которых указана в формуле~(\ref{eq:v_init});
%	\begin{equation}
%		\label{eq:v_init}
%		f_{init} = N + M
%	\end{equation}
%	\item заполнение массива $rowFactor$, трудоёмкость которого указана в формуле~(\ref{eq:v_rowF});
%	\begin{equation}
%		\label{eq:v_rowF}
%		\begin{gathered}
%			f_{rowFactor} = 2 + N \cdot (4 + \frac{M}{2} \cdot (4 + 6 + 1 + 2 + 3 \cdot 2)) = \\
%			= 2 + 4N + \frac{19NM}{2} = 2 + 4N + 9,5NM
%		\end{gathered} 
%	\end{equation}
%	\item заполнение массива $colFactor$, трудоёмкость которого указана в формуле~(\ref{eq:v_colF});
%	\begin{equation}
%		\label{eq:v_colF}
%		\begin{gathered}
%			f_{colFactor} = 2 + P \cdot (4 + \frac{M}{2} \cdot (4 + 6 + 1 + 2 + 3 \cdot 2)) = \\
%			= 2 + 4P + \frac{19PM}{2} = 2 + 4P + 9,5PM
%		\end{gathered}  
%	\end{equation}
%	\item цикл заполнения для чётных размеров, трудоёмкость которого указана в формуле~(\ref{eq:v_cycle});
%	\begin{equation}
%		\label{eq:v_cycle}
%		\begin{gathered}
%			f_{cycle} = 2 + N \cdot (4 + P \cdot (2 + 7 + 4 + \frac{M}{2} \cdot (4 + 28))) = \\
%			= 2 + 4N + 13NP + \frac{32NPM}{2}  = 2 + 4N + 13NP + 16NPM 
%		\end{gathered}
%	\end{equation}
%	\item цикла, который дополнительно нужен для подсчёта значений при нечётном размере матрицы, трудоемкость которого указана в формуле~(\ref{eq:v_check});
%	\begin{equation}
%		\label{eq:v_check}
%		\begin{gathered}
%			f_{check} = 3 + 
%			\begin{cases}
%				0, & \text{чётная} \\
%				2 + M \cdot (4 + P \cdot (2 + 14)), & \text{иначе}
%			\end{cases}
%		\end{gathered}  
%	\end{equation}
%\end{itemize}
%
%Тогда для худшего случая (нечётный общий размер матриц) имеем:
%
%\begin{equation}
%	\label{eq:vinograd_worst}
%	\begin{gathered}
%		f_{worst} = f_{init} + f_{rowFactor} + f_{colFactor} + f_{cycle} + f_{check} \approx 16NMP = O(N^3)
%	\end{gathered}
%\end{equation}
%
%Для лучшего случая (чётный общий размер матриц) имеем:
%
%\begin{equation}
%	\label{eq:vinograd_best}
%	\begin{gathered}
%		f_{best} = f_{init} + f_{rowFactor} + f_{colFactor} + f_{cycle} + f_{check} \approx 16NMP = O(N^3)
%	\end{gathered}
%\end{equation}
%
%\clearpage
%
%\subsection{Трудоемкость оптимизированного алгоритма Винограда для умножения двух матриц}
%
%Трудоемкость оптимизированного алгоритма Винограда состоит из:
%
%\begin{itemize}[label=---]
%	\item кэширования значения $\frac{M}{2}$ в циклах, которое равно 3;
%	\item создания и инициализации массивов $rowFactor$ и $colFactor$ (\ref{eq:v_init});
%	\item заполнения массива $rowFactor$, трудоёмкость которого (\ref{eq:v_rowF});
%	\item заполнения массива $colFactor$, трудоёмкость которого (\ref{eq:v_colF});
%	\item цикла заполнения для чётных размеров, трудоёмкость которого указана в формуле (\ref{сomplexity:v_opt_cycle});
%	\begin{equation}
%		\label{сomplexity:v_opt_cycle}
%		\begin{aligned}
%			f_{cycle} = 2 + N \cdot (4 + P \cdot (4 + 7 + \frac{M}{2} \cdot (2 + 10 + 5 + 2 + 4))) = \\
%			= 2 + 4N + 11NP + \frac{23NPM}{2}  = 2 + 4N + 11NP + 11,5 \cdot NPM 
%		\end{aligned}
%	\end{equation}
%	\item условия, которое нужно для дополнительных вычислений при нечётном размере матрицы, трудоемкость которого указана в формуле~(\ref{сomplexity:v_opt_check});
%	\begin{equation}
%		\label{сomplexity:v_opt_check}
%		\begin{aligned}
%			f_{check} = 3 + 
%			\begin{cases}
%				0, & \text{чётная} \\
%				2 + N \cdot (4 + P \cdot (2 + 10)), & \text{иначе}
%			\end{cases}
%		\end{aligned}  
%	\end{equation}
%\end{itemize}
%
%Тогда для худшего случая (нечётный общий размер матриц) имеем:
%\begin{equation}
%	\label{сomplexity:vinograd_opt_worst}
%	\begin{aligned}
%		f_{worst} = 3 + f_{init} + f_{atmp} + f_{btmp} + f_{cycle} + f_{check} \approx 11NMP = O(N^3)
%	\end{aligned}
%\end{equation}
%
%Для лучшего случая (чётный общий размер матриц) имеем:
%\begin{equation}
%	\label{сomplexity:vinograd_opt_best}
%	\begin{aligned}
%		f_{best} = 3 + f_{init} + f_{rowFactor} + f_{colFactor} \\
%		+ f_{cycle} + f_{check} \approx 11NMP = O(N^3)
%	\end{aligned}
%\end{equation}
%
%\clearpage
%
%\subsection{Трудоемкость алгоритма Штрассена для умножения двух матриц}
%
%Пусть 
%\begin{itemize}[label=---]
%	\item $REC$ -- трудоемкость рекурсивного алгоритма;
%	\item $DIR$ -- трудоемкость прямого решения;
%	\item $DIV$ -- трудоемкость разбиения ввода ($N$) на несколько частей;
%	\item $COM$ -- трудоемкость объединения решений.
%\end{itemize}
%
%Тогда трудоемкость рекурсивного алгоритма считается по следующей формуле:
%
%\begin{equation}
%	\label{eq:rec}
%	REC(N) =
%	\begin{cases}
%		DIR(N), & N \leq N_0\\
%		DIV(N) + \displaystyle\sum_{i=1}^{n} REC(F[i]) + COM(N), & N > N_0
%	\end{cases}
%\end{equation}
%
%где $N$ -- число входных элементов, $N_0$ -- наибольшее число, определяющее тривиальный случай (прямое решение), $n$ -- число рекурсивных вызовов для данного $N$, $F[i]$ -- число входных элементов для данного $i$.
%
%Для расчета трудоемкости алгоритма Штрассена предположим, что размеры переданных матриц -- степени двойки.
%
%Тогда трудоемкость алгоритма Штрассена определяется следующим образом:
%
%\begin{itemize}[label=---]
%	\item Для матрицы, размером $N \leq 2$ трудоемкость определяется как и в случае классического алгоритма умножения матриц, то есть согласно формуле \ref{eq:classic}
%	\item Для матриц размером $N > 2$ определяется так:
%	\begin{enumerate}[label={\arabic*)}]
%		\item Трудоемкость разбиения ввода ($N$) на части. Каждый следующий вызов берется размерность матрицы в 2 раза меньше предыдущей, и происходит создание
%		соответствующих подматриц и заполнение их значениями.
%		\begin{equation}
%			\label{eq:div}
%			\begin{gathered}
%				DIV(N) = 1 + 8 \cdot (3 + \frac{N}{2} \cdot ((3 + \frac{N}{2} \cdot (5 + 2 + 1)) + 2 + 1) = \\ 16 \cdot N^2 + 24 \cdot N + 25
%			\end{gathered}
%		\end{equation}
%		\item Трудоемкость вычисления матриц $M_i, \hspace{0.25cm} i = \overline{1, 7}$ (обозначим ее буквой $G = G(N)$):
%		\begin{equation}
%			\label{eq:G}
%			\begin{gathered}
%				G(N) = 10 \cdot (2 + \frac{N}{2} \cdot (2 + \frac{N}{2} \cdot (8 + 1 + 1) + 1 + 1)) + \\
%				+ 7 \cdot REC(\frac{N}{2})
%			\end{gathered}
%		\end{equation}
%		
%		где, так как $N = 2^k$ и согласно с \ref{eq:recmul}
%		
%		\begin{equation}
%			\begin{gathered}
%				REC(\frac{N}{2}) = REC(2^{k-1}) = 7 \cdot M(2^{k-2}) = \ldots 7^{i-1} M(2^{k-i}) = \ldots \\
%				7^{k-1} M(2^{k-k}) = 7^{k-1}
%			\end{gathered}
%		\end{equation}
%		
%		подставляя $k = \log_2(N)$ получаем, что 
%		
%		\begin{equation}
%			\begin{gathered}
%				REC(\frac{N}{2}) = \frac{N^{\log_2(7)}}{7}
%			\end{gathered}
%		\end{equation}
%		
%		Таким образом, трудоемкость вычисления матриц $M_i, \hspace{0.25cm} i = \overline{1, 7}$ определяется следующей формулой:
%		
%		\begin{equation}
%			\label{eq:Gfinish}
%			\begin{gathered}
%				G(N) = 10 \cdot (10 \cdot (\frac{N}{2})^2 + 4 \cdot \frac{N}{2} + 2) + N^{\log_2(7)} = \\
%				25 \cdot N^2 + 20 \cdot N + 20 + N^{\log_2(7)}
%			\end{gathered}
%		\end{equation}
%		
%		\item Трудоемкость объединения решений, а именно формирование результирующей матрицы из вычисленных матриц $M_i, \hspace{0.25cm} i = \overline{1, 7}$
%		
%		\begin{equation}
%			\label{eq:com}
%			\begin{gathered}
%				COM(N) = 8 \cdot (2 + \frac{N}{2} \cdot (2 + \frac{N}{2} \cdot (8 + 1 + 1) + 1 + 1)) + \\
%				4 \cdot (3 + \frac{N}{2} \cdot ((3 + \frac{N}{2} \cdot (5 + 2 + 1)) + 2 + 1) = \\
%				28 \cdot N^2 + 28 \cdot N + 28
%			\end{gathered}
%		\end{equation}	
%	\end{enumerate}
%	
%	Таким образом, для матриц размером $N > 2$ трудоемкость алгоритма Штрассена согласно \ref{eq:rec} определяется так:
%	
%	\begin{equation}
%		\label{eq:com}
%		\begin{gathered}
%			f_{strassen}(N) = DIV(N) + G(N) + COM(N) = \\ 16 \cdot N^2 + 24 \cdot N + 25 + 25 \cdot N^2 + 20 \cdot N + 20 + N^{\log_2(7)} + \\
%			28 \cdot N^2 + 28 \cdot N + 28 = \\
%			N^{\log_2(7)} + 69 \cdot N^2 + 72 \cdot N + 73 \approx N^{\log_2(7)} = O(N^{\log_2(7)})
%		\end{gathered}
%	\end{equation}
%\end{itemize}
%
%\subsection{Трудоемкость оптимизированного алгоритма Штрассена для умножения двух матриц}
%
%При программной реализации алгоритма Штрассена не нашлось мест для применения предложенных по варианту оптимизаций, поэтому трудоемкость алгоритма Штрассена осталасть такой же, как и в предыдущем пункте.

\clearpage

\section*{Вывод}

В данном разделе были построены схемы алгоритмов классического умножения матриц, умножения матриц с использованием алгоритма Винограда и алгоритма Штрассена. Также были приведены оценки трудоемкости этих алгоритмов.

Согласно расчетам трудоемкости, наиболее эффективным оказался алгоритм Штрассена. Трудоемкость оптимизированной версии алгоритма Винограда в 1.5 раза меньше, чем у его неоптимизированной версии и в 1.27 раз маньше, чем у классического алгоритма. 


