\chapter{Аналитический раздел}

В данном разделе будут рассмотрены алгоритм блочной сортировки, сортировки слиянием и поразрядной сортировки.

\section{Алгоритм блочной сортировки}

Блочная сортировка \cite{_bucket} -- алгоритм сортировки, в котором сортируемые элементы распределяются между конечным числом отдельных блоков так, чтобы все элементы в каждом следующем по порядку блоке были всегда больше (или меньше), чем в предыдущем. Каждый блок затем сортируется отдельно, либо рекурсивно тем же методом, либо другим. Затем элементы помещаются обратно в массив. 

\section{Алгоритм сортировки слиянием}

Сортировка слиянием -- алгоритм сортировки, который упорядочивает списки (или другие структуры данных, доступ к элементам которых можно получать только последовательно, например — потоки) в определённом порядке. Эта сортировка — хороший пример использования принципа <<разделяй и властвуй>> \cite{merge}.

Алгоритм действий в сортировке слиянием:

\begin{enumerate}[label={\arabic*)}]
	\item Сортируемый массив разбивается на две части примерно одинакового размера.
	\item Каждая из получившихся частей сортируется отдельно, например — тем же самым алгоритмом.
	\item Два упорядоченных массива половинного размера соединяются в один.
\end{enumerate}


\section{Алгоритм поразрядной сортировки}

Поразрядная сортировка \cite{radix} -- это алгоритм сортировки, где массив
несколько раз перебирается и элементы перегруппировываются в зависимости от того, какая цифра находится в определённом разряде. После обработки разрядов (всех или почти всех) массив оказывается упорядоченным. При этом разряды могут обрабатываться в противоположных направлениях — от младших к старшим или наоборот.

\section*{Вывод}

В данном разделе были рассмотрены алгоритм блочной сортировки, сортировки слиянием и поразрядной сортировки.
