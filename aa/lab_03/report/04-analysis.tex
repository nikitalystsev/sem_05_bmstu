\chapter{Аналитический раздел}

В данном разделе будут рассмотрены алгоритм блочной сортировки, сортировки слиянием и поразрядной сортировки.

\section{Алгоритм блочной сортировки}

В алгоритме блочной сортировки элементы распределяются между конечным числом отдельных блоков так, чтобы все элементы в каждом следующем по порядку блоке были всегда больше (или меньше), чем в предыдущем. 
Каждый блок затем сортируется отдельно: либо рекурсивно тем же методом, либо другим алгоритмом. Затем элементы помещаются обратно в массив. 

\section{Алгоритм сортировки слиянием}

Алгоритм действий в сортировке слиянием:

\begin{enumerate}[label={\arabic*)}]
	\item сортируемый массив разбивается на две части примерно одинакового размера;
	\item каждая из получившихся частей сортируется отдельно, например — тем же самым алгоритмом;
	\item два упорядоченных массива половинного размера соединяются в один.
\end{enumerate}


\section{Алгоритм поразрядной сортировки}

В алгоритме поразрядной сортировки массив несколько раз перебирается и элементы перераспределяются в зависимости от того, какая цифра находится в определённом разряде. 
После обработки всех разрядов массив оказывается упорядоченным \cite{radix}. 

\section*{Вывод}

В данном разделе были рассмотрены алгоритм блочной сортировки, сортировки слиянием и поразрядной сортировки.
