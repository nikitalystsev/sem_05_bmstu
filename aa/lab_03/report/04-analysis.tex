\chapter{Аналитический раздел}

В данном разделе будут рассмотрены алгоритм блочной сортировки, сортировки слиянием и поразрядной сортировки.

\section{Алгоритм блочной сортировки}

Блочная сортировка -- алгоритм сортировки, в котором сортируемые элементы распределяются между конечным числом отдельных блоков так, чтобы все элементы в каждом следующем по порядку блоке были всегда больше (или меньше), чем в предыдущем. Каждый блок затем сортируется отдельно, либо рекурсивно тем же методом, либо другим. Затем элементы помещаются обратно в массив \cite{bucket_wiki}. 


\section{Алгоритм сортировки слиянием}

Сортировка слиянием -- алгоритм сортировки, который упорядочивает списки (или другие структуры данных, доступ к элементам которых можно получать только последовательно, например — потоки) в определённом порядке. Эта сортировка — хороший пример использования принципа <<разделяй и властвуй>> \cite{merge_wiki}.

Алгоритм действий в сортировке слиянием:

\begin{enumerate}[label={\arabic*)}]
	\item Сортируемый массив разбивается на две части примерно одинакового размера;
	\item Каждая из получившихся частей сортируется отдельно, например — тем же самым алгоритмом;
	\item Два упорядоченных массива половинного размера соединяются в один.
\end{enumerate}


\section{Алгоритм поразрядной сортировки}

Поразрядная сортировка -- алгоритм сортировки, который выполняется за линейное время. Сравнение производится поразрядно: сначала сравниваются значения одного крайнего разряда, и элементы группируются по результатам этого сравнения, затем сравниваются значения следующего разряда, соседнего, и элементы либо упорядочиваются по результатам сравнения значений этого разряда внутри образованных на предыдущем проходе групп, либо переупорядочиваются в целом, но сохраняя относительный порядок, достигнутый при предыдущей сортировке. Затем аналогично делается для следующего разряда, и так до конца \cite{radix_wiki}.

\section*{Вывод}

В данном разделе были рассмотрены алгоритм блочной сортировки, сортировки слиянием и поразрядной сортировки.
