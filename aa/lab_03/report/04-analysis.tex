\chapter{Аналитический раздел}

В данном разделе будут рассмотрены понятия матрицы, умножения двух матриц, классический алгоритм умножения матриц и умножение матриц с помощью алгоритма Винограда.

\section{Матрица}

Матрица -- математический объект, записываемый в виде прямоугольной таблицы элементов кольца или поля (например, целых или комплексных чисел), которая представляет собой совокупность строк и столбцов, на пересечении которых находятся её элементы. Количество строк и столбцов матрицы задают размер матрицы \cite{Mtr}.


\begin{equation*}
	A_{m \times n} =
	\begin{pmatrix}
		a_{11} & a_{12} & \cdots & a_{1n} \\
		a_{21} & a_{22} & \cdots & a_{2n} \\
		\vdots  & \vdots  & \ddots & \vdots  \\
		a_{m1} & a_{m2} & \cdots & a_{mn}
	\end{pmatrix}
\end{equation*}

Для матрицы определены  следующие алгебраические операции: 

\begin{enumerate}[label={\arabic*)}]
	\item Cложение матриц, имеющих один и тот же размер.
	\item Умножение матрицы на число.
	\item Умножение матриц подходящего размера.
\end{enumerate}

Умножение двух матриц (обозначается: $AB$, реже $A \times B$) определяется следующим образом: каждый элемент результирующей матрицы -- это сумма произведений элементов соответствующих строк первой матрицы и столбца второй матрицы. При этом количество столбцов в первой матрице должно совпадать с количеством строк во второй матрице. Операция умножения матриц в общем случае не коммутативна, то есть $AB \neq BA$.

\section{Классический алгоритм умножения двух матриц}

Классический алгоритм умножение двух матриц вытекает из определения умножения двух матриц и реализует формулу \ref{eq:clmul}.

Пусть даны две прямоугольные матрицы $A$ и $B$ размерности $m \times n$ и $n \times q$ соответственно:

\begin{equation*}
A = 
\begin{pmatrix} 
	a_{11} & a_{12} & \cdots & a_{1m} \\
	a_{21} & a_{22} & \cdots & a_{2m} \\ 
	\vdots & \vdots & \ddots & \vdots \\ 
	a_{l1} & a_{l2} & \cdots & a_{lm}
\end{pmatrix},\;\;\;
B =   
\begin{pmatrix} 
	b_{11} & b_{12} & \cdots & b_{1n} \\
	b_{21} & b_{22} & \cdots & b_{2n} \\ 
	\vdots & \vdots & \ddots & \vdots \\ 
	b_{m1} & b_{m2} & \cdots & b_{mn}
\end{pmatrix}.
\end{equation*}

Тогда матрица $C$ размерностью $m \times q$:

\begin{equation*}
C = 
\begin{pmatrix} 
	c_{11} & c_{12} & \cdots & c_{1q} \\
	c_{21} & c_{22} & \cdots & c_{2q} \\ 
	\vdots & \vdots & \ddots & \vdots \\ 
	c_{m1} & c_{m2} & \cdots & c_{mq}
\end{pmatrix} 
\end{equation*}

где элемент результирующей матрицы $c_{ij}$ определяется так:

\begin{equation}
	\label{eq:clmul}
	c_{ij} = \sum_{k=1}^m a_{ik} \cdot b_{kj}
\end{equation}

\section{Алгоритм Винограда для умножения двух матриц}

Алгоритм Винограда \cite{Vin} -- алгоритм умножения квадратных матриц. Анализируя классический алгоритм умножения двух матриц, можно увидеть, что каждый элемент результирующей матрицы представляет собой скалярное произведение соответствующей строки и соответствующего столбца исходной матрицы. 

Рассмотрим 2 вектора: $V = (v_1, v_2, v_3, v_4)$ и $W = (w_1, w_2, w_3, w_4)$.

Их скалярное произведение равно:

\begin{equation}
	\label{eq:vimul}
	V \cdot W = v_1 \cdot w_1 + v_2 \cdot w_2 + v_3 \cdot w_3 + v_3 \cdot w_3
\end{equation}

\clearpage

Это равенство можно переписать в виде:

\begin{equation}
	\label{eq:vinscal}
	\begin{gathered} 
		V \cdot W = (v_1 + w_2) \cdot (v_2 + w_1) + (v_3 + w_4) \cdot (v_4 + w_3) \\
		- v_1 \cdot v_2 - v_3 \cdot v_4 - w_1 \cdot w_2 - w_3 \cdot w_4
	\end{gathered}
\end{equation}


Несмотря на то, что второе выражение требует вычисления большего количества операций, чем стандартный алгоритм: вместо четырех умножений - шесть, а вместо трех сложений - десять, последние слагаемые в формуле \ref{eq:vinscal} допускают предварительную обработку: его части можно вычислить заранее и запомнить для каждой строки первой матрицы и для каждого столбца второй матрицы, что позволит для каждого элемента выполнять лишь два умножения и пять сложений, складывая затем только лишь с 2 предварительно посчитанными суммами соседних элементов текущих строк и столбцов. Из-за того, что операция сложения быстрее операции умножения в ЭВМ, на практике алгоритм должен работать быстрее стандартного.

В случае нечетного значений размера изначальной матрицы следует произвести еще одну операцию - добавление произведения последних элементов соответствующих строк и столбцов.

\section{Оптимизированный алгоритм Винограда для умножения двух матриц}

При программной реализации алгоритма Винограда предлагается выполнить следующие оптимизации: 

\begin{enumerate}[label={\arabic*)}]
	\item Заменить умножение на 2 на побитовый сдвиг влево.
	\item Заменить выражение вида $x = x + k$ на выражение вида $x +\kern-0.25em= k$.
	\item Значение $\frac{Q}{2}$, используемое в циклах расчета предварительных данных, вычислить заранее.
\end{enumerate}

\clearpage

\section*{Вывод}

В данном разделе были рассмотрены понятия матрицы и операции умножения, классического алгоритма умножения матриц и алгоритма умножения матриц с помощью алгоритма Винограда, а также были приведены варинты оптимизаций алгоритма Винограда.
