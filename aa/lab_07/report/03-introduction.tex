\chapter*{ВВЕДЕНИЕ}
\addcontentsline{toc}{chapter}{ВВЕДЕНИЕ}

С развитием компьютерной техники проблема хранения и обработки больших объемов данных становилась все более актуальной.
Возникла необходимость организации хранилища для больших объемов данных, которое предоставляет возможность быстро находить и модифицировать данные. 
Один из способов организации такого хранилища --- двоичные деревья поиска \cite{info_book_bst}. 

Целью данной лабораторной работы является исследование лучших и худших случаев алгоритма поиска целого числа в несбалансированном двоичном дереве поиска (ДДП) и сбалансированном (АВЛ-дереве).

Для достижения поставленной цели необходимо решить следующие задачи:

\begin{enumerate}[label={\arabic*)}]
	\item описать используемый алгоритм поиска;
	\item выбрать средства программной реализации;
	\item реализовать данный алгоритм поиска;
	\item проанализировать алгоритм по количеству сравнений.
\end{enumerate}