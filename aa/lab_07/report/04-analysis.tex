\chapter{Аналитический раздел}

В данном разделе будет рассмотрено понятие двоичного дерева поиска, АВЛ-дерева, было дано описание алгоритма поиска в двоичном дерева поиска.

\section{Двоичное дерево поиска}

Двоичное дерево представляет собой в общем случае неупорядоченный набор узлов, который либо пуст (пустое дерево), либо разбит на три непересекающиеся части:

\begin{itemize}[label*=--]
	\item узел, называемый корнем;
	\item двоичное дерево, называемое левым поддеревом;
	\item двоичное дерево, называемое правым поддеревом.
\end{itemize}

Таким образом, двоичное дерево --- это рекурсивная структура данных.

Каждый узел двоичного дерева можно представить в виде структуры данных, состоящей из следующих полей:

\begin{itemize}[label*=--]
	\item данные, обладающие ключом, по которому их можно идентифицировать;
	\item указатель на левое поддерево;
	\item указатель на правое поддерево;
	\item указатель на родителя (необязательное поле).
\end{itemize}

Значение ключа уникально для каждого узла.

Дерево поиска --- это двоичное дерево, в котором узлы упорядочены определенным образом по значению ключей: для любого узла $X$ значения ключей всех узлов его левого поддерева меньше значения ключа $X$, а значения ключей всех узлов его правого поддерева больше значения ключа $X$ \cite{info_book_bst}.

\section{АВЛ-дерево}

Важной характеристикой двоичного дерева поиска, непосредственно влияющей на скорость поиска данных является коэффициент сбалансированности. 
Коэффициентом сбалансированности называют некоторую константу $k$, на которую могут отличаться высоты левого и правого поддерева любого произвольного узла $X$.

Таким образом АВЛ-дерево --- это двоичное дерево поиска, для которого определен коэффициент сбалансированности k = 1 \cite{info_awl_tree}.

\section{Алгоритм поиска в двоичном дереве поиска}

Процедура поиска узла по ключу заключается в том, что на каждом шаге значение искомого ключа сравнивается со значением ключа рассматриваемого узла, начиная с корня. 
Если значение искомого ключа меньше, чем значение ключа рассматриваемого узла, то поиск продолжается в левом поддереве, если больше — то в правом поддереве. 
И так, пока не будет найден узел с искомым ключом или пока поиск не достигнет того узла, ниже которого этот узел не может находиться. 
Если при поиске мы обнаруживаем, что узел далее надо искать, например, в правом
поддереве, а оно пусто, следовательно, мы можем сделать вывод, что искомого ключа в дереве нет \cite{info_book_bst}.

\section*{Вывод}

В данном разделе было рассмотрено понятие двоичного дерева поиска, АВЛ-дерева, было дано описание алгоритма поиска в двоичном дерева поиска.