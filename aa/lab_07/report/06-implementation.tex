\chapter{Технологический раздел}

В данном разделе будут перечислены требования к программному обеспечению, средства реализации и листинги кода.

\section{Требования к программному обеспечению}

К программе предъявляется ряд требований:

\begin{itemize} [label=--]
	\item наличие меню для взаимодействия с программой;
	\item предоставление интерфейса для выбора файла, содержащего данные для построения ДДП и АВЛ-дерева, для ввода количества узлов в дереве и генерации нового файла со значениями узлов;
	\item предоставление интерфейса для осуществления поиска целого числа в ДДП и АВЛ-дереве.
\end{itemize}

\section{Средства реализации}

В качестве языка программирования для этой лабораторной работы был выбран $C++$ \cite{pl}, поскольку в нем есть встроенный модуль $ctime$, предоставляющий необходимый функционал для замеров процессорного времени:

В качестве функции, которая будет осуществлять замеры процессорного времени, будет использована функция $clock\_gettime$ из встроенного модуля $ctime$ \cite{cpu_time_func}.

\section{Сведения о модулях программы}

Программа состоит из пяти модулей: 

\begin{enumerate}[label={\arabic*)}]
	\item \texttt{main.cpp} --- файл, содержащий точку входа в программу;
	\item \texttt{interface.cpp} --- модуль, содержащий функции для взаимодействия с пользователем;

\clearpage

	\item \texttt{measurements.cpp} --- модуль, содержащий функции для проведения замеров количества сравнений в алгоритме поиска;
	\item \texttt{tree.cpp} --- модуль, содержащий описание бинарного дерева и набора функций для работы с бинарным деревом.
\end{enumerate}

\clearpage

\section{Реализации алгоритмов}

%В листингах \ref{lst:serialConveyorPart1.txt} и \ref{lst:serialConveyorPart2.txt} представлена реализация последовательной конвейерной обработки.
%
%\includelisting
%{serialConveyorPart1.txt} % Имя файла с расширением (файл должен быть расположен в директории inc/lst/)
%{Реализация последовательной конвейерной обработки (начало)} % Подпись листинга
%
%\clearpage
%
%\includelisting
%{serialConveyorPart2.txt} % Имя файла с расширением (файл должен быть расположен в директории inc/lst/)
%{Реализация последовательной конвейерной обработки (конец)} % Подпись листинга
%
%\clearpage
%
%В листингах \ref{lst:parallelConveyorPart1.txt} и \ref{lst:parallelConveyorPart2.txt} представлена реализация параллельной конвейерной обработки.
%
%\includelisting
%{parallelConveyorPart1.txt} % Имя файла с расширением (файл должен быть расположен в директории inc/lst/)
%{Реализация основного потока для конвейерной обработки, создающий вспомогательные потоки} % Подпись листинга
%
%\clearpage
%
%\includelisting
%{parallelConveyorPart2.txt} % Имя файла с расширением (файл должен быть расположен в директории inc/lst/)
%{Реализация функции вспомогательного потока, считывающего тексты из файлов} % Подпись листинга
%
%\includelisting
%{parallelConveyorPart3.txt} % Имя файла с расширением (файл должен быть расположен в директории inc/lst/)
%{Реализация функции вспомогательного потока, выполняющая параллельную версию алгоритма составления файла словаря с количеством употреблений каждой $N$-граммы букв из одного слова в тексте на русском языке в 2 потока (начало)} % Подпись листинга
%
%\includelisting
%{parallelConveyorPart4.txt} % Имя файла с расширением (файл должен быть расположен в директории inc/lst/)
%{Реализация функции вспомогательного потока, выполняющая параллельную версию алгоритма составления файла словаря с количеством употреблений каждой $N$-граммы букв из одного слова в тексте на русском языке в 2 потока (конец)} % Подпись листинга
%
%\includelisting
%{parallelConveyorPart5.txt} % Имя файла с расширением (файл должен быть расположен в директории inc/lst/)
%{Реализация функции вспомогательного потока, выполняющая логирование заявки в файл (начало)} % Подпись листинга
%
%\includelisting
%{parallelConveyorPart6.txt} % Имя файла с расширением (файл должен быть расположен в директории inc/lst/)
%{Реализация функции вспомогательного потока, выполняющая логирование заявки в файл (конец)} % Подпись листинга

\section*{Вывод}

В данном разделе были перечислены требования к программному обеспечению, средства реализации и листинги кода.

    