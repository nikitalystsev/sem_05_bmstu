\chapter{Технологический раздел}

В данном разделе будут перечислены требования к программному обеспечению, средства реализации и листинги кода.

\section{Требования к программному обеспечению}

К программе предъявляется ряд требований:

\begin{itemize} [label=--]
	\item наличие меню для взаимодействия с программой;
	\item предоставление интерфейса для выбора файла, содержащего данные для построения ДДП и АВЛ-дерева, для ввода количества узлов в дереве и генерации нового файла со значениями узлов;
	\item предоставление интерфейса для осуществления поиска целого числа в ДДП и АВЛ-дереве.
\end{itemize}

\section{Средства реализации}

В качестве языка программирования для этой лабораторной работы был выбран $C++$ \cite{pl}, поскольку в нем есть встроенный модуль $ctime$ \cite{cpu_time_func}, предоставляющий необходимый функционал для замеров процессорного времени.

\section{Сведения о модулях программы}

Программа состоит из четырех модулей: 

\begin{enumerate}[label={\arabic*)}]
	\item \texttt{main.cpp} --- файл, содержащий точку входа в программу;
	\item \texttt{interface.cpp} --- модуль, содержащий функции для взаимодействия с пользователем;
	\item \texttt{measurements.py} --- модуль, содержащий функции для проведения замеров количества сравнений в алгоритме поиска;
	\item \texttt{tree.cpp} --- модуль, содержащий описание бинарного дерева и набора функций для работы с бинарным деревом.
\end{enumerate}

\clearpage

\section{Реализации алгоритмов}

В листинге \ref{lst:tree.txt} представлены структуры <<$tree\_t$>> и  <<$vertex\_t$>>, содержащие описание бинарного дерева и одной вершины бинарного дерева соответственно.

\includelisting
{tree.txt} % Имя файла с расширением (файл должен быть расположен в директории inc/lst/)
{Реализация структур данных, описывающих бинарное дерево} % Подпись листинга

В листингах \ref{lst:search-tree-part1.txt} и \ref{lst:search-tree-part2.txt} представлена реализация функции поиска в ДДП и АВЛ-дереве.

\includelisting
{search-tree-part1.txt} % Имя файла с расширением (файл должен быть расположен в директории inc/lst/)
{Реализация функции поиска в ДДП и АВЛ-дереве (начало)} % Подпись листинга

\includelisting
{search-tree-part2.txt} % Имя файла с расширением (файл должен быть расположен в директории inc/lst/)
{Реализация функции поиска в ДДП и АВЛ-дереве (конец)} % Подпись листинга

\section*{Вывод}

В данном разделе были перечислены требования к программному обеспечению, средства реализации и листинги кода.

    