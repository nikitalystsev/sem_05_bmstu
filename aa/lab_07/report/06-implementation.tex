\chapter{Технологический раздел}

В данном разделе будут перечислены требования к программному обеспечению, средства реализации и листинги кода.

\section{Требования к программному обеспечению}

К программе предъявляется ряд требований:

\begin{itemize} [label=--]
	\item наличие меню для выбора запускаемого режима работы конвейера --- последовательного и параллельного;
	\item предоставление интерфейса для ввода количества заявок, числа строк в файлах, длины строки в файле и числа $N$;
	\item формирование файла с логом работы конвейера, логирование событий обработки должно происходить после окончания работы, собственно, конвейера.
\end{itemize}

\section{Средства реализации}

В качестве языка программирования для этой лабораторной работы был выбран $C++$ \cite{pl} по следующим причинам:

\begin{itemize}[label=--]
	\item в $C++$ есть встроенный модуль $ctime$, предоставляющий необходимый функционал для замеров процессорного времени;
	\item в $C++$ есть встроенный модуль $thread$ \cite{info_thread}, предоставляющий необходимый интерфейс для работы с нативными потоками.
\end{itemize}

В качестве функции, которая будет осуществлять замеры процессорного времени, будет использована функция $clock\_gettime$ из встроенного модуля $ctime$ \cite{cpu_time_func}.

\section{Сведения о модулях программы}

Программа состоит из семи модулей: 

\begin{enumerate}[label={\arabic*)}]
	\item \texttt{lab\_04\_code.cpp} --- модуль, хранящий реализации параллельной версии алгоритма составления файла словаря с количеством употреблений каждой $N$-граммы букв из одного слова в тексте на русском языке;
	\item \texttt{conveyor.cpp} --- модуль, хранящий реализации параллельной версии алгоритма работы конвейера;
	\item \texttt{serial.cpp} --- модуль, хранящий реализации последовательной версии алгоритма работы конвейера;
	\item \texttt{processTime.cpp} --- модуль, содержащий функцию для замера процессорного времени;
	\item \texttt{timeMeasurements.cpp} --- модуль, содержащий функции, позволяющие провести сравнительный анализ использования времени;
	\item \texttt{main.cpp} --- файл, содержащий точку входа в программу;
	\item \texttt{task7} --- модуль, содержащий набор скриптов для проведения замеров программы по времени и памяти и построения графиков по полученным данным.
\end{enumerate}

\clearpage

\section{Реализации алгоритмов}

В листингах \ref{lst:serialConveyorPart1.txt} и \ref{lst:serialConveyorPart2.txt} представлена реализация последовательной конвейерной обработки.

\includelisting
{serialConveyorPart1.txt} % Имя файла с расширением (файл должен быть расположен в директории inc/lst/)
{Реализация последовательной конвейерной обработки (начало)} % Подпись листинга

\clearpage

\includelisting
{serialConveyorPart2.txt} % Имя файла с расширением (файл должен быть расположен в директории inc/lst/)
{Реализация последовательной конвейерной обработки (конец)} % Подпись листинга

\clearpage

В листингах \ref{lst:parallelConveyorPart1.txt} и \ref{lst:parallelConveyorPart2.txt} представлена реализация параллельной конвейерной обработки.

\includelisting
{parallelConveyorPart1.txt} % Имя файла с расширением (файл должен быть расположен в директории inc/lst/)
{Реализация основного потока для конвейерной обработки, создающий вспомогательные потоки} % Подпись листинга

\clearpage

\includelisting
{parallelConveyorPart2.txt} % Имя файла с расширением (файл должен быть расположен в директории inc/lst/)
{Реализация функции вспомогательного потока, считывающего тексты из файлов} % Подпись листинга

\includelisting
{parallelConveyorPart3.txt} % Имя файла с расширением (файл должен быть расположен в директории inc/lst/)
{Реализация функции вспомогательного потока, выполняющая параллельную версию алгоритма составления файла словаря с количеством употреблений каждой $N$-граммы букв из одного слова в тексте на русском языке в 2 потока (начало)} % Подпись листинга

\includelisting
{parallelConveyorPart4.txt} % Имя файла с расширением (файл должен быть расположен в директории inc/lst/)
{Реализация функции вспомогательного потока, выполняющая параллельную версию алгоритма составления файла словаря с количеством употреблений каждой $N$-граммы букв из одного слова в тексте на русском языке в 2 потока (конец)} % Подпись листинга

\includelisting
{parallelConveyorPart5.txt} % Имя файла с расширением (файл должен быть расположен в директории inc/lst/)
{Реализация функции вспомогательного потока, выполняющая логирование заявки в файл (начало)} % Подпись листинга

\includelisting
{parallelConveyorPart6.txt} % Имя файла с расширением (файл должен быть расположен в директории inc/lst/)
{Реализация функции вспомогательного потока, выполняющая логирование заявки в файл (конец)} % Подпись листинга

\section*{Вывод}

В данном разделе были перечислены требования к программному обеспечению, средства реализации и листинги кода.

    