\chapter*{ЗАКЛЮЧЕНИЕ}
\addcontentsline{toc}{chapter}{ЗАКЛЮЧЕНИЕ}

В ходе выполнения лабораторной работы были решены следующие задачи:

\begin{enumerate}[label={\arabic*)}]
	\item изучена и описана организация конвейерной обработки данных;
	\item описаны алгоритмы обработки данных, которые использовались в
	текущей лабораторной работе;
	\item определены средства программной реализации;
	\item реализована программа, выполняющая конвейерную обработку с количеством лент не менее трех в однопоточной и многопоточной реализаций;
	\item сравнены и проанализированы реализации алгоритмов по затраченному времени.
\end{enumerate}

Цель данной лабораторной работы, а именно описание параллельных конвейерных вычислений, также была достигнута.

Для рассматриваемого алгоритма использование конвейерной обработки для обработки заявок приводит лишь к увеличению времени работы программы по сравнению с последовательной обработкой заявок: начиная с числа заявок, равного 90, наблюдается увеличение времени работы конвейера по сравнению с последовательным алгоритмом обработки заявок (на 90 заявках параллельный конвейер работает в 1.05 медленнее последовательного, а на 100 заявках параллельный конвейер работает уже в 1.14 медленнее последовательного).