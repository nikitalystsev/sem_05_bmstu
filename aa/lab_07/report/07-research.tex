\chapter{Исследовательский раздел}

В данном разделе будут проведены сравнения реализаций алгоритмов сортировки  по времени работы и по затрачиваемой памяти.

\section{Технические характеристики}

Технические характеристики устройства, на котором проводились исследования: 

\begin{itemize}[label=--]
	\item операционная система: Ubuntu 22.04.3 LTS x86\_64 \cite{info_os};
	\item оперативная память: 16 Гб;
	\item процессор: 11th Gen Intel® Core™ i7-1185G7 @ 3.00 ГГц × 8, 4 физических ядра, 8 логических ядер.
\end{itemize}

\section{Демонстрация работы программы}

На рисунке \ref{img:ex1} представлен пример результата работы программы. 

\includeimage
{ex1} % Имя файла без расширения (файл должен быть расположен в директории inc/img/)
{f} % Обтекание (без обтекания)
{h} % Положение рисунка (см. figure из пакета float)
{0.7\textwidth} % Ширина рисунка
{Пример работы программы} % Подпись рисунка	

\clearpage
На рисунке \ref{img:ex2} представлен пример фрагмента лог-файла работы конвейера. 

\includeimage
{ex2} % Имя файла без расширения (файл должен быть расположен в директории inc/img/)
{f} % Обтекание (без обтекания)
{h} % Положение рисунка (см. figure из пакета float)
{0.7\textwidth} % Ширина рисунка
{Пример фрагмента лог-файла работы конвейера} % Подпись рисунка	

\section{Проведение исследования}

\section*{Цель исследования}

Целью исследования является проведение сравнительного анализа времени работы последовательного и параллельного алгоритма работы конвейера от числа поступающих заявок.

\section*{Наборы варьируемых и фиксированных параметров}

Замеры времени проводились для числа заявок, равному 10, 20, 30, 40, 50, 60, 70, 80, 90, 100.

В качестве фиксированного параметра было выбрано число $N$, равное 3, число строк в файле, равное 100 и длина строки, равная 100 символам.

Замеры времени для каждого размера 1000 раз. Время работы алгоритмов измерялось с использованием функции $clock\_gettime$ из встроенного модуля $ctime$ \cite{cpu_time_func}.  

\section*{Результаты первого исследования}

\begin{table}[ht]
	\small
	\begin{center}
		\begin{threeparttable}
			\caption{Замер времени для конвейера с числом заявок от 10 до 100}
			\label{tbl:time}
			\begin{tabular}{|r|r|r|}
				\hline
				& \multicolumn{2}{c|}{\bfseries Время, мкс} \\ \cline{2-3}
				\bfseries \makecell{Число заявок, \\ единицы} & \bfseries \makecell{Последовательная \\ версия} & \bfseries \makecell{Параллельная \\ версия} \\ \cline{2-3}
				\hline
				10 & 158 528.300 & 182 195.500 \\
				\hline
				20 & 295 954.100 & 327 198.500 \\
				\hline
				30 & 522 347.500 & 597 439.000  \\
				\hline
				40 & 768 368.400 & 804 989.700 \\
				\hline
				50 & 971 115.000 & 914 300.400  \\
				\hline
				60 & 1 084 875.000 & 1 209 416.000  \\
				\hline
				70 & 1 370 885.000 & 1 380 609.000  \\
				\hline
				80 & 1 473 052.000 & 1 378 848.000  \\
				\hline
				90 & 1 613 111.000 & 1 691 645.000  \\
				\hline
				100 & 1 871 918.000 & 2 125 750.000  \\
				\hline
			\end{tabular}	
		\end{threeparttable}
	\end{center}
\end{table}

\clearpage

\includesvgimage
{research1} % Имя файла без расширения (файл должен быть расположен в директории inc/img/)
{f} % Обтекание (без обтекания)
{h} % Положение рисунка (см. figure из пакета float)
{1\textwidth} % Ширина рисунка
{Результаты замеров времени для конвейера с числом заявок от 10 до 100} % Подпись рисунка

Из полученных данных следует, что результаты использования конвейерной обработки не совпали с ожидаемыми: начиная с числа заявок, равного 90, наблюдается увеличение времени работы конвейера по сравнению с последовательным алгоритмом обработки заявок (на 90 заявках параллельный конвейер работает в 1.05 медленнее последовательного, а на 100 заявках параллельный конвейер работает уже в 1.14 медленнее последовательного). Вероятно, это связано с тем, что значительная часть работы уходит на создание и обслуживание вспомогательных потоков в конвейерной обработке.

\clearpage

\section*{Вывод}

В результате исследования было установлено, что использование конвейерной обработки  для обработки заявок оказалось менее эффективно, чем последовательная обработка заявок: начиная с числа заявок, равного 90, наблюдается увеличение времени работы конвейера по сравнению с последовательным алгоритмом обработки заявок (на 90 заявках параллельный конвейер работает в 1.05 медленнее последовательного, а на 100 заявках параллельный конвейер работает уже в 1.14 медленнее последовательного). Это связано с тем, что значительная часть работы уходит на создание и обслуживание вспомогательных потоков в конвейерной обработке.

