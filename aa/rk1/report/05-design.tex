\chapter{Конструкторский раздел}

\section{Разработка алгоритмов}

\subsection{Последовательная версия алгоритма}

На рисунках \ref{img:serialVersionPart1} и \ref{img:serialVersionPart2} представлена схема последовательной версии алгоритма составления файла словаря с количеством употреблений каждой N-граммы букв из одного слова в тексте на русском языке.

%\includesvgimage
%{serialVersionPart1} % Имя файла без расширения (файл должен быть расположен в директории inc/img/)
%{f} % Обтекание (без обтекания)
%{h} % Положение рисунка (см. figure из пакета float)
%{0.6\textwidth} % Ширина рисунка
%{Схема алгоритма для обработки целого текста} % Подпись рисунка	
%
%\includesvgimage
%{serialVersionPart2} % Имя файла без расширения (файл должен быть расположен в директории inc/img/)
%{f} % Обтекание (без обтекания)
%{h} % Положение рисунка (см. figure из пакета float)
%{1\textwidth} % Ширина рисунка
%{Схема алгоритма для обработки строки текста} % Подпись рисунка	

\clearpage

\subsection{Параллельная версия алгоритма}


На рисунках \ref{img:parallelVersionPart1} и \ref{img:parallelVersionPart2} представлена схема параллельной версии алгоритма составления файла словаря с количеством употреблений каждой N-граммы букв из одного слова в тексте на русском языке.

%\includesvgimage
%{parallelVersionPart1} % Имя файла без расширения (файл должен быть расположен в директории inc/img/)
%{f} % Обтекание (без обтекания)
%{h} % Положение рисунка (см. figure из пакета float)
%{0.6\textwidth} % Ширина рисунка
%{Схема алгоритма для обработки целого текста} % Подпись рисунка	
%
%\includesvgimage
%{parallelVersionPart2} % Имя файла без расширения (файл должен быть расположен в директории inc/img/)
%{f} % Обтекание (без обтекания)
%{h} % Положение рисунка (см. figure из пакета float)
%{1\textwidth} % Ширина рисунка
%{Схема алгоритма функции для одного потока} % Подпись рисунка	


\clearpage

\section*{Вывод}

В данном разделе были построены схемы последовательной и параллельной версии алгоритма составления файла словаря с количеством употреблений каждой N-граммы букв из одного слова в тексте на русском языке.



