\chapter{Конструкторский раздел}

\section{Разработка алгоритмов}

\subsection{Последовательный алгоритм работы стадий конвейера}

На рисунке \ref{img:serialConveyor} представлен последовательный алгоритм работы стадий
конвейера.

\includesvgimage
{serialConveyor} % Имя файла без расширения (файл должен быть расположен в директории inc/img/)
{f} % Обтекание (без обтекания)
{h} % Положение рисунка (см. figure из пакета float)
{0.6\textwidth} % Ширина рисунка
{Схема последовательного алгоритма работы стадий конвейера} % Подпись рисунка	
%
%\includesvgimage
%{serialVersionPart2} % Имя файла без расширения (файл должен быть расположен в директории inc/img/)
%{f} % Обтекание (без обтекания)
%{h} % Положение рисунка (см. figure из пакета float)
%{1\textwidth} % Ширина рисунка
%{Схема алгоритма для обработки строки текста} % Подпись рисунка	

\clearpage

\subsection{Параллельный алгоритм работы стадий конвейера}
Параллельная работа будет реализована с помощью добавления 3-х вспомогательных потоков, где каждый поток отвечает за свою стадию обработки.

Вспомогательному потоку в числе аргументов в качестве структуры будут переданы:

\begin{itemize}[label*=--]
	\item имя файла с исходным текстом на русском языке;
	\item имя выходного файла для словаря $N$-грамм;
	\item число $N$.
	\item временные отметки начала и конца выполнения каждой стадии обработки заявки.
\end{itemize}

На рисунке \ref{img:parallelConveyorPart1} представлена схема алгоритма работы главного потока при параллельной работе стадий конвейера.

На рисунках \ref{img:parallelConveyorPart2} – \ref{img:parallelConveyorPart4} представлены схемы алгоритмов каждого из обработчиков (потоков) при параллельной работе.

\includesvgimage
{parallelConveyorPart1} % Имя файла без расширения (файл должен быть расположен в директории inc/img/)
{f} % Обтекание (без обтекания)
{h} % Положение рисунка (см. figure из пакета float)
{1\textwidth} % Ширина рисунка
{Схема параллельного алгоритма работы стадий конвейера} % Подпись рисунка	

\includesvgimage
{parallelConveyorPart2} % Имя файла без расширения (файл должен быть расположен в директории inc/img/)
{f} % Обтекание (без обтекания)
{h} % Положение рисунка (см. figure из пакета float)
{1\textwidth} % Ширина рисунка
{Схема алгоритма потока, выполняющего чтение текстов из файлов} % Подпись рисунка	

\includesvgimage
{parallelConveyorPart3} % Имя файла без расширения (файл должен быть расположен в директории inc/img/)
{f} % Обтекание (без обтекания)
{h} % Положение рисунка (см. figure из пакета float)
{0.8\textwidth} % Ширина рисунка
{Схема алгоритма потока, выполняющего составление файлов-словарей с $N$-граммами} % Подпись рисунка	

\includesvgimage
{parallelConveyorPart4} % Имя файла без расширения (файл должен быть расположен в директории inc/img/)
{f} % Обтекание (без обтекания)
{h} % Положение рисунка (см. figure из пакета float)
{0.9\textwidth} % Ширина рисунка
{Схема алгоритма потока, выполняющего логирование заявок в файл} % Подпись рисунка	

\clearpage

\section*{Вывод}

В данном разделе были представлены схемы последовательной и параллельной работы стадий конвейера.



