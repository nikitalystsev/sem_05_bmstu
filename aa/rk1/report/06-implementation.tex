\chapter{Технологический раздел}

В данном разделе будут перечислены требования к программному обеспечению, средства реализации и листинги кода.

\section{Требования к программному обеспечению}

К программе предъявляется ряд требований:

\begin{itemize} [label=--]
	\item на вход имя файла с текстом на русском языке, имя файла-словаря, число $N$ --- $N$-грамма;
	\item в программе для распараллеливания вычислений используются только нативные потоки;
	\item в результате работы программы получаем файл-словарь с количеством употреблений каждой $N$-граммы букв из одного слова в тексте на русском языке.
\end{itemize}

\section{Средства реализации}

В качестве языка программирования для этой лабораторной работы был выбран $C++$ \cite{pl} по следующим причинам:

\begin{itemize}[label=--]
	\item в $C++$ есть встроенный модуль $ctime$, предоставляющий необходимый функционал для замеров процессорного времени;
	\item в $C++$ есть встроенный модуль $thread$ \cite{info_thread}, предоставляющий необходимый интерфейс для работы с нативными потоками.
\end{itemize}

В качестве функции, которая будет осуществлять замеры процессорного времени, будет использована функция $clock\_gettime$ из встроенного модуля $ctime$ \cite{cpu_time_func}.

\section{Сведения о модулях программы}

Программа состоит из пяти модулей: 

\begin{enumerate}[label={\arabic*)}]
	\item \texttt{algorithms.cpp} --- модуль, хранящий реализации последовательной и параллельной версии алгоритма составления файла словаря с количеством употреблений каждой $N$-граммы букв из одного слова в тексте на русском языке;
	\item \texttt{processTime.cpp} --- модуль, содержащий функцию для замера процессорного времени;
	\item \texttt{timeMeasurements.cpp} --- модуль, содержащий функции, позволяющие провести сравнительный анализ использования времени;
	\item \texttt{main.cpp} --- файл, содержащий точку входа в программу;
	\item \texttt{task7} --- модуль, содержащий набор скриптов для проведения замеров программы по времени и памяти и построения графиков по полученным данным.
\end{enumerate}

\clearpage

\section{Реализации алгоритмов}

В листингах \ref{lst:serialVersionPart1.txt} -- \ref{lst:serialVersionPart3.txt} представлена последовательная версия алгоритма составления файла словаря с количеством употреблений каждой $N$-граммы букв из одного слова в тексте на русском языке.

%\includelisting
%{serialVersionPart1.txt} % Имя файла с расширением (файл должен быть расположен в директории inc/lst/)
%{Реализация функции вычисления N-грамм в слове и построчного считывания текста из файла с преобразованиями} % Подпись листинга
%
%\clearpage
%
%\includelisting
%{serialVersionPart2.txt} % Имя файла с расширением (файл должен быть расположен в директории inc/lst/)
%{Реализация функции обработки строки} % Подпись листинга
%
%\clearpage
%
%\includelisting
%{serialVersionPart3.txt} % Имя файла с расширением (файл должен быть расположен в директории inc/lst/)
%{Реализация функции обработки всего текста} % Подпись листинга

\clearpage

В листингах \ref{lst:parallelVersionPart1.txt} -- \ref{lst:serialVersionPart3.txt} представлена последовательная версия алгоритма составления файла словаря с количеством употреблений каждой $N$-граммы букв из одного слова в тексте на русском языке.

%\includelisting
%{parallelVersionPart1.txt} % Имя файла с расширением (файл должен быть расположен в директории inc/lst/)
%{Реализация функции вычисления N-грамм в слове и построчного считывания текста из файла с преобразованиями} % Подпись листинга
%
%\clearpage
%
%\includelisting
%{parallelVersionPart2.txt} % Имя файла с расширением (файл должен быть расположен в директории inc/lst/)
%{Реализация функции обработки строк для каждого потока} % Подпись листинга
%
%\clearpage
%
%\includelisting
%{parallelVersionPart3.txt} % Имя файла с расширением (файл должен быть расположен в директории inc/lst/)
%{Реализация функции многопоточной обработки всего текста (часть 1)} % Подпись листинга
%
%\includelisting
%{parallelVersionPart4.txt} % Имя файла с расширением (файл должен быть расположен в директории inc/lst/)
%{Реализация функции многопоточной обработки всего текста (часть 2)} % Подпись листинга

\section*{Вывод}

В данном разделе были реализованы последовательная и параллельная версии алгоритма составления файла словаря с количеством употреблений каждой $N$-граммы букв из одного слова в тексте на русском языке.

    