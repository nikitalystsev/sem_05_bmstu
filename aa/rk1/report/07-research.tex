\chapter{Исследовательский раздел}

В данном разделе будут проведены сравнения реализаций алгоритмов сортировки  по времени работы и по затрачиваемой памяти.

\section{Технические характеристики}

Технические характеристики устройства, на котором проводились исследования: 

\begin{itemize}[label=--]
	\item операционная система: Ubuntu 22.04.3 LTS x86\_64 \cite{info_os};
	\item оперативная память: 16 Гб;
	\item процессор: 11th Gen Intel® Core™ i7-1185G7 @ 3.00 ГГц × 8, 4 физических ядра, 8 логических ядер.
\end{itemize}

\section{Демонстрация работы программы}

На рисунке \ref{img:ex1} представлен пример результата работы программы. Пользователь, указывая соответствующие пункты меню, запускает последовательную обработку заявок, затем параллельное исполнение конвейера, затем выходит из программы.

\clearpage

\includeimage
{ex1} % Имя файла без расширения (файл должен быть расположен в директории inc/img/)
{f} % Обтекание (без обтекания)
{h} % Положение рисунка (см. figure из пакета float)
{1\textwidth} % Ширина рисунка
{Пример работы программы} % Подпись рисунка	

\section{Проведение исследования}

\section*{Цель исследования}

Целью исследования является проведение сравнительного анализа времени работы последовательного и параллельного алгоритма работы конвейера от числа поступающих заявок.

\section*{Наборы варьируемых и фиксированных параметров}

Замеры времени проводились для числа заявок, равному 10, 20, 30, 40, 50, 60, 70, 80, 90, 100.

В качестве фиксированного параметра было выбрано число $N$, равное 3, число строк в файле, равное 100 и длина строки, равная 100 символам.

\clearpage

Замеры времени для каждого размера 1000 раз. Время работы алгоритмов измерялось с использованием функции $clock\_gettime$ из встроенного модуля $ctime$ \cite{cpu_time_func}.  

\clearpage

\section*{Результаты первого исследования}

\begin{table}[ht]
	\small
	\begin{center}
		\begin{threeparttable}
			\caption{Замер времени для конвейера с числом заявок от 10 до 100}
			\label{tbl:time}
			\begin{tabular}{|r|r|r|}
				\hline
				& \multicolumn{2}{c|}{\bfseries Время, мкс} \\ \cline{2-3}
				\bfseries \makecell{Число заявок, \\ единицы} & \bfseries \makecell{Последовательная \\ версия} & \bfseries \makecell{Параллельная \\ версия} \\ \cline{2-3}
				\hline
				10 & 435.759 & 882.219  \\
				\hline
				20 & 923.062 & 1 532.931 \\
				\hline
				30 & 1 435.287 & 2 343.806  \\
				\hline
				40 & 2 000.813 & 3 070.613 \\
				\hline
				50 & 2 383.545 & 3 892.355  \\
				\hline
				60 & 3 051.124 & 4 674.078  \\
				\hline
				70 & 3 511.962 & 5 709.621  \\
				\hline
				80 & 4 054.163 & 6 309.628  \\
				\hline
				90 & 4 709.397 & 7 285.999  \\
				\hline
				100 & 5 569.732 & 8 170.710  \\
				\hline
			\end{tabular}	
		\end{threeparttable}
	\end{center}
\end{table}

\clearpage

%\includesvgimage
%{research1} % Имя файла без расширения (файл должен быть расположен в директории inc/img/)
%{f} % Обтекание (без обтекания)
%{h} % Положение рисунка (см. figure из пакета float)
%{1\textwidth} % Ширина рисунка
%{Результаты замеров времени работы алгоритмов для файлов с числом строк от 10 до 100} % Подпись рисунка

Из полученных данных следует, что однопоточный процесс демонстрирует более высокую эффективность по сравнению с процессом, включающим в себя создание вспомогательного потока для обработки всех строк файла. Это объясняется дополнительными временными затратами, связанными с созданием потока и передачей ему необходимых аргументов.



\section*{Вывод}

Для файла с текстом на русском языке размером в 100 строк:

\begin{itemize}[label*=--]
	\item однопоточный процесс --- 5569.732 мкс;
	\item параллельная версия с 1-м вспомогательным потоком --- 8170.710;
	\item параллельная версия с 2-мя вспомогательными потоками -- 7397.898;
	\item параллельная версия с 64-мя вспомогательными потоками -- 12837.810;
\end{itemize}

Таким образом, для данного алгоритма использование дополнительных потоков приводит лишь к увеличению времени работы программы (даже с 2-мя потоками время работы программы в 1.33 раза больше, чем у однопоточного процесса).Для данной архитектуры вычислительной машины рекомендуется удерживаться при числе дополнительных потоков, равном 2.

