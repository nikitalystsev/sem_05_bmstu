\chapter*{ВВЕДЕНИЕ}
\addcontentsline{toc}{chapter}{ВВЕДЕНИЕ}

Использование параллельной обработки открывает новые способы
для ускорения работы программ. 
Конвейерная обработка является одним
из примеров, где использование принципов параллельности помогает ускорить обработку данных. 
Суть та же, что и при работе реальных конвейерных лент --- материал (данное) поступает на обработку, после окончания обработки материал передается на место следующего обработчика, при этом
предыдущий обработчик не ждет полного цикла обработки материала, а
получает новый материал и работает с ним.

Целью данной лабораторной работы является описание параллельных конвейерных вычислений.

Для достижения поставленной цели необходимо решить следующие задачи:

\begin{enumerate}[label={\arabic*)}]
	\item изучить и описать организацию конвейерной обработки данных;
	\item описать алгоритмы обработки данных, которые будут использоваться в
	текущей лабораторной работе;
	\item определить средства программной реализации;
	\item реализовать программу, выполняющую конвейерную обработку с количеством лент не менее трех в однопоточной и многопоточной реализаций;
	\item сравнить и проанализировать реализации алгоритмов по затраченному времени.
\end{enumerate}
