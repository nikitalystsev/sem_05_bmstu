\chapter{Аналитический раздел}

В данном разделе рассмотрена информация, касающаяся основ конвейерной обработки данных.

\section{Конвейерная обработка данных}

Конвейер --- организация вычислений, при которой увеличивается количество выполняемых инструкций за единицу времени за счет использования принципов параллельности.

Конвейеризация (или конвейерная обработка) в общем случае основана
на разделении подлежащей исполнению функции на более мелкие части, называемые ступенями, и выделении для каждой из них отдельного блока аппаратуры. Производительность при этом возрастает, благодаря тому что одновременно на различных ступенях конвейера выполняется несколько команд.
Конвейерная обработка такого рода широко применяется во всех современных быстродействующих
процессорах.

Конвейеризация увеличивает пропускную способность процессора (количество команд, завершающихся в единицу времени), но она не сокращает время выполнения отдельной команды. 
В действительности она даже несколько увеличивает время выполнения каждой команды из-за накладных расходов, связанных с хранением промежуточных результатов. 
Однако увеличение пропускной способности означает, что программа будет выполняться быстрее по
сравнению с простой, неконвейерной схемой \cite{info_conveyor}.

\section{N-граммы}

Пусть задан некоторый конечный алфавит 

\begin{equation}
	\label{eq:alph}
	V = {w_i}
\end{equation} 

где $w_i$ --- символ. 

Языком $L(V)$ называют множество цепочек конечной длины из символов $w_i$.
N-граммой на алфавите $V$ (\ref{eq:alph}) называют произвольную цепочку из $L(V)$ длиной $N$, например последовательность из $N$ букв русского языка одного слова, одной фразы, одного текста или, в более интересном случае, последовательность из грамматически допустимых описаний $N$ подряд стоящих слов \cite{info_ngram}.

\section{Последовательная версия алгоритма}

Алгоритм составления файла словаря с количеством употреблений каждой N-граммы букв из одного слова в тексте на русском языке состоит из следующих шагов:

\begin{enumerate}[label={\arabic*)}]
	\item считывание текста в массив строк;
	\item преобразование считанного текста (перевод букв в нижний регистр, удаление знаков препинания);
	\item обработка каждой строки текста.
\end{enumerate}

Обработка строки текста включает следующие шаги:

\begin{enumerate}[label={\arabic*)}]
	\item обработка каждого слова из строки текста;
	\item выделение существующих в этом слове $N$-грамм;
	\item увеличение количества выделенных $N$-грамм в словаре.
\end{enumerate}

\section{Параллельная версия алгоритма}

В алгоритме составления файла словаря с количеством употреблений каждой $N$-граммы букв из одного слова в тексте на русском языке обработка строк текста происходит независимо, поэтому есть возможность произвести распараллеливание данных вычислений. 

Для этого строки текста поровну распределяются между потоками. Каждый поток получает локальную копию словаря для $N$-грамм, производит вычисления над своим набором строк и после завершения работы всех потоков словари с количеством употреблений $N$-грамм каждого потока объединяются в один. 
Так как каждая строка массива передается в монопольное использование каждому потоку, не возникает конфликтов доступа к разделяемым ячейки памяти,следовательно, в использовании средства синхронизации в виде мьютекса нет необходимости.

\section{Описание алгоритма}

Ленты конвейера (обработчики) будут передавать друг другу заявки. Первый этап, или обработчик, будет формировать заявку, которая будет передаваться от этапа к этапу.

Заявка будет содержать:

\begin{itemize}[label*=--]
	\item имя файла с исходным текстом на русском языке;
	\item имя выходного файла для словаря $N$-грамм;
	\item число $N$.
	\item временные отметки начала и конца выполнения каждой стадии обработки заявки.
\end{itemize}

В качестве операций, выполняющихся на конвейере, взяты следующие:

\begin{enumerate}[label={\arabic*)}]
	\item чтение текста на русском языке из файла;
	\item выполнение параллельной версии алгоритма составления файла словаря с количеством употреблений каждой N-граммы букв из одного слова в тексте на русском языке в 2 потока;
	\item логирование заявки в файл в формате: <имя\_файла> + словарь $N$-грамм;
\end{enumerate}

\section*{Вывод}

В данном разделе было рассмотрено понятие конвейерной обработки,
а также выбраны этапы обработки файлов, которые будут обрабатывать ленты конвейера. 
Также рассмотрено понятие $N$-грамм.

Программа будет получать на вход количество задач, число $N$, число строк в файле и длина каждой строки, а также выбор алгоритма: последовательный или конвейерный.
