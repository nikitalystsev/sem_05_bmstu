\chapter{Технологический раздел}

В данном разделе будут перечислены требования к реализуемому программному обеспечению, средства реализации, листинги кода и функциональные тесты.

\section{Требования к программному обепечению}

\subsection{Требования к вводу}

К входным данным предъявляется несколько условий:

\begin{itemize}[label=--]
	\item на вход подаются две строки;
	\item входные строки могут содержать цифры, буквы как русского так и английского алфавита;
	\item верхний и нижний регистр букв считается различным.
\end{itemize}

\subsection{Требования к программе}

Программа должна удовлетворять следующим условиям: 

\begin{itemize}[label=--]
	\item пустые последовательности строк должны корректно обрабатываться;
	\item результат работы программы -- число (расстояние Левенштейна или Дамерау-Левенштейна);
	\item наличие консольного интерфейса взаимодействия с программой;
	\item наличие функционала для замеров процессорного времени и оценки затрат по памяти реализаций алгоритмов поиска расстояний Левенштейна и Дамерау-Левенштейна.
\end{itemize}

\clearpage

\section{Средства реализации}

В качестве языка программирования для этой лабораторной работы был выбран $C++$ \cite{pl} по следующим причинам:

\begin{itemize}[label=--]
	\item в $C++$ есть встроенный модуль $ctime$, предоставляющий необходимый функционал для замеров процессорного времени;
	\item в стандартной библиотеке $C++$ есть оператор $sizeof$, позволяющий получить размер переданного объекта в байтах. Следовательно, $C++$ предоставляет возможности для проведения точных оценок по используемой памяти;
	\item в $C++$ есть тип данных $std::wstring$, который позволяет хранить и использовать как кириллические, так и латинские символы.
\end{itemize}

В качестве функции, которая будет осуществлять замеры процессорного времени, будет использована функция $clock\_gettime$ из встроенного модуля $ctime$ \cite{cpu_time_func}.

\section{Сведения о модулях программы}

Программа состоит из семи модулей: 

\begin{enumerate}[label={\arabic*)}]
	\item \texttt{algorithms.cpp} --- модуль, хранящий реализации алгоритмов поиска расстояний Левенштейна и Дамерау-Левенштейна;
	\item \texttt{processTime.cpp} --- модуль, содержащий функцию для замера процессорного времени;
	\item \texttt{memoryMeasurements.cpp} --- модуль, содержащий функции, позволяющие провести сравнительный анализ использования памяти в реализациях алгоритмов поиска расстояний Левенштейна и Дамерау-Левенштейна;
	\item \texttt{timeMeasurements.cpp} --- модуль, содержащий функции, позволяющие провести сравнительный анализ использования времени в реализациях алгоритмов поиска расстояний Левенштейна и Дамерау-Левенштейна;
	\item \texttt{main.cpp} --- файл, содержащий точку входа в программу, из которой происходит вызов алгоритмов по разработанному интерфейсу;
	\item \texttt{interface.cpp} --- модуль, содержащий функции для обработки действий пользователя при взаимодействии с программой;
	\item \texttt{task7} --- модуль, содержащий набор скриптов для проведения замеров программы по времени и памяти и построения графиков по полученным данным.
\end{enumerate}

\section{Реализации алгоритмов}


В листингах \ref{lst:levNotRecur.txt}, \ref{lst:damLevNotRecurWithMatrix.txt}, \ref{lst:damLevRecur.txt}, \ref{lst:damLevRecurWithCaching.txt} приведены реализации алгоритмов нахождения расстояния Левенштейна и Дамерау-Левенштейна.

\includelisting
{levNotRecur.txt} % Имя файла с расширением (файл должен быть расположен в директории inc/lst/)
{Функция нахохжения расстояния Левенштейна нерекурсивным способом} % Подпись листинга

\clearpage

\includelisting
{damLevNotRecurWithMatrix.txt} % Имя файла с расширением (файл должен быть расположен в директории inc/lst/)
{Функция нахохжения расстояния Дамерау-Левенштейна нерекyрекурсивным способом} % Подпись листинга

\clearpage

\includelisting
{damLevRecur.txt} % Имя файла с расширением (файл должен быть расположен в директории inc/lst/)
{Функция нахохжения расстояния Дамерау-Левенштейна рекурсивным способом} % Подпись листинга

\clearpage

\includelisting
{damLevRecurWithCaching.txt} % Имя файла с расширением (файл должен быть расположен в директории inc/lst/)
{Функция нахохжения расстояния Дамерау-Левенштейна рекурсивным способом c кешированием} % Подпись листинга

\clearpage

\section{Функциональные тесты}

В таблице \ref{tbl:func_tests} приведены тестовые данные, на которых было протестированно разработанное ПО. Все тесты были успешно пройдены.

\begin{table}[ht]
	\small
	\begin{center}
		\begin{threeparttable}
			\caption{Функциональные тесты}
			\label{tbl:func_tests}
			\begin{tabular}{|c|c|c|c|c|c|}
				\hline
				\multicolumn{2}{|c|}{\bfseries Входные данные}
				& \multicolumn{4}{c|}{\bfseries Расстояние и алгоритм} \\ 
				\hline 
				&
				& \multicolumn{1}{c|}{\bfseries Левенштейна} 
				& \multicolumn{3}{c|}{\bfseries Дамерау-Левенштейна} \\ \cline{3-6}
				
				\bfseries Строка 1 & \bfseries Строка 2 & \bfseries Итеративный & \bfseries Итеративный
				
				& \multicolumn{2}{c|}{\bfseries Рекурсивный} \\ \cline{5-6}
				& & & & \bfseries Без кеша & \bfseries С кешом \\
				\hline
				"пустая" & "пустая" & 0 & 0 & 0 & 0 \\
				\hline
				"пустая" & qwerty & 6 & 6 & 6 & 6 \\
				\hline
				doctorwho & "пустая" & 2 & 2 & 2 & 2 \\
				\hline
				code & decoder & 3 & 3 & 3 & 3 \\
				\hline
				transposition & transpose & 5 & 5 & 5 & 5 \\
				\hline
				asas молоко& saas & 2 & 1 & 1 & 1 \\
				\hline
				друзья & рдузия & 3 & 2 & 2 & 2 \\
				\hline
				гибралтар & лабрадор & 5 & 5 & 5 & 5 \\
				\hline
				АВТОР & АФФТАР & 3 & 3 & 3 & 3 \\
				\hline
				moloko & молоко & 6 & 6 & 6 & 6 \\
				\hline
			\end{tabular}	
		\end{threeparttable}
	\end{center}
\end{table}

\section*{Вывод}

В данном разделе были реализованы и протестированы четыре алгоритма:
вычисления расстояния Левенштейна (нерекурсивно), Дамерау-Левенштейна нерекурсивно, рекурсивно, рекурсивно с кешированием.
    