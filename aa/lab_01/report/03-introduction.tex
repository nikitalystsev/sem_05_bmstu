\chapter*{ВВЕДЕНИЕ}
\addcontentsline{toc}{chapter}{ВВЕДЕНИЕ}

Расстояние Левенштейна (редакционное расстояние, дистанция редактирования) — метрика, определяющаяся как минимальное количество односимвольных операций (а именно вставки, удаления, замены), необходимых для превращения одной последовательности символов в другую.

Расстояние Левенштейна и Дамерарау-Левенштейна активно применяется для исправления ошибок в слове (в поисковых системах, базах данных, при вводе текста, при автоматическом распознавании отсканированного текста или речи), для сравнения текстовых файлов, в биоинформатике для сравнения генов, хромосом и белков.

Расстояние Дамерау-Левенштейна — модификация расстояния Левенштейна. К операциям вставки, удаления и замены символов добавляется операция транспозиции (перестановки) двух соседних символов.

Целью данной лабораторной работы является изучение, описание, реализация и исследование алгоритмов поиска расстояний Левенштейна и Дамерау-Левенштейна.

Для достижения поставленной цели необходимо решить следующие задачи: 

\begin{enumerate}[label={\arabic*)}]
	\item Изучить и описать алгоритмы поиска расстояний Левенштейна и Дамерау-Левенштейна.
	\item Создать программное обеспечение, реализующее следующие алгоритмы:
	\begin{itemize}[label=--]
		\item нерекурсивный алгоритм поиска расстояния Левенштейна;
		\item нерекурсивный алгоритм поиска расстояния Дамерау-Левенштейна;
		\item рекурсивный алгоритм поиска расстояния Дамерау-Левенштейна;
		\item рекурсивный алгоритм поиска расстояния Дамерау-Левенштейна с кешированием.
	\end{itemize}
\clearpage
	\item Провести анализ эффективности реализаций алгоритмов по памяти и по времени.
	\item Обосновать полученные результаты в отчете к выполненной лабораторной работе.
\end{enumerate}
