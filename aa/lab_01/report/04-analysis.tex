\chapter{Аналитический раздел}

% расставить ссылки

В этом разделе будет дано описание алгоритмов поиска расстояний Левенштейна и Дамерау-Левенштейна. 

\section{Алгоритм поиска расстояния Левенштейна}

Расстояние Левенштейна \cite{Lev}, редакционное расстояние — метрика сходства между двумя строковыми последовательностями, минимальное количество редакторский операций вставки, удаления, замены символа, необходимых для превращения одной строки в другую.

Каждая редакторская операция имеет свою цену. Для алгоритма поиска расстояния Левенштейна устанавливаются следующие цены:  

\begin{enumerate}[label=\arabic*)]
	\item $w(a, b) = 1, \medspace a \neq b$ --- цена замены символа $a$ на $b$;
	
	 $w(a,a)=0$ --- замены не происходит.
	 
	\item $w(\varepsilon, b) = 1$ --- цена вставки символа $b$;
	\item $w(a, \varepsilon) = 1$ --- цена удаления символа $a$.
\end{enumerate}

Пусть $S_1$ -- первая строка, тогда ее длина будет равна $L_1$ и пусть $S_2$ -- вторая строка, имеющая длину $L_2$. $S_1[1...i]$ -- подстрока $S_1$ длиной $i$ символов, начиная с первого, $S_2[1...j]$ -- подстрока $S_2$ длиной $j$ символов.

Введем функцию $D(i,j)$, результатом работы которой является редакционное расстояние между двумя подстроками $S_1[1...i]$ $S_2[1...j]$.

Функция $D$ определяется следующей рекуррентной формулой:

\begin{equation}
	\label{eq:L}
	D(i, j) =
	\begin{cases}
		0, &\text{i = 0, j = 0}\\
		i, &\text{j = 0, i > 0}\\
		j, &\text{i = 0, j > 0}\\
		min \begin{cases}
			D(i], j - 1) + 1,\\
			D(i - 1, j) + 1,\\
			D(i - 1, j - 1) +  m(S_{1}[i], S_{2}[j]), \\
		\end{cases}
		&\text{i > 0, j > 0}
	\end{cases}
\end{equation}
где сравнение символов строк $S_1$ и $S_2$ рассчитывается как:
\begin{equation}
	\label{eq:m}
	m(a, b) = \begin{cases}
		0, &\text{если a = b,}\\
		1, &\text{иначе.}
	\end{cases}
\end{equation}


Тогда расстояние Левенштейна $d(S_1, S_2)$ между двумя строками $S_1$ и $S_2$ будет равно $D(L_1, L_2)$.

\subsection{Нерекурсивый алгоритм поиска расстояния Левенштейна}

Формула \ref{eq:L} является рекуррентной. С ростом $i, j$ растет время выполнения программы, так как во время работы множества промежуточных значений $D(i,j)$ вычисляются по нескольку раз.

Для оптимизации времени работы алгоритма можно использовать матрицу для хранения промежуточных значений. Эта матрица имеет размеры $(L_1 + 1) \times (L_2 + 1)$. Значения в ячейке $[i, j]$ равно значению $D(i, j)$. Вся матрица заполняется в соответствии с формулой \ref{eq:L}.

Алгоритм, использующий матрицу, будет неэффективен по памяти при больших $L_1$ и $L_2$, поскольку вычисленные промежуточные значения будут храниться в памяти после их использования.

Заметим, что для задачи поиска расстояния Левенштейна в этом алгоритме нет необходимости хранить всю матрицу целиком --- достаточно лишь текущей и предыдущей строк. Это существенно уменьшит потребляемую память с сохранением хорошей скорости работы.

\section{Алгоритм поиска расстояния Дамерау-Левенштейна}

Расстояние Дамерау-Левенштейна \cite{DamLev} --- модификация расстояния Левенштейна. К операциям вставки, удаления и замены символа добавляется операция транспозиции (перестановки) двух соседних символов.

Расстояние Дамерау-Левенштейна может быть найдено по формуле \ref{eq:DL}.
\begin{equation}
	\label{eq:DL}
	d_{a,b}(i, j) = 
	\begin{cases}
		max(i, j), &  min(i, j) = 0,\\
		min \begin{cases}
			d_{a,b}(i], j - 1) + 1,\\
			d_{a,b}(i - 1, j) + 1,\\
			d_{a,b}(i - 1, j - 1) +  m(a[i], b[j]), \\
			d_{a,b}(i - 2, j - 2) +  1,
		\end{cases} & 
		\begin{split}
		& i, j > 1 \\ 
		& a[i] = b[j-1] \\
		& b[j] = a[i-1] \\
		\end{split}  \\
		min \begin{cases} 
			d_{a,b}(i], j - 1) + 1,\\
			d_{a,b}(i - 1, j) + 1,\\
			d_{a,b}(i - 1, j - 1) +  m(a[i], b[j]), \\
		\end{cases} & \text{иначе}
	\end{cases},
\end{equation}

\subsection{Нерекурсивый алгоритм поиска расстояния Дамерау-Левенштейна}

Формула \ref{eq:DL} является рекуррентной. С ростом $i, j$ растет время выполнения программы, так как во время работы множества промежуточных значений $d_{a,b}(i,j)$ вычисляются по нескольку раз.

Для оптимизации времени работы алгоритма можно использовать матрицу для хранения промежуточных значений. Эта матрица имеет размеры $(L_1 + 1) \times (L_2 + 1)$, где $L_1$ и $L_2$ --- длины первой и второй строк соответственно. Значения в ячейке $[i, j]$ равно значению $d_{a,b}(i, j)$. Вся матрица заполняется в соответствии с формулой \ref{eq:DL}.

Алгоритм, использующий матрицу, будет неэффективен по памяти при больших $L_1$ и $L_2$, поскольку вычисленные промежуточные значения будут храниться в памяти после их использования.

\subsection{Рекурсивый алгоритм поиска расстояния Дамерау-Левенштейна}

Как и в случае с формулой \ref{eq:L}  формула \ref{eq:DL} является рекуррентной. Прямая реализация данного алгоритма будет неэффективной по времени из-за того, что промежуточные значения $d_{a,b}(i,j)$ вычисляются не по одному разу в ходе работы алгоритма.

\subsection{Рекурсивый алгоритм поиска расстояния Дамерау-Левенштейна с кешированием}

Идея рекурсивного алгоритма Дамерау-Левенштейна с кешированием заключается в том, чтобы эффективно вычислить минимальное расстояние между двумя строками, используя рекурсивные вызовы. При этом сохраняются результаты вычислений для каждой пары индексов в кеше, чтобы избежать повторных вычислений. Кешем является двумерная матрица размерами $(L_1 + 1) \times (L_2 + 1)$, где $L_1$ и $L_2$ --- длины первой и второй строк соответственно.

Это позволяет существенно улучшить производительность алгоритма, особенно в случае, когда вычисления могут быть пересчитаны многократно для одних и тех же подстрок.

\section*{Вывод}

В данном разделе были рассмотрены понятия расстояний Левенштейна и Дамерау-Левенштейна, объяснена разница между этими понятиями, были приведены формулы, описывающие алгоритмы поиска этих расстояний, а также были рассмотрены различные способы их реализации.
