\chapter{Аналитический раздел}

В данном разделе будет рассмотрена задача коммивояжера и будут описаны алгоритмы её решения.

\section{Задача коммивояжера}

Цель задачи коммивояжера \cite{info_tsp} заключается в нахождении самого выгодного маршрута (кратчайшего, самого быстрого, наиболее дешевого), проходящего через все заданные точки (пункты, города) по одному разу.

Условия задачи должны содержать критерий выгодности маршрута (должен ли он быть максимально коротким, быстрым, дешевым или все вместе), а также исходные данные в виде матрицы затрат (расстояния, стоимости, времени) при перемещении между рассматриваемыми пунктами.

\section{Алгоритм полного перебора}

Рассмотрим $N$ городов и матрицу расстояний между ними. 
Найдем самый короткий маршрут посещения всех городов ровно по одному разу, без возвращения в первый город:

\begin{enumerate}[label={\arabic*)}]
	\item число вариантов для выбора первого города равно $N$;
	\item число вариантов для выбора второго города равно $N-1$;
	\item с каждым следующим городом число вариантов уменьшается на 1;
	\item число всех вариантов маршрутра равно $N!$;
	\item минимальный по сумме значений матрицы расстояний вариант маршрута --- искомый.
\end{enumerate}

В связи со сложностью $N!$ полный перебор вариантов занимает существенное время, а при большом количестве городов становится технически невозможным.

\section{Муравьиный алгоритм}

\section*{Вывод}

