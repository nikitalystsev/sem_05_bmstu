\chapter{Аналитический раздел}

В данном разделе будет рассмотрена задача коммивояжера и будут описаны алгоритмы её решения.

\section{Задача коммивояжера}

Цель задачи коммивояжера \cite{info_tsp} заключается в нахождении самого выгодного маршрута (кратчайшего, самого быстрого, наиболее дешевого), проходящего через все заданные точки (пункты, города) по одному разу.

Условия задачи должны содержать критерий выгодности маршрута (должен ли он быть максимально коротким, быстрым, дешевым или все вместе), а также исходные данные в виде матрицы затрат (расстояния, стоимости, времени) при перемещении между рассматриваемыми пунктами.

\section{Алгоритм полного перебора}

Рассмотрим $n$ городов и матрицу расстояний между ними. 
Найдем самый короткий маршрут посещения всех городов ровно по одному разу, без возвращения в первый город:

\begin{enumerate}[label={\arabic*)}]
	\item число вариантов для выбора первого города равно $n$;
	\item число вариантов для выбора второго города равно $n-1$;
	\item с каждым следующим городом число вариантов уменьшается на 1;
	\item число всех вариантов маршрутра равно $n!$;
	\item минимальный по сумме значений матрицы расстояний вариант маршрута --- искомый.
\end{enumerate}

В связи со сложностью $n!$ полный перебор вариантов занимает существенное время, а при большом количестве городов становится технически невозможным.

\section{Муравьиный алгоритм}

Идея муравьиного алгоритма \cite{info_antAlg} --- моделирование поведения муравьев, связанное с их способностью быстро находить кратчайший путь и адаптироваться к изменяющимся условиям, находя новый кратчайший путь.

Муравей обладает следующими чувствами:

\begin{itemize}[label*=--]
	\item \texttt{зрение} --- муравей $k$, находясь в городе $i$, может оценить длину (метку) $D_{ij}$ дуги/ребра $i-j$ и привлекательность этого ребра/дуги, описываемую формулой \ref{eq:D_func}:
	\begin{equation}
		\label{eq:D_func}
		\eta_{ij} = 1 / D_{ij},
	\end{equation}
	
	\item \texttt{обоняние} --- муравей чует концентрацию феромона $\tau_{ij}(t)$ на ребре/дуге $i-j$ в текущий день $t$.
	
	\item \texttt{память} --- муравей $k$ запоминает список/кортеж посещенных за текущий день $t$ городов --- $J_k(t)$
\end{itemize}

Муравей выполняет следующую последовательность действий, пока не посетит все города:

\begin{enumerate}[label={\arabic*)}]
	\item находясь в городе $i$, муравей $k$ выбирает следующий город $j$ по вероятностному правилу (формула \ref{eq:P_func}):
	\begin{equation}
		\label{eq:P_func}
		P_{ij, k}(t) = 
		\begin{cases}
			0,  & j \in J_k(t) \\
			\frac{\eta_{ij}^\alpha \cdot \tau_{ij}^\beta}{\displaystyle\sum_{q \notin J_k(t)} \eta^\alpha_{iq} \cdot \tau^\beta_{iq}}, & \textrm{если город $j$ необходимо посетить} 
		\end{cases}
	\end{equation}
	где $\alpha$ --- параметр влияния видимости пути, $\beta$ --- параметр влияния феромона, $\tau_{ij}$ --- число феромона на дуге/ребре $i-j$, $\eta_{ij}$ --- эвристическое желание посетить город $j$, если муравей находится в городе $i$. 
	Выбор города является вероятностным, данное правило определяет ширину зоны города $j$. 
	В общую зону всех не посещенных муравьем городов бросается случайное число, которое и определяет выбор муравья;
	\item муравей $k$ проходит путь по дуге/ребре $i-j$ и оставляет на нем феромон.
\end{enumerate}

Информация о числе феромона на пути используется другими муравьями для выбора пути. 
Те муравьи, которые случайно выберут кратчайший путь, будут быстрее его проходить, и за несколько передвижений он будет более обогащен феромоном. 
Cледующие муравьи будут предпочитать именно этот путь, продолжая обогащать его феромоном. 

После прохождения маршрутов всеми муравьями (ночью, перед наступлением следующего дня) значение феромона на путях обновляется в соответствии со следующим правилом \eqref{eq:update_phero_1}:
\begin{equation}
	\label{eq:update_phero_1}
	\tau_{ij}(t+1) = (1-\rho)\tau_{ij}(t) + \Delta \tau_{ij},
\end{equation}

где $\rho \in [0, 1]$ --- коэффициент испарения. 
Чтобы найденное локальное решение не было единственным, моделируется испарение феромона.

При этом
\begin{equation}
	\label{update_phero_2}
	\Delta \tau_{ij} = \sum_{k=1}^N \Delta\tau_{ij, k},
\end{equation}

где $N$ --- число муравьев,
\begin{equation}
	\label{update_phero_3}
	\Delta\tau_{ij,k} = 
	\begin{cases}
		0,  & \textrm{если ребро/дуга $i-j$ не посещена муравьем $k$ в день $t$} \\
		\frac{Q}{L_{k}}, & \textrm{иначе} 
	\end{cases}
\end{equation}

где $L_{k}$ --- длина маршрута (сумма меток дуг/ребер) $k$-го муравья в день $t$, $Q$ --- квота (некоторая константа) феромона 1-го муравья на день, $Q$ выбирается соразмерной длине наименьшего маршрута (формула \ref{eq:Q_func}).

\begin{equation}
	\label{eq:Q_func}
	 Q = \frac{\sum \text{недиагональных элементов матрицы смежности}}{\text{линейный размер матрицы смежности}}
\end{equation}

Поскольку вероятность \ref{eq:P_func} перехода в заданную точку не должна быть равна нулю, необходимо обеспечить неравенство $\tau_{ij}(t)$ нулю посредством введения дополнительного минимально возможного значения феромона $\tau_{min}$  и в случае, если $\tau_{ij}(t+1)$ принимает значение, меньшее $\tau_{min}$, откатывать значение феромона до этой величины $\tau_{min}$.

\section*{Вывод}

В данном разделе была рассмотрена задача коммивояжера, а также способы её решения --- алгоритм полного перебора и муравьиный алгоритм.
