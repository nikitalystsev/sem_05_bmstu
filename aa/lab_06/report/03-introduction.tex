\chapter*{ВВЕДЕНИЕ}
\addcontentsline{toc}{chapter}{ВВЕДЕНИЕ}

% пока без доп текста

Цель лабораторной работы – описание методов решения задачи коммивояжера полным перебором и на основе муравьиного алгоритма.

Для достижения поставленной цели необходимо решить следующие задачи:

\begin{enumerate}[label={\arabic*)}]
	\item изучить и описать задачу коммивояжера;
	\item изучить и описать методы решения задачи коммивояжера ---  метод полного перебора и метод на основе муравьиного алгоритма;
	\item разработать и реализовать программный продукт, позволяющий решить задачу коммивояжера исследуемыми методами;
	\item сравнить по времени метод полного перебора и метод на основе муравьиного алгоритма.
	\item обосновать полученные результаты в отчете о выполненной лабораторной работе.
\end{enumerate}

Выданный индивидуальный вариант для выполнения лабораторной работы:

\begin{itemize}[label*=--]
	\item неориентированый граф;
	\item без элитных муравьев;
	\item карта городов России от Калининграда до Владивостока;
	\item незамкнутый маршрут.
\end{itemize}