\chapter{Исследовательский раздел}

В данном разделе будет проведен сравнительный анализ алгоритмов при различных ситуациях на
основе полученных данных.

\section{Технические характеристики}

Технические характеристики устройства, на котором проводились исследования: 

\begin{itemize}[label=--]
	\item операционная система: Ubuntu 22.04.3 LTS x86\_64 \cite{info_os};
	\item оперативная память: 16 Гб;
	\item процессор: 11th Gen Intel® Core™ i7-1185G7 @ 3.00 ГГц × 8, 4 физических ядра, 8 логических ядер.
\end{itemize}

\section{Проведение первого исследования}

\section*{Цель исследования}

Целью исследования является проведение сравнительного анализа времени работы алгоритма полного перебора и муравьиного алгоритма от линейного размера матрицы смежности.

\section*{Наборы варьируемых и фиксированных параметров}

Замеры времени проводились для линейных размеров матрицы, равных 2, 3, 4, 5, 6, 7, 8, 9, 10.

В качестве фиксированного параметра был выбран разброс значений для матрицы смежности от 5 до 10. Для муравьиного алгоритма были фиксированы следующие параметры:

\begin{itemize}[label*=--]
	\item $\alpha$ --- параметр влияния видимости пути (значение 0.5);
	\item $\beta$ --- параметр влияния феромона (значение 0.5);

\clearpage

	\item $\rho$ --- коэффициент испарения (значение 0.5);
	\item время жизни муравьиной колонии (значение 250).
\end{itemize}

Замеры времени для каждого линейного размера матрицы проводились 10 раз. 

\section*{Результаты первого исследования}

\begin{table}[ht]
	\small
	\begin{center}
		\begin{threeparttable}
			\caption{Замер времени для  для линейных размеров матрицы от 2 до 10}
			\label{tbl:time}
			\begin{tabular}{|r|r|r|}
				\hline
				& \multicolumn{2}{c|}{\bfseries Время, c} \\ \cline{2-3}
				\bfseries \makecell{Линейный размер матрицы, \\ единицы} & \bfseries \makecell{Алгоритм \\ полного перебора} & \bfseries \makecell{Муравьиный \\ алгоритм} \\ \cline{2-3}
				\hline
		         2 &   0.000019 &   0.000323 \\ \hline
		         3 &   0.000010 &   0.000901 \\ \hline
		         4 &   0.000034 &   0.001879 \\ \hline
		         5 &   0.000156 &   0.003271 \\ \hline
		         6 &   0.001017 &   0.005237 \\ \hline
		         7 &   0.007903 &   0.007688 \\ \hline
		         8 &   0.070830 &   0.010854 \\ \hline
		         9 &   0.708097 &   0.014859 \\ \hline
		        10 &   7.946709 &   0.019937 \\ \hline
			\end{tabular}	
		\end{threeparttable}
	\end{center}
\end{table}

\clearpage

\includesvgimage
{research1} % Имя файла без расширения (файл должен быть расположен в директории inc/img/)
{f} % Обтекание (без обтекания)
{h} % Положение рисунка (см. figure из пакета float)
{1\textwidth} % Ширина рисунка
{Результаты замеров времени для для линейных размеров матрицы от 2 до 10} % Подпись рисунка

На линейных размерах матрицы смежности от 2 до 6 алгоритм полного перебора работает значительно быстрее муравьиного алгоритма (на матрице смежности размером $2\times2$ в 17 раз, а на матрице смежности размером $6\times6$ в 5.14 раза), но на линейных размерах матрицы смежности от 7 до 10 муравьиный алгоритм выигрывает у алгоритма полного перебора по времени выполнения (на матрице смежности размером $7\times7$ в 1.02 раз, а на матрице смежности размером $10\times10$ в уже в 398.59 раза). Это связано с тем, что алгоритм полного перебора имеет вычислительную сложность $O(n!)$, в то время как сложность муравьиного алгоритма $O(t \cdot m \cdot n^2)$, где $t$ --- количество итераций, $m$ --- количество муравьев, $n$ --- число вершин \cite{info_compl_ant_alg}.

\clearpage

\section{Проведение второго исследования}

Автоматическая параметризация была проведена на единственном классе данных, состоящем из трех полносвязных графов с 10 вершинами в каждом с одинаковым разбросом значений меток ребер/дуг графа.

Итоговая таблица значений параметризации будет состоять из следующих колонок:
\begin{itemize}[label=---]
	\item $\alpha$ --- параметр влияния видимости пути;
	\item $\rho$ --- коэффициент испарения;
	\item \textit{days} --- количество дней жизни колонии муравьев;
	\item \textit{Result} --- эталонный результат, полученный методом полного перебора для проведения данного эксперимента;
	\item \textit{Mistake} --- разность полученного основанным на муравьином алгоритме методом значения и эталонного значения на данных значениях параметров, показатель качества решения.
\end{itemize}

\section*{Цель исследования}

Цель исследования --- определить комбинацию параметров, которые позволяют решить задачу наилучшим образом для выбранного класса данных. 
Качество решения зависит от количества дней и погрешности измерений.

\subsection{Граф №1}

Согласно с вариантом индивидуального задания представим полносвязный неориентированный граф, вершинами которого являются города от Калиниграда до Владивостока:

\begin{enumerate}[label={\arabic*)}]
	\label{list:cities}
	\item Калининград;
	\item Санкт-Петербург;
	\item Москва;
	\item Нижний Новгород;
	\item Екатеринбург;
	\item Омск;
	\item Новосибирск;
	\item Красноярск;
	\item Иркутск;
	\item Владивосток.
\end{enumerate}

Меткой ребра этого графа будет выступать расстояние между городами в километрах.

Матрица расстояний этого графа (формула \ref{eq:rus_route}):

\begin{equation}
	\label{eq:rus_route}
	K_1 = 
	\begin{pmatrix}
		0 & 826 & 1089 & 1483 & 2483 & 3303 & 3858 & 4362 & 5209 & 7359 \\ 
		826 & 0 & 634 & 896 & 1782 & 2584 & 3105 & 3574 & 4416 & 6538 \\
		1089 & 634 & 0 & 401 & 1415 & 2236 & 2812  & 3354 & 4203 & 6418 \\
		1483 & 896 & 401 & 0 & 1016  & 1835 & 2412  &  2962 & 3810 & 6031 \\
		2483 & 1782 & 1415 & 1016 & 0  & 820 & 1398  &  1968 & 2811 & 5060 \\
		3303 & 2584 & 2236 & 1835 & 820  & 0 &  609  &  1229 & 2043 & 4324 \\
		3858 & 3105 & 2812 & 2412 & 1398  & 609 &  0  &  629 & 1434 & 3716 \\
		4362 & 3574 & 3354 & 2962 & 1968  & 1229 &  629  &  0 & 849  & 3103 \\
		5029 & 4416 & 4203 & 3810 & 2811  & 2043 &  1434  &  849 & 0  & 2285 \\
		7359 & 6538 & 6418 & 6031 & 5060  & 4324 &  3716  &  3103 & 2285  & 0 \\
	\end{pmatrix}
\end{equation}

\clearpage

\section*{Вывод}


