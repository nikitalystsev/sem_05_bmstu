\chapter{Исследовательский раздел}

В данном разделе будет проведен сравнительный анализ алгоритмов при различных ситуациях на
основе полученных данных.

\section{Технические характеристики}

Технические характеристики устройства, на котором проводились исследования: 

\begin{itemize}[label=--]
	\item операционная система: Ubuntu 22.04.3 LTS x86\_64 \cite{info_os};
	\item оперативная память: 16 Гб;
	\item процессор: 11th Gen Intel® Core™ i7-1185G7 @ 3.00 ГГц × 8, 4 физических ядра, 8 логических ядер.
\end{itemize}

\section{Проведение первого исследования}

\section*{Цель исследования}

Целью исследования является проведение сравнительного анализа времени работы алгоритма полного перебора и муравьиного алгоритма от линейного размера матрицы смежности.

\section*{Наборы варьируемых и фиксированных параметров}

Замеры времени проводились для линейных размеров матрицы, равных 2, 3, 4, 5, 6, 7, 8, 9, 10.

В качестве фиксированного параметра был выбран разброс значений для матрицы смежности от 5 до 10. Для муравьиного алгоритма были фиксированы следующие параметры:

\begin{itemize}[label*=--]
	\item $\alpha$ --- параметр влияния видимости пути (значение 0.5);
	\item $\beta$ --- параметр влияния феромона (значение 0.5);

\clearpage

	\item $\rho$ --- коэффициент испарения (значение 0.5);
	\item время жизни муравьиной колонии (значение 250).
\end{itemize}

Замеры времени для каждого линейного размера матрицы проводились 10 раз. 

\section*{Результаты первого исследования}

\begin{table}[ht]
	\small
	\begin{center}
		\begin{threeparttable}
			\caption{Замер времени для  для линейных размеров матрицы от 2 до 10}
			\label{tbl:time}
			\begin{tabular}{|r|r|r|}
				\hline
				& \multicolumn{2}{c|}{\bfseries Время, c} \\ \cline{2-3}
				\bfseries \makecell{Линейный размер матрицы, \\ единицы} & \bfseries \makecell{Алгоритм \\ полного перебора} & \bfseries \makecell{Муравьиный \\ алгоритм} \\ \cline{2-3}
				\hline
		         2 &   0.000019 &   0.000323 \\ \hline
		         3 &   0.000010 &   0.000901 \\ \hline
		         4 &   0.000034 &   0.001879 \\ \hline
		         5 &   0.000156 &   0.003271 \\ \hline
		         6 &   0.001017 &   0.005237 \\ \hline
		         7 &   0.007903 &   0.007688 \\ \hline
		         8 &   0.070830 &   0.010854 \\ \hline
		         9 &   0.708097 &   0.014859 \\ \hline
		        10 &   7.946709 &   0.019937 \\ \hline
			\end{tabular}	
		\end{threeparttable}
	\end{center}
\end{table}

\clearpage

\includesvgimage
{research1} % Имя файла без расширения (файл должен быть расположен в директории inc/img/)
{f} % Обтекание (без обтекания)
{h} % Положение рисунка (см. figure из пакета float)
{1\textwidth} % Ширина рисунка
{Результаты замеров времени для для линейных размеров матрицы от 2 до 10} % Подпись рисунка

На линейных размерах матрицы смежности от 2 до 6 алгоритм полного перебора работает значительно быстрее муравьиного алгоритма (на матрице смежности размером $2\times2$ в 17 раз, а на матрице смежности размером $6\times6$ в 5.14 раза), но на линейных размерах матрицы смежности от 7 до 10 муравьиный алгоритм выигрывает у алгоритма полного перебора по времени выполнения (на матрице смежности размером $7\times7$ в 1.02 раз, а на матрице смежности размером $10\times10$ в уже в 398.59 раза). Это связано с тем, что алгоритм полного перебора имеет вычислительную сложность $O(n!)$, в то время как сложность муравьиного алгоритма $O(t \cdot m \cdot n^2)$, где $t$ --- количество итераций, $m$ --- количество муравьев, $n$ --- число вершин \cite{info_compl_ant_alg}.

\clearpage

\section{Проведение второго исследования}

Автоматическая параметризация была проведена на единственном классе данных, состоящем из трех полносвязных графов с 10 вершинами в каждом с одинаковым разбросом значений меток ребер/дуг графа.

Итоговая таблица значений параметризации будет состоять из следующих колонок:
\begin{itemize}[label=---]
	\item $\alpha$ --- параметр влияния видимости пути;
	\item $\rho$ --- коэффициент испарения;
	\item \textit{days} --- количество дней жизни колонии муравьев;
	\item \textit{Result} --- эталонный результат, полученный методом полного перебора для проведения данного эксперимента;
	\item \textit{Mistake} --- разность полученного основанным на муравьином алгоритме методом значения и эталонного значения на данных значениях параметров, показатель качества решения.
\end{itemize}

\section*{Цель исследования}

Цель исследования --- определить комбинацию параметров, которые позволяют решить задачу наилучшим образом для выбранного класса данных. 
Качество решения зависит от количества дней и погрешности измерений.

\subsection{Граф №1}

Согласно с вариантом индивидуального задания представим полносвязный неориентированный граф, вершинами которого являются города от Калиниграда до Владивостока:

\begin{enumerate}[label={\arabic*)}]
	\label{list:cities}
	\item Калининград;
	\item Санкт-Петербург;
	\item Москва;
	\item Нижний Новгород;
	\item Екатеринбург;
	\item Омск;
	\item Новосибирск;
	\item Красноярск;
	\item Иркутск;
	\item Владивосток.
\end{enumerate}

Меткой ребра этого графа будет выступать расстояние между городами в километрах.

Матрица расстояний этого графа (формула \ref{eq:rus_route}):

\begin{equation}
	\label{eq:rus_route}
	K_1 = 
	\begin{pmatrix}
		0 & 826 & 1089 & 1483 & 2483 & 3303 & 3858 & 4362 & 5209 & 7359 \\ 
		826 & 0 & 634 & 896 & 1782 & 2584 & 3105 & 3574 & 4416 & 6538 \\
		1089 & 634 & 0 & 401 & 1415 & 2236 & 2812  & 3354 & 4203 & 6418 \\
		1483 & 896 & 401 & 0 & 1016  & 1835 & 2412  &  2962 & 3810 & 6031 \\
		2483 & 1782 & 1415 & 1016 & 0  & 820 & 1398  &  1968 & 2811 & 5060 \\
		3303 & 2584 & 2236 & 1835 & 820  & 0 &  609  &  1229 & 2043 & 4324 \\
		3858 & 3105 & 2812 & 2412 & 1398  & 609 &  0  &  629 & 1434 & 3716 \\
		4362 & 3574 & 3354 & 2962 & 1968  & 1229 &  629  &  0 & 849  & 3103 \\
		5029 & 4416 & 4203 & 3810 & 2811  & 2043 &  1434  &  849 & 0  & 2285 \\
		7359 & 6538 & 6418 & 6031 & 5060  & 4324 &  3716  &  3103 & 2285  & 0 \\
	\end{pmatrix}
\end{equation}

Для данного графа приведена таблица \ref{tbl:table_graph1} с выборкой параметров, которые наилучшим образом решают поставленную задачу.

\clearpage

\begin{table}[ht]
	\small
	\begin{center}
		\begin{threeparttable}
			\caption{Выборка из параметров для графа №1}
			\label{tbl:table_graph1}
			\begin{tabular}{|c|c|c|c|c|}
				\hline
				$\alpha$ & $\rho$ & Days & Result & Mistake \\ 
				\hline
				0.1 &  0.1 &   50 &  8069 &     0 \\
				0.1 &  0.1 &  300 &  8069 &     0 \\
				0.1 &  0.2 &   50 &  8069 &     0 \\
				0.1 &  0.2 &  100 &  8069 &     0 \\
				0.1 &  0.2 &  500 &  8069 &     0 \\
				0.1 &  0.3 &    3 &  8069 &     0 \\
				0.1 &  0.3 &  500 &  8069 &     0 \\
				0.1 &  0.4 &  500 &  8069 &     0 \\
				0.1 &  0.5 &  500 &  8069 &     0 \\
				0.1 &  0.6 &  300 &  8069 &     0 \\
				0.1 &  0.6 &  500 &  8069 &     0 \\
				0.1 &  0.7 &   10 &  8069 &     0 \\
				0.1 &  0.7 &  300 &  8069 &     0 \\
				0.1 &  0.7 &  500 &  8069 &     0 \\
				0.1 &  0.8 &   50 &  8069 &     0 \\
				0.1 &  0.8 &  500 &  8069 &     0 \\ \hline
				0.2 &  0.1 &  300 &  8069 &     0 \\
				0.2 &  0.1 &  500 &  8069 &     0 \\
				0.2 &  0.2 &  300 &  8069 &     0 \\
				0.2 &  0.4 &  300 &  8069 &     0 \\
				0.2 &  0.4 &  500 &  8069 &     0 \\
				0.2 &  0.5 &   50 &  8069 &     0 \\
				0.2 &  0.5 &  500 &  8069 &     0 \\
				0.2 &  0.6 &  500 &  8069 &     0 \\
				0.2 &  0.7 &  100 &  8069 &     0 \\
				0.2 &  0.8 &  100 &  8069 &     0 \\
				0.2 &  0.8 &  300 &  8069 &     0 \\
				0.2 &  0.8 &  500 &  8069 &     0 \\ \hline
				0.3 &  0.1 &  300 &  8069 &     0 \\
				0.3 &  0.5 &  500 &  8069 &     0 \\
				0.3 &  0.6 &  500 &  8069 &     0 \\ \hline
				0.4 &  0.3 &  500 &  8069 &     0 \\
				0.4 &  0.4 &  300 &  8069 &     0 \\ \hline
				0.5 &  0.8 &  500 &  8069 &     0 \\ \hline
			\end{tabular}	
		\end{threeparttable}
	\end{center}
\end{table}

\clearpage

\subsection{Граф №2}

Граф №2 представляет собой матрицу смежности размером 10 элементов (разброс значений от 500 до 9999), которая представлена в формуле \ref{eq:mtr_adj_gen1}.

\begin{equation}
	\label{eq:mtr_adj_gen1}
	K_2  
	\begin{pmatrix}
		0  & 2054 & 4772 & 2285 & 8108 & 7511 & 1603 & 4980 &  4946 & 6961 \\
		2054 &  0 & 9820 & 7343 & 1913 & 3186 & 9741 & 9824 & 8436 & 4025 \\
		4772 & 9820 & 0 & 2700 & 7491 & 1448 & 1528 & 5163 & 9493 & 5676 \\
		2285 & 7343 & 2700 & 0 & 7228 & 5495 & 8920 & 6178 & 6876 & 8871 \\
		8108 & 1913 & 7491 & 7228 & 0 & 897 & 3805 & 8851 & 8109 & 7628 \\
		7511 & 3186 & 1448 & 5495 & 897 & 0 & 9043 & 7877 & 7899 & 1560 \\
		1603 & 9741 & 1528 & 8920 & 3805 & 9043 & 0 & 3334 & 8791 & 5418 \\
		4980 & 9824 & 5163 & 6178 & 8851 & 7877 & 3334 & 0 & 4370 & 633 \\
		4946 & 8436 & 9493 & 6876 & 8109 & 7899 & 8791 & 4370 & 0 & 3545 \\
		6961 & 4025 & 5676 & 8871 & 7628 & 1560 & 5418 & 633 & 3545 & 0 \\
	\end{pmatrix}
\end{equation}

Для данного графа приведена таблица \ref{tbl:table_graph2_part1} с выборкой параметров, которые наилучшим образом решают поставленную задачу.

\clearpage

\begin{table}[ht]
	\small
	\begin{center}
		\begin{threeparttable}
			\caption{Выборка из параметров для графа №2 (начало)}
			\label{tbl:table_graph2_part1}
			\begin{tabular}{|c|c|c|c|c|}
				\hline
				$\alpha$ & $\rho$ & Days & Result & Mistake \\ 
				\hline
				 0.1 &  0.1 &  100 & 17258 &     0 \\
				 0.1 &  0.1 &  300 & 17258 &     0 \\
				 0.1 &  0.1 &  500 & 17258 &     0 \\
				 0.1 &  0.2 &   50 & 17258 &     0 \\
				 0.1 &  0.2 &  500 & 17258 &     0 \\
				 0.1 &  0.3 &   50 & 17258 &     0 \\
				 0.1 &  0.3 &  300 & 17258 &     0 \\
				 0.1 &  0.3 &  500 & 17258 &     0 \\
				 0.1 &  0.4 &  100 & 17258 &     0 \\
				 0.1 &  0.4 &  500 & 17258 &     0 \\
				 0.1 &  0.5 &  300 & 17258 &     0 \\
				 0.1 &  0.5 &  500 & 17258 &     0 \\
				 0.1 &  0.6 &  300 & 17258 &     0 \\
				 0.1 &  0.7 &  100 & 17258 &     0 \\
				 0.1 &  0.7 &  300 & 17258 &     0 \\
				 0.1 &  0.8 &  300 & 17258 &     0 \\
				 0.1 &  0.8 &  500 & 17258 &     0 \\ \hline
				 0.2 &  0.1 &  300 & 17258 &     0 \\
				 0.2 &  0.1 &  500 & 17258 &     0 \\
				 0.2 &  0.4 &  500 & 17258 &     0 \\
				 0.2 &  0.5 &  300 & 17258 &     0 \\
				 0.2 &  0.5 &  500 & 17258 &     0 \\
				 0.2 &  0.6 &  100 & 17258 &     0 \\
				 0.2 &  0.6 &  300 & 17258 &     0 \\
				 0.2 &  0.6 &  500 & 17258 &     0 \\
				 0.2 &  0.7 &  500 & 17258 &     0 \\
				 0.2 &  0.8 &  500 & 17258 &     0 \\ \hline
				 0.3 &  0.2 &    3 & 17258 &     0 \\
				 0.3 &  0.2 &  300 & 17258 &     0 \\
				 0.3 &  0.2 &  500 & 17258 &     0 \\
				 0.3 &  0.6 &  300 & 17258 &     0 \\
				 0.3 &  0.6 &  500 & 17258 &     0 \\
				 0.3 &  0.7 &   50 & 17258 &     0 \\
				 0.3 &  0.7 &  100 & 17258 &     0 \\
				 0.3 &  0.7 &  500 & 17258 &     0 \\
				 0.3 &  0.8 &  500 & 17258 &     0 \\ \hline
			\end{tabular}	
		\end{threeparttable}
	\end{center}
\end{table}

\clearpage

\begin{table}[ht]
	\small
	\begin{center}
		\begin{threeparttable}
			\caption{Выборка из параметров для графа №2 (конец)}
			\label{tbl:table_graph2_part2}
			\begin{tabular}{|c|c|c|c|c|}
				\hline
				$\alpha$ & $\rho$ & Days & Result & Mistake \\ 
				\hline
				0.4 &  0.3 &  500 & 17258 &     0 \\
				0.4 &  0.6 &  100 & 17258 &     0 \\
				0.4 &  0.6 &  300 & 17258 &     0 \\
				0.4 &  0.6 &  500 & 17258 &     0 \\
				0.4 &  0.8 &  500 & 17258 &     0 \\ \hline
				0.5 &  0.7 &  500 & 17258 &     0 \\
				0.5 &  0.8 &  500 & 17258 &     0 \\ \hline
				0.6 &  0.5 &   50 & 17258 &     0 \\ \hline
			\end{tabular}	
		\end{threeparttable}
	\end{center}
\end{table}


\subsection{Граф №3}

Граф №3 представляет собой матрицу смежности размером 10 элементов (разброс значений от 500 до 9999), которая представлена в формуле \ref{eq:mtr_adj_gen2}.

\begin{equation}
	\label{eq:mtr_adj_gen2}
	K_3       
	\begin{pmatrix}
		0  & 7350  & 4050  & 1676  & 6394  & 4900  & 5605 &  8001 &  8137 &  1936 \\
		7350  & 0  & 705  & 5103  & 1038  & 4199  &  1786  & 3492  & 6111 &  5218 \\
		4050  & 705  & 0  & 6688  & 9899  &  4109  & 3122  & 8410  & 1078  &  4511 \\
		1676  & 5103  & 6688  & 0 &  8275  & 9030  & 1376  & 7076  & 8435 &  8500 \\
		6394  & 1038  & 9899  & 8275  & 0  & 7240  & 2432  & 4687  & 1021  & 9136 \\
		4900  & 4199  & 4109  & 9030  & 7240  & 0  & 6837  & 2817  & 1987  & 4830 \\
		5605  & 1786  & 3122  & 1376  & 2432  & 6837  & 0  & 7360  & 4111  & 7236 \\
		8001  & 3492  & 8410  & 7076  & 4687  & 2817  & 7360  & 0  & 9470  & 8066 \\
		8137  & 6111  & 1078  & 8435  & 1021  & 1987  & 4111  & 9470  & 0  & 1313 \\
		1936  & 5218  & 4511  & 8500  & 9136  & 4830  & 7236  & 8066  & 1313  & 0 \\
	\end{pmatrix}
\end{equation}

Для данного графа приведена таблица \ref{tbl:table_graph3_part1} с выборкой параметров, которые наилучшим образом решают поставленную задачу. 

\clearpage

\begin{table}[ht]
	\small
	\begin{center}
		\begin{threeparttable}
			\caption{Выборка из параметров для графа №3}
			\label{tbl:table_graph3_part1}
			\begin{tabular}{|c|c|c|c|c|}
				\hline
				$\alpha$ & $\rho$ & Days & Result & Mistake \\ 
				\hline
				0.1 &  0.1 &  300 & 15045 &     0 \\
				0.1 &  0.1 &  500 & 15045 &     0 \\
				0.1 &  0.2 &  300 & 15045 &     0 \\
				0.1 &  0.2 &  500 & 15045 &     0 \\
				0.1 &  0.3 &  300 & 15045 &     0 \\
				0.1 &  0.3 &  500 & 15045 &     0 \\
				0.1 &  0.4 &  500 & 15045 &     0 \\
				0.1 &  0.5 &  100 & 15045 &     0 \\
				0.1 &  0.5 &  500 & 15045 &     0 \\
				0.1 &  0.6 &  300 & 15045 &     0 \\
				0.1 &  0.6 &  500 & 15045 &     0 \\
				0.1 &  0.7 &  300 & 15045 &     0 \\
				0.1 &  0.8 &   50 & 15045 &     0 \\ \hline
				0.2 &  0.1 &  300 & 15045 &     0 \\
				0.2 &  0.1 &  500 & 15045 &     0 \\
				0.2 &  0.2 &   50 & 15045 &     0 \\
				0.2 &  0.2 &  500 & 15045 &     0 \\
				0.2 &  0.3 &  300 & 15045 &     0 \\
				0.2 &  0.5 &  500 & 15045 &     0 \\
				0.2 &  0.6 &  300 & 15045 &     0 \\
				0.2 &  0.6 &  500 & 15045 &     0 \\
				0.2 &  0.8 &  500 & 15045 &     0 \\ \hline
				0.3 &  0.2 &  300 & 15045 &     0 \\
				0.3 &  0.2 &  500 & 15045 &     0 \\
				0.3 &  0.4 &  500 & 15045 &     0 \\
				0.3 &  0.5 &  500 & 15045 &     0 \\
				0.3 &  0.7 &  300 & 15045 &     0 \\
				0.3 &  0.8 &   50 & 15045 &     0 \\
				0.3 &  0.8 &  500 & 15045 &     0 \\ \hline
				0.5 &  0.2 &  300 & 15045 &     0 \\
				0.5 &  0.4 &  500 & 15045 &     0 \\
				0.5 &  0.5 &  100 & 15045 &     0 \\
				0.5 &  0.6 &  300 & 15045 &     0 \\
				0.5 &  0.6 &  500 & 15045 &     0 \\ \hline
				0.6 &  0.6 &  300 & 15045 &     0 \\
				0.6 &  0.7 &  500 & 15045 &     0 \\ \hline
				0.7 &  0.7 &  500 & 15045 &     0 \\ \hline
			\end{tabular}	
		\end{threeparttable}
	\end{center}
\end{table}

\clearpage

\section*{Вывод}

В результате исследования было получено, что использование муравьиного алгоритма наиболее эффективно при больших размерах матриц. 
Так, на матрице смежности размером $2\times2$ в 17 раз, а на матрице смежности размером $6\times6$ в 5.14 раза), но на линейных размерах матрицы смежности от 7 до 10 муравьиный алгоритм выигрывает у алгоритма полного перебора по времени выполнения (на матрице смежности размером $7\times7$ в 1.02 раз, а на матрице смежности размером $10\times10$ в уже в 398.59 раза).
Следовательно, при размерах матриц больше 9 следует использовать муравьиный алгоритм, но стоит учитывать, что он не гарантирует получения глобального оптимума при решении задачи.

Для класса данных было получено, что наилучшим образом алгоритм работает на значениях параметров, которые представлены далее:

\begin{itemize}[label=---]
	\item $\alpha = 0.1, \rho = 0.1, 0.2, 0.3, 0.4, 0.5, 0.6, 0.7, 0.8$;
	\item $\alpha = 0.2, \rho = 0.1, 0.4, 0.5, 0.6, 0.8$;
	\item $\alpha = 0.3, \rho = 0.6$;
\end{itemize} 

Для этого класса данных рекомендуется использовать данные параметры.

