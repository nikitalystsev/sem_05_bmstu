\chapter{Конструкторский раздел}

В данном разделе будут представлены схемы алгоритма полного перебора и муравьиного алгоритма.

\section{Разработка алгоритмов}

\subsection{Алгоритм полного перебора}

На рисунке \ref{img:brute-force-alg} представлена схема алгоритма полного перебора для решения задачи коммивояжера.

\includesvgimage
{brute-force-alg} % Имя файла без расширения (файл должен быть расположен в директории inc/img/)
{f} % Обтекание (без обтекания)
{h} % Положение рисунка (см. figure из пакета float)
{0.6\textwidth} % Ширина рисунка
{Схема алгоритма полного перебора} % Подпись рисунка	

\subsection{Муравьиный алгоритм}

На рисунке \ref{img:ant-alg} представлена схема муравьиного алгоритма для решения задачи коммивояжера.

\includesvgimage
{ant-alg} % Имя файла без расширения (файл должен быть расположен в директории inc/img/)
{f} % Обтекание (без обтекания)
{h} % Положение рисунка (см. figure из пакета float)
{0.8\textwidth} % Ширина рисунка
{Схема муравьиного алгоритма} % Подпись рисунка	

\clearpage

\section{Структура разрабатываемого ПО}

Для реализации разрабатываемого программного обеспечения будет использоваться
метод структурного программирования. 
Каждый из алгоритмов будет представлен отдельной функцией, при необходимости будут выделены подпрограммы для каждой из них. 
Также будут реализованы функции для ввода-вывода и функция, вызывающая все подпрограммы для связности и полноценности программы.

\section*{Вывод}

В данном разделе были представлены схемы алгоритма полного перебора и муравьиного алгоритма, была
описана структура разрабатываемого ПО.



