\chapter*{ЗАКЛЮЧЕНИЕ}
\addcontentsline{toc}{chapter}{ЗАКЛЮЧЕНИЕ}

В ходе выполнения лабораторной работы были решены следующие задачи:


\begin{enumerate}[label={\arabic*)}]
	\item изучены и описаны задачу коммивояжера;
	\item изучены и описаны методы решения задачи коммивояжера ---  метод полного перебора и метод на основе муравьиного алгоритма;
	\item разработан и реализован программный продукт, позволяющий решить задачу коммивояжера исследуемыми методами;
	\item проведено сравнение по времени метод полного перебора и метод на основе муравьиного алгоритма.
	\item обоснованы полученные результаты в отчете о выполненной лабораторной работе.
\end{enumerate}

Цель данной лабораторной работы, а описание методов решения задачи коммивояжера полным перебором и на основе муравьиного алгоритма, также была достигнута.

Исходя из полученных результатов можно сказать, что использование муравьиного алгоритма наиболее эффективно при больших размерах матриц. 
Так, на матрице смежности размером $2\times2$ в 17 раз, а на матрице смежности размером $6\times6$ в 5.14 раза), но на линейных размерах матрицы смежности от 7 до 10 муравьиный алгоритм выигрывает у алгоритма полного перебора по времени выполнения (на матрице смежности размером $7\times7$ в 1.02 раз, а на матрице смежности размером $10\times10$ в уже в 398.59 раза).
Следовательно, при размерах матриц больше 9 следует использовать муравьиный алгоритм, но стоит учитывать, что он не гарантирует получения глобального оптимума при решении задачи.
