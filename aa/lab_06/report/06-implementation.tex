\chapter{Технологический раздел}

В данном разделе будут перечислены требования к программному обеспечению, средства реализации и листинги кода.

\section{Требования к программному обеспечению}

К программе предъявляется ряд требований:

\begin{itemize} [label=--]
	\item программа должна получать на вход матрицу смежности, для которой можно будет выбрать один из алгоритмов поиска оптимальных
	путей (полным перебором или муравьиным алгоритмом);
	\item программа должна позволять пользователю определять коэффициенты и количество дней для муравьиного алгоритма;
	\item программа должна давать возможность получить минимальную сумму пути, а также сам путь, используя один из алгоритмов.
\end{itemize}

\section{Средства реализации}

В качестве языка программирования для этой лабораторной работы был выбран $Python$ \cite{info_pl} по следующим причинам:

\begin{itemize}[label=--]
	\item в $Python$ есть библиотека $itertools$ \cite{info_itertools}, содержащая функцию $permutations$ для вычисления всех перестановок переданной последовательности;
	\item в $Python$ есть библиотека $numpy$ \cite{info_numpy}, предоставляющая необходимый функционал для работы с многомерными массивами;
	\item в $Python$ есть библиотека $time$ \cite{info_cpu_time_func}, содержащая функцию $process\_time$ для замеров процессорного времени.
\end{itemize}

В качестве функции, которая будет осуществлять замеры процессорного времени, будет использована функция $process\_time$ из библиотека $time$ \cite{info_cpu_time_func}.

\section{Сведения о модулях программы}

Программа состоит из пяти модулей: 

\begin{itemize}[label*=--]
	\item \texttt{main.py} --- файл, содержащий интерфейс программы и точку входа;
	\item \texttt{brute\_force\_alg.py} --- файл, содержащий код алгоритма полного перебора;
	\item \texttt{ant\_alg.py} --- файл, содержащий код муравьиного алгоритма;
	\item \texttt{utils.py} --- файл, содержащий служебные алгоритмы;
	\item \texttt{measurements.py} --- файл, содержащий функции для выполнения замеров времени.
\end{itemize}

\section{Реализации алгоритмов}

В листинге \ref{lst:brute-force-alg.txt} представлена реализация алгоритма полного перебора.

В листингах \ref{lst:ant-alg-part1.txt} и \ref{lst:ant-alg-part2.txt} представлена реализация муравьиного алгоритма.

В листинге \ref{lst:calc-Q.txt} представлена реализация функции вычисления константы $Q$.

В листинге \ref{lst:init-mtr-phero-mtr-visib.txt} представлена реализация функции создания и инициализации матрицы феромонов и матрицы видимости.

В листингах \ref{lst:prob-part1.txt} и \ref{lst:prob-part2.txt} представлена реализация функций для подсчета необходимых вероятностей.

В листингах \ref{lst:ant-route-part1.txt} и \ref{lst:ant-route-part2.txt} представлена реализация функций для определения маршрута муравья.

В листингах \ref{lst:update-phero-part1.txt} и \ref{lst:update-phero-part2.txt} представлена реализация функции для обновления матрицы феромонов.

\clearpage

\includelisting
{brute-force-alg.txt} % Имя файла с расширением (файл должен быть расположен в директории inc/lst/)
{Реализация алгоритма полного перебора} % Подпись листинга

\includelisting
{ant-alg-part1.txt} % Имя файла с расширением (файл должен быть расположен в директории inc/lst/)
{Реализация муравьиного алгоритма (начало)} % Подпись листинга

\clearpage

\includelisting
{ant-alg-part2.txt} % Имя файла с расширением (файл должен быть расположен в директории inc/lst/)
{Реализация муравьиного алгоритма (конец)} % Подпись листинга

\includelisting
{calc-Q.txt} % Имя файла с расширением (файл должен быть расположен в директории inc/lst/)
{Реализация функции для вычисления константы $Q$} % Подпись листинга

\includelisting
{init-mtr-phero-mtr-visib.txt} % Имя файла с расширением (файл должен быть расположен в директории inc/lst/)
{Реализация функции создания и инициализации матрицы феромонов и матрицы видимости} % Подпись листинга

\includelisting
{prob-part1.txt} % Имя файла с расширением (файл должен быть расположен в директории inc/lst/)
{Реализация функций подсчета необходимых вероятностей (начало)} % Подпись листинга

\clearpage

\includelisting
{prob-part2.txt} % Имя файла с расширением (файл должен быть расположен в директории inc/lst/)
{Реализация функций подсчета необходимых вероятностей (конец)} % Подпись листинга

\includelisting
{ant-route-part1.txt} % Имя файла с расширением (файл должен быть расположен в директории inc/lst/)
{Реализация функции для определения следующего города} % Подпись листинга

\clearpage

\includelisting
{ant-route-part2.txt} % Имя файла с расширением (файл должен быть расположен в директории inc/lst/)
{Реализация функции для определения маршрута муравья} % Подпись листинга

\includelisting
{update-phero-part1.txt} % Имя файла с расширением (файл должен быть расположен в директории inc/lst/)
{Реализация функции для обновления матрицы феромонов (начало)} % Подпись листинга

\includelisting
{update-phero-part2.txt} % Имя файла с расширением (файл должен быть расположен в директории inc/lst/)
{Реализация функции для обновления матрицы феромонов (конец)} % Подпись листинга

\section*{Вывод}

В данном разделе были перечислены требования к программному обеспечению, средства реализации и листинги кода.

    