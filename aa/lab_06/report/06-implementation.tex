\chapter{Технологический раздел}

В данном разделе будут перечислены требования к программному обеспечению, средства реализации и листинги кода.

\section{Требования к программному обеспечению}

К программе предъявляется ряд требований:

\begin{itemize} [label=--]
	\item программа должна получать на вход матрицу смежности, для которой можно будет выбрать один из алгоритмов поиска оптимальных
	путей (полным перебором или муравьиным алгоритмом);
	\item программа должна позволять пользователю определять коэффициенты и количество дней для муравьиного алгоритма;
	\item программа должна давать возможность получить минимальную сумму пути, а также сам путь, используя один из алгоритмов.
\end{itemize}

\section{Средства реализации}

В качестве языка программирования для этой лабораторной работы был выбран $Python$ \cite{info_pl} по следующим причинам:

\begin{itemize}[label=--]
	\item в $Python$ есть библиотека $itertools$ \cite{info_itertools}, содержащая функцию $permutations$ для вычисления всех перестановок переданной последовательности;
	\item в $Python$ есть библиотека $numpy$ \cite{info_numpy}, предоставляющая необходимый функционал для работы с многомерными массивами;
	\item в $Python$ есть библиотека $time$ \cite{info_cpu_time_func}, содержащая функцию $process\_time$ для замеров процессорного времени.
\end{itemize}

В качестве функции, которая будет осуществлять замеры процессорного времени, будет использована функция $process\_time$ из библиотека $time$ \cite{info_cpu_time_func}.

\section{Сведения о модулях программы}

Программа состоит из семи модулей: 

\begin{enumerate}[label={\arabic*)}]
	\item \texttt{main.py} % доработать
\end{enumerate}

\section{Реализации алгоритмов}

%В листингах \ref{lst:serialConveyorPart1.txt} и \ref{lst:serialConveyorPart2.txt} представлена реализация последовательной конвейерной обработки.
%
%\includelisting
%{serialConveyorPart1.txt} % Имя файла с расширением (файл должен быть расположен в директории inc/lst/)
%{Реализация последовательной конвейерной обработки (начало)} % Подпись листинга

\section*{Вывод}

В данном разделе были перечислены требования к программному обеспечению, средства реализации и листинги кода.

    