\chapter{Технологический раздел}

В данном разделе будут перечислены средства реализации, листинги кода и функциональные тесты.

\section{Средства реализации}

В качестве языка программирования для этой лабораторной работы был выбран $C++$ \cite{pl} по следующим причинам:

\begin{itemize}[label=--]
	\item в $C++$ есть встроенный модуль $ctime$, предоставляющий необходимый функционал для замеров процессорного времени;
	\item в стандартной библиотеке $C++$ есть оператор $sizeof$, позволяющий получить размер переданного объекта в байтах. Следовательно, $C++$ предоставляет возможности для проведения точных оценок по используемой памяти.
\end{itemize}

В качестве функции, которая будет осуществлять замеры процессорного времени, будет использована функция $clock\_gettime$ из встроенного модуля $ctime$ \cite{cpu_time_func}.

\section{Сведения о модулях программы}

Программа состоит из шести модулей: 

\begin{enumerate}[label={\arabic*)}]
	\item \texttt{algorithms.cpp} --- модуль, хранящий реализации алгоритмов умножения матриц;
	\item \texttt{processTime.cpp} --- модуль, содержащий функцию для замера процессорного времени;
	\item \texttt{memoryMeasurements.cpp} --- модуль, содержащий функции, позволяющие провести сравнительный анализ использования памяти в реализациях алгоритмов умножения матриц;
	\item \texttt{timeMeasurements.cpp} --- модуль, содержащий функции, позволяющие провести сравнительный анализ использования времени в реализациях алгоритмов умножения матриц;
	\item \texttt{main.cpp} --- файл, содержащий точку входа в программу;
	\item \texttt{task7} --- модуль, содержащий набор скриптов для проведения замеров программы по времени и памяти и построения графиков по полученным данным.
\end{enumerate}

\section{Реализации алгоритмов}

В листингах \ref{lst:classicMatrixMul.txt} - \ref{lst:strassenMulPart4.txt} приведены реализации алгоритмов умножения матриц.

\includelisting
{classicMatrixMul.txt} % Имя файла с расширением (файл должен быть расположен в директории inc/lst/)
{Реализация классического алгоритма умножения двух матриц} % Подпись листинга

\clearpage

\includelisting
{vinogradMatrixMul.txt} % Имя файла с расширением (файл должен быть расположен в директории inc/lst/)
{Реализация алгоритма Винограда \textbf{для} умножения двух матриц} % Подпись листинга

\clearpage

\includelisting
{vinogradMatrixMulWithOpt.txt} % Имя файла с расширением (файл должен быть расположен в директории inc/lst/)
{Реализация оптимизированного алгоритма Винограда для умножения двух матриц} % Подпись листинга

\clearpage

\includelisting
{strassenMulPart1.txt} % Имя файла с расширением (файл должен быть расположен в директории inc/lst/)
{Реализация функций, необходимых для работы алгоритма Штрассена} % Подпись листинга

Функции $split$ выполняет заполнение переданной подматрицы $B$ необходимыми значениями из основной матрицы $A$, а функция $join$ выполняет заполнение переданной результирующей матрицы $B$ необходимыми значениями из подматрицы $A$. Функции $add$ и $sub$ выполняют сложение и вычитание матриц $A$ и $B$, и результат записывается в матрицу $C$.

\includelisting
{strassenMulPart2.txt} % Имя файла с расширением (файл должен быть расположен в директории inc/lst/)
{Реализация алгоритма Штрассена (начало)} % Подпись листинга

\includelisting
{strassenMulPart3.txt} % Имя файла с расширением (файл должен быть расположен в директории inc/lst/)
{Реализация алгоритма Штрассена (продолжение)} % Подпись листинга

\includelisting
{strassenMulPart4.txt} % Имя файла с расширением (файл должен быть расположен в директории inc/lst/)
{Реализация алгоритма Штрассена (конец)} % Подпись листинга

\clearpage

\section{Функциональные тесты}

В таблице \ref{tbl:func_tests_std}, \ref{tbl:func_tests_vin} и \ref{tbl:func_tests_stras} приведены функциональные тесты для разработанных алгоритмов умножения матриц. Все тесты пройдены успешно.

\begin{table}[ht]
	\small
	\begin{center}
		\begin{threeparttable}
			\caption{Функциональные тесты для классического алгоритма умножения матриц}
			\label{tbl:func_tests_std}
			\begin{tabular}{|c|c|c|c|c|}
				\hline
				\multicolumn{2}{|c|}{\bfseries Входные данные}
				& \multicolumn{2}{c|}{\bfseries Результат для классического алгоритма} \\
				\hline 
				\bfseries Матрица 1
				& \bfseries Матрица 2
				& \bfseries Ожидаемый результат
				& \bfseries Фактический результат \\
				\hline
				$\begin{pmatrix}
					1 & 5 & 7\\
					2 & 6 & 8\\
					3 & 7 & 9
				\end{pmatrix}$ 
				&  
				$\begin{pmatrix}
					&
				\end{pmatrix}$
				&
				\text{Сообщение об ошибке}
				&
				\text{Сообщение об ошибке} \\ 
				\hline
				$\begin{pmatrix}
					1 & 5 & 7\\
				\end{pmatrix}$ 
				&  
				$\begin{pmatrix}
					1 & 2 & 3\\
				\end{pmatrix}$
				&
				\text{Сообщение об ошибке}
				&
				\text{Сообщение об ошибке} \\ 
				\hline
				$\begin{pmatrix}
					1 & 2 & 3\\
					4 & 5 & 6 \\
					7 & 8 & 9 \\
				\end{pmatrix}$ 
				&  
				$\begin{pmatrix}
					1 & 0 & 0\\
					0 & 1 & 0 \\
					0 & 0 & 1 \\
				\end{pmatrix}$
				&
				$\begin{pmatrix}
					1 & 0 & 0\\
					0 & 1 & 0 \\
					0 & 0 & 1 \\
				\end{pmatrix}$
				&
				$\begin{pmatrix}
					1 & 2 & 3\\
					4 & 5 & 6 \\
					7 & 8 & 9 \\
				\end{pmatrix}$ \\ 
				\hline
				$\begin{pmatrix}
					3 & 5\\
					2 & 1\\
					9 & 7\\
				\end{pmatrix}$
				&
				$\begin{pmatrix}
					1 & 2 & 3\\
					4 & 5 & 6 \\
				\end{pmatrix}$
				&
				$\begin{pmatrix}
					23 & 31 & 39 \\
					6 & 9 & 12
				\end{pmatrix}$ 
				&
				$\begin{pmatrix}
					23 & 31 & 39 \\
					6 & 9 & 12
				\end{pmatrix}$ \\ 
				\hline
				$\begin{pmatrix}
					10
				\end{pmatrix}$
				&
				$\begin{pmatrix}
					35
				\end{pmatrix}$
				&
				$\begin{pmatrix}
					350
				\end{pmatrix}$ 
				&
				$\begin{pmatrix}
					350
				\end{pmatrix}$ \\ 
				\hline
			\end{tabular}
		\end{threeparttable}
	\end{center}
\end{table}

\begin{table}[ht]
	\small
	\begin{center}
		\begin{threeparttable}
			\caption{Функциональные тесты для умножения матриц по алгоритму Винограда}
			\label{tbl:func_tests_vin}
			\begin{tabular}{|c|c|c|c|c|}
				\hline
				\multicolumn{2}{|c|}{\bfseries Входные данные}
				& \multicolumn{2}{c|}{\bfseries Результат для алгоритма Винограда} \\
				\hline 
				\bfseries Матрица 1
				& \bfseries Матрица 2
				& \bfseries Ожидаемый результат
				& \bfseries Фактический результат \\
				\hline
				$\begin{pmatrix}
					1 & 5 & 7\\
					2 & 6 & 8\\
					3 & 7 & 9
				\end{pmatrix}$ 
				&  
				$\begin{pmatrix}
					&
				\end{pmatrix}$
				&
				\text{Сообщение об ошибке}
				&
				\text{Сообщение об ошибке} \\ 
				\hline
				$\begin{pmatrix}
					1 & 5 & 7\\
				\end{pmatrix}$ 
				&  
				$\begin{pmatrix}
					1 & 2 & 3\\
				\end{pmatrix}$
				&
				\text{Сообщение об ошибке}
				&
				\text{Сообщение об ошибке} \\ 
				\hline
				$\begin{pmatrix}
					1 & 2 & 3\\
					4 & 5 & 6 \\
					7 & 8 & 9 \\
				\end{pmatrix}$ 
				&  
				$\begin{pmatrix}
					1 & 0 & 0\\
					0 & 1 & 0 \\
					0 & 0 & 1 \\
				\end{pmatrix}$
				&
				$\begin{pmatrix}
					1 & 0 & 0\\
					0 & 1 & 0 \\
					0 & 0 & 1 \\
				\end{pmatrix}$
				&
				$\begin{pmatrix}
					1 & 2 & 3\\
					4 & 5 & 6 \\
					7 & 8 & 9 \\
				\end{pmatrix}$ \\ 
				\hline
				$\begin{pmatrix}
					3 & 5\\
					2 & 1\\
					9 & 7\\
				\end{pmatrix}$
				&
				$\begin{pmatrix}
					1 & 2 & 3\\
					4 & 5 & 6 \\
				\end{pmatrix}$
				&
				$\begin{pmatrix}
					23 & 31 & 39 \\
					6 & 9 & 12
				\end{pmatrix}$ 
				&
				$\begin{pmatrix}
					23 & 31 & 39 \\
					6 & 9 & 12
				\end{pmatrix}$ \\ 
				\hline
				$\begin{pmatrix}
					10
				\end{pmatrix}$
				&
				$\begin{pmatrix}
					35
				\end{pmatrix}$
				&
				$\begin{pmatrix}
					350
				\end{pmatrix}$ 
				&
				$\begin{pmatrix}
					350
				\end{pmatrix}$ \\ 
				\hline
			\end{tabular}
		\end{threeparttable}
	\end{center}
\end{table}

\begin{table}[ht]
	\small
	\begin{center}
		\begin{threeparttable}
			\caption{Функциональные тесты для умножения матриц по алгоритму Штрассена}
			\label{tbl:func_tests_stras}
			\begin{tabular}{|c|c|c|c|c|}
				\hline
				\multicolumn{2}{|c|}{\bfseries Входные данные}
				& \multicolumn{2}{c|}{\bfseries Результат для алгоритма Штрассена} \\
				\hline 
				\bfseries Матрица 1
				& \bfseries Матрица 2
				& \bfseries Ожидаемый результат
				& \bfseries Фактический результат \\
				\hline
				$\begin{pmatrix}
					1 & 2 \\
					3 & 4 
				\end{pmatrix}$ 
				&  
				$\begin{pmatrix}
					5 & 6 \\
					7 & 8 
				\end{pmatrix}$
				&
				$\begin{pmatrix}
					19 & 22 \\
					43 & 50
				\end{pmatrix}$
				&
				$\begin{pmatrix}
					19 & 22 \\
					43 & 50
				\end{pmatrix}$ \\
				\hline
				$\begin{pmatrix}
					1 & 5 & 7\\
				\end{pmatrix}$ 
				&  
				$\begin{pmatrix}
					1 & 2 & 3\\
				\end{pmatrix}$ 
				&
				\text{Сообщение об ошибке}
				&
				\text{Сообщение об ошибке} \\ 
				\hline
				$\begin{pmatrix}
					1 & 2 & 3\\
					4 & 5 & 6 \\
					7 & 8 & 9 \\
				\end{pmatrix}$ 
				&  
				$\begin{pmatrix}
					1 & 0 & 0\\
					0 & 1 & 0 \\
					0 & 0 & 1 \\
				\end{pmatrix}$
				&
				$\begin{pmatrix}
					1 & 0 & 0\\
					0 & 1 & 0 \\
					0 & 0 & 1 \\
				\end{pmatrix}$
				&
				$\begin{pmatrix}
					1 & 2 & 3\\
					4 & 5 & 6 \\
					7 & 8 & 9 \\
				\end{pmatrix}$ \\ 
				\hline
				$\begin{pmatrix}
					3 & 5\\
					2 & 1\\
					9 & 7\\
				\end{pmatrix}$
				&
				$\begin{pmatrix}
					1 & 2 & 3\\
					4 & 5 & 6 \\
				\end{pmatrix}$
				&
				\text{Сообщение об ошибке}
				&
				\text{Сообщение об ошибке} \\ 
				\hline
				$\begin{pmatrix}
					10
				\end{pmatrix}$
				&
				$\begin{pmatrix}
					35
				\end{pmatrix}$
				&
				$\begin{pmatrix}
					350
				\end{pmatrix}$ 
				&
				$\begin{pmatrix}
					350
				\end{pmatrix}$ \\ 
				\hline
			\end{tabular}
		\end{threeparttable}
	\end{center}
\end{table}

\section*{Вывод}

В данном разделе были реализованы и протестированы 4 алгоритма:
классический алгоритм умножения, алгоритм Винограда для умножения двух матриц, оптимизированный алгоритм Винограда умножения двух матриц и алгоритм Штрассена умножения двух матриц.

    