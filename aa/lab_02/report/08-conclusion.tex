\chapter*{ЗАКЛЮЧЕНИЕ}
\addcontentsline{toc}{chapter}{ЗАКЛЮЧЕНИЕ}

В ходе выполнения лабораторной работы были решены следующие задачи:

\begin{enumerate}[label={\arabic*)}]
	\item Изучены и описаны алгоритмы классического умножения матриц и умножения матриц с использованием алгоритма Винограда и Штрассена.
	\item Создано программное обеспечение, реализующее следующие алгоритмы:
	\begin{itemize}[label=--]
		\item классический алгоритм умножения матриц;
		\item умножение матриц с использованием алгоритма Винограда;
		\item умножение матриц с использованием оптимизированной версии алгоритма Винограда;
		\item умножение матриц с использованием алгоритма Штрассена.
	\end{itemize}
	
	\item Проведен анализ эффективности реализаций алгоритмов по памяти и по времени.
	\item Проведена оценка сложности алгоритмов и сказать влияние оптимизаций на характеристики программной реализации.
	\item Подготовлен отчет по лабораторной работе.
\end{enumerate}

Цель данной лабораторной работы, а именно исследование классического алгоритма умножения матриц, умножения матриц с использованием алгоритма Винограда и его оптимизированной версии согласно варианту, а также умножения матриц с использованием алгоритма Штрассена, также была достигнута.

Согласно теоретическим расчетам трудоемкости алгоритмов умножения матриц наименее трудоемким оказался алгоритм Штрассена, наиболее трудоемким -- неоптимизированная версия алгоритма Винограда, однако результаты оценок работы алгоритмов по памяти и по времени оказались противоположными: оптимизированная версия алгоритма Винограда оказалась самой эффективной по времени среди всех алгоритмов, в то время как алгоритм Штрассена оказался самым неэффективным по времени. По количеству расходуемой памяти самым эффективным оказался классический алгоритм, а самым неээффективным -- алгоритм Штрассена. Это связано с тем, что в классическом алгоритме умножения двух матриц не проиходит вычисления промежуточных слагаемых, в то время как при каждом рекурсивном вызове в алгоритме Штрассена для подматриц выделяется память под их хранение, а также выделяется память для хранения матриц $M_i$, $i = \overline{1, 7}$.

Для двух матриц порядка $n \times n$ алгоритм Винограда имеет асимптотическую сложность $O(n^{2.3755})$, классический алгоритм -- $O(n^3)$, алгоритм Штрассена -- $O(n^{2.807})$.

Применение оптимизаций в алгоритме Винограда позволили уменьшить время работы алгоритма, но незначительно увеличить расход используемой памяти.
