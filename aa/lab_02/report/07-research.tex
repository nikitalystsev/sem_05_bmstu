\chapter{Исследовательский раздел}

В данном разделе будут проведены сравнения реализаций алгоритмов умножения матриц  по времени работы и по затрачиваемой памяти.

\section{Технические характеристики}

Технические характеристики устройства, на котором проводились исследования: 

\begin{itemize}[label=--]
	\item операционная система: Ubuntu 22.04.3 LTS x86\_64 \cite{os};
	\item оперативная память: 16 Гб;
	\item процессор: 11th Gen Intel® Core™ i7-1185G7 @ 3.00 ГГц × 8.
\end{itemize}

\section{Время выполнения алгоритмов}

Время работы алгоритмов измерялось с использованием функции $clock\_gettime$ из встроенного модуля $ctime$. 

Замеры времени для каждого размера матрицы проводились 1000 раз. На вход подавались случайно сгенерированные матрицы заданного размера. 


\begin{table}[ht]
	\small
	\begin{center}
		\begin{threeparttable}
			\caption{Замер времени для матриц размером от 8 до 256}
			\label{tbl:time}
			\begin{tabular}{|r|r|r|r|r|}
				\hline
				& \multicolumn{4}{c|}{\bfseries Время, мкс} \\ \cline{2-5}
				\bfseries \makecell{Линейный размер, \\ штуки} & \bfseries Классический & \bfseries Виноград & \bfseries Виноград (опт) & \bfseries Штрассен  \\ \cline{2-5}
				\hline
				8 & 6.21 & 6.35 & 6.19 & 6.22 \\
				\hline
				10 & 14.27 & 10.63 & 9.60 & 49.77 \\
				\hline
				16 & 44.47 & 37.21 & 34.00 & 47.09 \\
				\hline
				20 & 92.43 & 78.02 & 66.18 & 364.69 \\
				\hline
				32 & 363.93 & 298.19 & 251.25 & 365.21 \\
				\hline
				40 &  707.57 & 565.64 & 489.92 & 2 820.93 \\
				\hline
				50 & 1 352.12 & 1 109.68 & 927.32 & 2 884.98 \\
				\hline
				64 & 2 867.77 & 2 238.58 & 1 920.84 & 2 877.87 \\
				\hline
				80 & 5 463.36 & 4 403.55 & 3 695.43 & 22 619.08 \\
				\hline
				128 & 27 344.14 & 21 422.10 & 18 168.62 & 25 578.16 \\
				\hline
				256 & 218 888.80 & 167 395.60 & 133 006.20 & 178 033.00 \\
				\hline
			\end{tabular}	
		\end{threeparttable}
	\end{center}
\end{table}

\clearpage

\includesvgimage
{timeMatrixMul} % Имя файла без расширения (файл должен быть расположен в директории inc/img/)
{f} % Обтекание (без обтекания)
{h} % Положение рисунка (см. figure из пакета float)
{1\textwidth} % Ширина рисунка
{Результаты замеров времени работы алгоритмов для  матриц размером от 8 до 256} % Подпись рисунка

Исходя из полученных в таблице \ref{tbl:time} данных можно понять, что наиболее быстрым алгоритмом умножения из всех четырех является оптимизированный алгоритм Винограда: на больших размерах он работает в 1.64 раза быстрее классического алгоритма, в 1.25 раз быстрее своей стандартной версии, в 1.33 раза быстрее алгоритма Штрассена.

Алгоритм Штрассена оказался самым неэффективным по времени среди всех алгоритмов: на размерах матриц, отличных от степени двойки, он проигрывает всем остальным алгоритмам, а на больших размерах, являющихся размерами двойки, данный алгоритм быстрее классического всего в 1.22 раза.
Скачок на графике \ref{img:timeMatrixMul} у алгоритма Штрассена связан с тем, что на размерах, отличных от степени двойки, производится перевыделение памяти и увеличения размеров матриц до ближайшей большей степени двойки. Выиграть по времени у алгоритма Винограда и его оптимизированной версии не получилось ни на одной размерности матриц.

\section{Использование памяти}

\begin{table}[ht]
	\small
	\begin{center}
		\begin{threeparttable}
			\caption{Замер памяти для матриц размером от 10 до 100}
			\label{tbl:mem}
			\begin{tabular}{|r|r|r|r|r|}
				\hline
				& \multicolumn{4}{c|}{\bfseries Память, Кб} \\ \cline{2-5}
				\bfseries \makecell{Линейный размер, \\ штуки} & \bfseries Классический & \bfseries Виноград & \bfseries Виноград (опт) & \bfseries Штрассен  \\ \cline{2-5}
				\hline
				10 & 1.51 & 1.63 & 1.64 & 17.45 \\
				\hline
				20 & 5.26 & 5.46 & 5.47 & 88.61 \\
				\hline
				30 & 11.36 & 11.63 & 11.64 & 106.89 \\
				\hline
				40 & 19.79 & 20.15 & 20.16 & 438.78 \\
				\hline
				50 & 30.57 & 31.01 & 31.06 & 481.90 \\
				\hline
				60 &  43.70 & 44.21 & 44.22 & 534.40 \\
				\hline
				70 & 59.17 & 59.76 & 59.77 & 2 031.60 \\
				\hline
				80 & 76.98 & 77.65 & 77.66 & 2 120.66 \\
				\hline
				90 & 97.13 & 97.88 & 97.89 & 2 221.45 \\
				\hline
				100 & 119.63 & 120.46 & 120.46 & 2 333.95 \\
				\hline
			\end{tabular}	
		\end{threeparttable}
	\end{center}
\end{table}

\clearpage

\includesvgimage
{memMatrixMul} % Имя файла без расширения (файл должен быть расположен в директории inc/img/)
{f} % Обтекание (без обтекания)
{h} % Положение рисунка (см. figure из пакета float)
{1\textwidth} % Ширина рисунка
{Результаты замеров расходов памяти алгоритмов для матриц размером от 10 до 100} % Подпись рисунка

Анализируя таблицу \ref{tbl:mem} можно увидеть, что самым эффективным по памяти является классический алгоритм. Это обусловлено тем, что в этом алгоритме нет дополнительных переменных, которые нужны в других алгоритмах.

Алгоритм Штрассена, как и в случае с оценкой алгоритмов по времени, является самым не эффективным: при размере матриц $10 \times 10$ он расходует памяти в среднем в 11 раз больше, чем любой другой алгоритм. 
Это связано с тем, что при каждом рекурсивном вызове для подматриц выделяется память под их хранение, а также выделяется память для хранения матриц $M_i$, $i = \overline{1, 7}$.

\clearpage

\section*{Вывод}


В данном разделе были проведены замеры времени работы, а также расчеты используемой памяти реализаций алгоритмов умножения матриц. 

Исходя из результатов, полученных при оценках памяти и времени работы алгоритмов, можно сказать, что самым эффективным по времени и по памяти является оптимизированная версия алгоритма Винограда, а самым неэффективным по тем же характеристикам является алгоритм Штрассена.

Применение оптимизаций замены умножения на 2 на побитовый сдвиг влево и замены выражения вида $x = x + k$ на выражение вида $x +\kern-0.25em= k$ в алгоритме Винограда позволили существенно уменьшить время работы алгоритма по сравнению с его неоптимизированной версией, однако расход используемой памяти был незначительно увеличен дополнительной 4-х байтовой целочисленной переменной.
