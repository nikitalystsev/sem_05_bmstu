\chapter{Конструкторский раздел}

\section{Разработка алгоритмов}

\subsection{Алгоритм блочной сортировки}

На рисунке \ref{img:bucketSort} представлена схема алгоритма блочной сортировки.

\includesvgimage
{bucketSort} % Имя файла без расширения (файл должен быть расположен в директории inc/img/)
{f} % Обтекание (без обтекания)
{h} % Положение рисунка (см. figure из пакета float)
{0.5\textwidth} % Ширина рисунка
{Схема алгоритма блочной сортировки} % Подпись рисунка

\clearpage

\subsection{Алгоритм сортировки слиянием}

На рисунке \ref{img:mergeSort} представлена схема алгоритма сортировки слиянием.

На рисунке \ref{img:merge} представлена схема алгоритма функции слияния двух отсортированных подмассивов.

\includesvgimage
{mergeSort} % Имя файла без расширения (файл должен быть расположен в директории inc/img/)
{f} % Обтекание (без обтекания)
{h} % Положение рисунка (см. figure из пакета float)
{1\textwidth} % Ширина рисунка
{Схема алгоритма сортировки слиянием} % Подпись рисунка

\includesvgimage
{merge} % Имя файла без расширения (файл должен быть расположен в директории inc/img/)
{f} % Обтекание (без обтекания)
{h} % Положение рисунка (см. figure из пакета float)
{0.8\textwidth} % Ширина рисунка
{Схема алгоритма функции слияния двух отсортированных подмассивов} % Подпись рисунка

\clearpage

\subsection{Алгоритм поразрядной сортировки}

На рисунке \ref{img:radixSort} представлена схема алгоритма сортировки слиянием.

На рисунке \ref{img:countSort} представлена схема алгоритма функции сортировки по определенному разряду.

\includesvgimage
{radixSort} % Имя файла без расширения (файл должен быть расположен в директории inc/img/)
{f} % Обтекание (без обтекания)
{h} % Положение рисунка (см. figure из пакета float)
{0.7\textwidth} % Ширина рисунка
{Схема алгоритма поразрядной сортировки} % Подпись рисунка

\includesvgimage
{countSort} % Имя файла без расширения (файл должен быть расположен в директории inc/img/)
{f} % Обтекание (без обтекания)
{h} % Положение рисунка (см. figure из пакета float)
{0.7\textwidth} % Ширина рисунка
{Схема алгоритма функции сортировки по определенному разряду} % Подпись рисунка

\clearpage

\section{Оценка трудоемкости алгоритмов}

\subsection{Модель вычислений для проведения оценки трудоемкости алгоритмов}

Была введена модель вычислений для определения трудоемкости каждого отдельного взятого алгоритма сортировки.

\begin{enumerate}[label={\arabic*)}]
	\item Трудоемкость базовых операций имеет:
	\begin{itemize}[label=---]
		\item равную 1:
		\begin{equation}
			\label{for:operations_1}
			\begin{gathered}
				+, -, =, +=, -=, ==, !=, <, >, <=, >=, [], ++, {-}-,\\
				\&\&, >>, <<, ||, \&, |
			\end{gathered}
		\end{equation}
		\item равную 2:
		\begin{equation}
			\label{for:operations_2}
			*, /, \%, *=, /=, \%=
		\end{equation}
	\end{itemize}
	\item Трудоемкость условного оператора:
	\begin{equation}
		\label{for:if}
		f_{if} = f_{\text{условия}} + 
		\begin{cases}
			min(f_1, f_2), & \text{лучший случай}\\
			max(f_1, f_2), & \text{худший случай}
		\end{cases}
	\end{equation}
	\item Трудоемкость цикла:
	\begin{equation}
		\label{for:for}
		\begin{gathered}
			f_{for} = f_{\text{инициализация}} + f_{\text{сравнения}} + M_{\text{итераций}} \cdot (f_{\text{тело}} +\\
			+ f_{\text{инкремент}} + f_{\text{сравнения}})
		\end{gathered}
	\end{equation}
	\item Трудоемкость передачи параметра в функции и возврат из функции равны 0.
\end{enumerate}

\clearpage

\subsection{Трудоемкость алгоритма блочной сортировки}

Алгоритм состоит из четырех последовательно идущих циклов:

\begin{enumerate}[label={\arabic*)}]
	\item поиск минимума и максимума среди всех элементов массива;
	\item распределение элементов массива по соответствующим блокам;
	\item cортировка элементов каждого блока другим алгоритмом сортировки;
	\item cоединение всех блоков воедино.
\end{enumerate}

Для сортировки блоков на шаге $(3)$ была использована функция $std::sort$ \cite{std_sort} из заголовочного файла <<$algorithm$>> библиотеки языка $C++$ со сложностью $O(N \cdot \log_2(N))$.

Для поиска максимального и минимального элемента в массиве на шаге $(1)$ были использованы функции $std::max\_element$ \cite{std_max} и $std::min\_element$ \cite{std_min} из заголовочного файла <<$algorithm$>> библиотеки языка $C++$ cо сложностями, равными $O(N)$.

Трудоемкость инициализации пяти переменных:

\begin{equation}
	\label{eq:bucketPart1}
	\begin{gathered}
		f_1 = 5
	\end{gathered}
\end{equation}

Трудоемкость цикла распределения элементов массива по соответствующим блокам описывается формулой \ref{eq:bucketPart2}.

\begin{equation}
	\label{eq:bucketPart2}
	\begin{gathered}
		% 1 за push_back. Хз, какая у него на самом деле трудоемкость
		f_2 = 1 + 1 + N \cdot (8 + 1 + 1) = 2 + 10 \cdot N = O(N)
	\end{gathered}
\end{equation}

Обозначим число блоков за $K$.
Цикл сортировки каждого блока алгоритмом $std::sort$ в лучшем случае (элементы распределены по блокам равномерно, асимптотика их просмотра $O(K)$, входной массив расположен так, что внутренняя сортировка работает за лучшее время -- $O(N)$) имеет асимптотику:

\begin{equation}
	\label{eq:bucketPart3}
	\begin{gathered}
		f_3 = O(N + K)
	\end{gathered}
\end{equation}

В худшем случае (элементы не имеют математической разницы между собой и внутренняя сортировка работает за худшее время -- $O(N^2)$) асимптотика цикла сортировки каждого блока представлена формулой \ref{eq:bucketPart4}:

\begin{equation}
	\label{eq:bucketPart4}
	\begin{gathered}
		f_3 = O(N^2)
	\end{gathered}
\end{equation}

Так как элементы массива равномерно распределены по $K$ блокам, то трудоемкость цикла соединения всех блоков воедино описывается формулой \ref{eq:bucketPart5}:

\begin{equation}
	\label{eq:bucketPart5}
	\begin{gathered}
		f_4 = 1 + 1 + N \cdot (1 + 3 + \frac{N}{K} \cdot (5 + 1 + 3) + 1 + 1) = \\
		9 \cdot \frac{N^2}{K} + 6 \cdot N + 2
	\end{gathered}
\end{equation}

Итоговая трудоемкость $f_{bucket}$ равна:

В лучшем случае:

\begin{equation}
	\label{eq:bucketFull1}
	\begin{gathered}
		f_{bucket} = f_1 + f_2 + f_3 + f_4 = 5 + 2 + 10 \cdot N + O(N + K) + \\
		9 \cdot \frac{N^2}{K} + 6 \cdot N + 2 =  \\
		9 \cdot \frac{N^2}{K} + 16 \cdot N + 7 + O(N + K) = O(N + K)
	\end{gathered}
\end{equation}

В худшем случае:

\begin{equation}
	\label{eq:bucketFull2}
	\begin{gathered}
		f_{bucket} = f_1 + f_2 + f_3 + f_4 = 5 + 2 + 10 \cdot N + O(N^2) + \\
		9 \cdot \frac{N^2}{K} + 6 \cdot N + 2 =  \\
		9 \cdot \frac{N^2}{K} + 16 \cdot N + 7 + O(N^2) = O(N^2)
	\end{gathered}
\end{equation}

\clearpage

\subsection{Трудоемкость алгоритма сортировки слиянием}

Пусть 

\begin{itemize}[label=---]
	\item $REC$ -- трудоемкость рекурсивного алгоритма;
	\item $DIR$ -- трудоемкость прямого решения;
	\item $DIV$ -- трудоемкость разбиения ввода ($N$) на несколько частей;
	\item $COM$ -- трудоемкость объединения решений.
\end{itemize}

Тогда трудоемкость рекурсивного алгоритма считается по следующей формуле:

\begin{equation}
	\label{eq:rec}
	REC(N) =
	\begin{cases}
		DIR(N), & N \leq N_0\\
		DIV(N) + \displaystyle\sum_{i=1}^{n} REC(F[i]) + COM(N), & N > N_0
	\end{cases}
\end{equation}

где $N$ -- число входных элементов, $N_0$ -- наибольшее число, определяющее тривиальный случай (прямое решение), $n$ -- число рекурсивных вызовов для данного $N$, $F[i]$ -- число входных элементов для данного $i$.

Трудоемкость алгоритма сортировки слиянием определяется следующим образом:

\begin{enumerate}[label={\arabic*)}]
	\item Трудоемкость разбиения ввода ($N$) на части описывается формулой \ref{eq:div}. 
	Каждый следующий вызов берется размерность массива в 2 раза меньше предыдущей путем вычисления индекса срединного элемента массива.
	\begin{equation}
		\label{eq:div}
		\begin{gathered}
			DIV(N) = 1 + 2 + 1 = 4
		\end{gathered}
	\end{equation}
	\item Трудоемкость сортировки левого и правого подмассива (обозначим ее буквой $G = G(N)$) представлена формулами \ref{eq:G} и \ref{eq:Gfinish}:
	\begin{equation}
		\label{eq:G}
		\begin{gathered}
			G(N) = 2 \cdot REC(\frac{N}{2})
		\end{gathered}
	\end{equation}
	
	Число разбиений $K$ массива размером $N$ на подмассивы размером в два раза меньше в алгоритме сортировки слиянием определяется формулой \ref{eq:klog}.
	
	\begin{equation}
		\label{eq:klog}
		\begin{gathered}
			K = \log_2(N)
		\end{gathered}
	\end{equation}

	Поскольку выполняется сортировка массива размером $N$, то 
	
	\begin{equation}
		\begin{gathered}
			REC(\frac{N}{2}) = \frac{N}{2} \cdot \log_2(\frac{N}{2}) = \frac{1}{2} \cdot N \cdot \log_2(N) - \frac{1}{2} \cdot N
		\end{gathered}
	\end{equation}
	
	Таким образом, трудоемкость сортировки левого и правого подмассива определяется формулой \ref{eq:Gfinish}.
	
	\begin{equation}
		\label{eq:Gfinish}
		\begin{gathered}
			G(N) = 2 \cdot (\frac{1}{2} \cdot N \cdot \log_2(N) - \frac{1}{2} \cdot N) = N \cdot \log_2(N) - N
		\end{gathered}
	\end{equation}
	
	\item Трудоемкость объединения решений, а именно слияние двух отсортированных подмассивов
	
	\begin{equation}
		\label{eq:com}
		\begin{gathered}
			COM(N) = 2 + 1 + \frac{N}{2} \cdot (4 + 1 + 1) + \frac{N}{2} \cdot (4 + 1 + 1) + \\
			3 + 1 + \frac{N}{2} \cdot (1 + 4 + 1 + 1) = \frac{19}{2} \cdot N + 7
		\end{gathered}
	\end{equation}

	Таким образом, трудоемкость алгоритма сортировки слиянием согласно формуле \ref{eq:rec} определяется так:
	
	\begin{equation}
		\label{eq:merge}
		\begin{gathered}
			f_{merge} = DIV(N) + G(N) + COM(N) = \\
			4 + N \cdot \log_2(N) - N + \frac{19}{2} \cdot N + 7 = \\
			N \cdot \log_2(N) + \frac{17}{2} \cdot N + 11 = O(N \cdot \log_2(N))
		\end{gathered}
	\end{equation}

\end{enumerate}

\clearpage

\subsection{Трудоемкость алгоритма поразрядной сортировки}

Трудоемкость алгоритма поразрядной сортировки состоит из цикла по всем разрядам наибольшего числа в массиве и сортировки массива по каждому из разрядов.

Асимптотика трудоемкости поиска наибольшего элемента массива равна:

\begin{equation}
	\label{eq:radixPart1}
	\begin{gathered}
		f_1 = O(N)
	\end{gathered}
\end{equation}

Пусть число разрядов наибольшего числа равно $K$. 
Тогда трудоемкость цикла сортировки по всем разрядам наибольшего числа равна:

\begin{equation}
	\label{eq:radixPart2}
	\begin{gathered}
		f_2 = 1 + 2 + K \cdot (f_{count} + 1 + 2)
	\end{gathered}
\end{equation}


где $f_{count}$ -- трудоемкость сортировки по одному разряду.

Трудоемкость сортировки по одному разряду представлена формулой \ref{eq:radixPart3}.

\begin{equation}
	\label{eq:radixPart3}
	\begin{gathered}
		f_{count} = 2 + 9 \cdot N + 47 + 2 + 18 \cdot N + 2 + 5 \cdot N = \\
		32 \cdot N + 53
	\end{gathered}
\end{equation}

Итоговая трудоемкость поразрядной сортировки в лучшем и худшем случае описывается формулой \ref{eq:radixPart4}.

\begin{equation}
	\label{eq:radixPart4}
	\begin{gathered}
		f_{radix} = 3 + K \cdot (32 \cdot N + 53 + 3) = \\
		32 \cdot K \cdot N + 56 \cdot K + 3 = O(K \cdot N)
	\end{gathered}
\end{equation}

\section*{Вывод}

В данном разделе были построены схемы алгоритмов блочной сортировки, сортировки слиянием и поразрядной сортировки, а также были выполнены теоретические расчеты трудоемкости этих алгоритмов.

Согласно расчетам трудоемкости, на равномерно распределенных данных самым эффективным алгоритмом сортировки будет алгоритм блочной сортировки со сложностью $O(N + K)$. 
Для сортировки же произвольных данных из всех трех алгоритмов лучше всего подошел бы алгоритм сортировки слиянием со сложностью $O(N \cdot \log_2(N))$.  


