\chapter{Теория проектирования реляционных баз данных: функциональные зависимости, нормальные формы}

Классический подход к проектированию реляционных баз данных заключается в том, что сначала предметная область представляется в виде одного или нескольких отношений, а далее осуществляется процесс \textit{нормализации} схем отношений, причем каждая следующая нормальная форма обладает свойствами лучшими, чем предыдущая.

Каждой нормальной форме соответствует некоторый определенный набор ограничений, и отношение находится в некоторой нормальной форме, если удовлетворяет свойственному ей набору ограничений. 
Примером набора ограничений является ограничение первой нормальной формы – значения всех атрибутов отношения атомарны. 
Поскольку требование первой нормальной формы является базовым требованием классической реляционной модели данных, будем считать, что исходный набор отношений уже соответствует этому требованию.

В теории реляционных баз данных обычно выделяется следующая последовательность нормальных форм:

\begin{itemize}[label*=--]
	\item первая нормальная форма (1НФ или 1NF);
	\item вторая нормальная форма (2НФ или 2NF);
	\item третья нормальная форма (3НФ или 3NF);
	\item нормальная форма Бойса-Кодда (НФБК или BCNF);
	\item четвертая нормальная форма (4НФ или 4NF);
	\item пятая нормальная форма, или нормальная форма проекции-соединения (5НФ или 5NF или PJ/NF).
\end{itemize}

\clearpage

Основные свойства нормальных форм такие:

\begin{itemize}[label*=--]
	\item каждая следующая нормальная форма в некотором смысле лучше предыдущей;
	\item при переходе к следующей нормальной форме свойства предыдущих нормальных свойств сохраняются.
\end{itemize}

Процесс проектирования реляционной базы данных на основе метода нормализации преследует две основные цели:

\begin{itemize}[label*=--]
	\item избежать избыточности хранения данных;
	\item устранить аномалии обновления отношений.
\end{itemize}


В основе метода нормализации лежит \textit{декомпозиция} отношения, находящегося в предыдущей нормальной форме, в два или более отношения, удовлетворяющих требованиям следующей нормальной формы. 
Считаются правильными такие декомпозиции отношения, которые обратимы, т.~е. имеется возможность собрать исходное отношение из декомпозированных отношений без потери информации.

Наиболее важные на практике нормальные формы отношений основываются на фундаментальном в теории реляционных баз данных понятии функциональной зависимости.


\section{Функциональные зависимости}

\textbf{Определение 1.} Пусть R --- это отношение, а Х и Y --- произвольные подмножества множества атрибутов отношения R. Тогда Y \textbf{функционально зависимо} от Х, что в символическом виде записывается как X -> Y $\Leftrightarrow \forall$ значение множества Х связано в точности с одним значением множества Y.

Левая и правая стороны ФЗ будут называются \textit{\bfseries детерминантом} и \textit{\bfseries зависимой частью} соответственно.

\textbf{Определение 2.} Пусть R является переменной-отношением, а Х и Y --- произвольными подмножествами множества атрибутов переменной-отношения R. Тогда Х -> Y $\Leftrightarrow \forall$ \textit{\bfseries допустимого значения отношения} R $\forall$ значение Х связано в точности с одним значением Y.

\textbf{Определение 2a.} Пусть R(A1, A2, $\ldots$, An) --- схема отношения. Функциональная зависимость, обозначаемая X -> Y между двумя наборами атрибутов X и Y, которые являются подмножествами R определяет ограничение на возможность существования кортежа в некотором отношении r. Ограничение означает, что для любых двух кортежей $t_1$ и $t_2$ в r, для которых имеет место $t_1$[X] = $t_2$[X], также имеет место $t_1$[Y] = $t_2$[Y].


\begin{itemize}[label*=--]
	\item Если ограничение на схеме отношения R утверждает, что не может быть более одного кортежа со значением атрибутов X в любом отношении экземпляре отношения r, то X является потенциальным ключом R. 
	Это означает, что X -> Y для любого подмножества атрибутов Y из R. 
	Если X является потенциальным ключ R, то X -> R.
	\item Если X -> Y в R, это не означает, что Y -> X в R.
\end{itemize}

\textbf{Определение 3.} ФЗ (X -> Y) тривиальная $\Leftrightarrow$ Y $\subseteq$ X.

\textbf{Определение 4.} Множество всех ФЗ, которые задаются данным множеством ФЗ S, называется замыканием S и обозначается символом S+.

Пусть в перечисленных ниже правилах А, В и C – произвольные подмножества множества атрибутов заданной переменной-отношения R, а символическая запись АВ означает \{ A, B \}. 
Тогда правила вывода определяются следующим образом:

\begin{enumerate}[label={\arabic*)}]
	\item  Правило \textit{\bfseries рефлексивности}: (B $\subseteq$ A) $\Rightarrow$ (А -> В).
	\item  Правило \textit{\bfseries дополнения}: (A -> B) $\Rightarrow$ АС -> BC.
	\item  Правило \textit{\bfseries транзитивности}: (А -> В) и (В -> С) $\Rightarrow$ (А -> C).
	\item  Правило \textit{\bfseries самоопределения}: А -> А.
	\item  Правило \textit{\bfseries декомпозиции}: (A -> BC) $\Rightarrow$ (А -> В) и (А -> С).
	\item  Правило \textit{\bfseries объединения}: (А -> В) и (А -> С) $\Rightarrow$ (А -> BC). 
	
\clearpage

	\item  Правило \textit{\bfseries композиции}: (А -> В) и (С -> D) $\Rightarrow$ (АС -> BD).
	\item  \textit{\bfseries Общая теорема объединения}: (А -> В) и(С -> D) $\Rightarrow$ (А(С – B) -> BD).
\end{enumerate}


\textbf{Определение 5.} \textit{Суперключ} переменной-отношения R --- это множество атрибутов переменной-отношения R, которое содержит в виде подмножества (но не обязательно собственного подмножества), по крайней мере, один потенциальный ключ.

\textbf{Определение 6.} Два множества ФЗ S1 и S2 \textit{\bfseries эквивалентны} $\Leftrightarrow$ они \textit{\bfseries являются покрытиями} друг для друга, т. е. S1+ = S2+.

Каждое множество ФЗ эквивалентно, по крайней мере, одному \textit{\bfseries неприводимому} множеству.

\textbf{Определение 7.} Множество ФЗ является \textit{\bfseries неприводимым}  $\Leftrightarrow$ оно обладает всеми перечисленными ниже свойствами.

\begin{itemize}[label*=--]
	\item Каждая ФЗ этого множества имеет одноэлементную правую часть.
	\item Ни одна ФЗ множества не может быть устранена без изменения замыкания этого множества.
	\item Ни один атрибут не может быть устранен из левой части любой ФЗ данного множества без изменения замыкания множества.
\end{itemize}

Если I является неприводимым множеством, которое эквивалентно множеству S, то проверка выполнения ФЗ из множества I автоматически обеспечит выполнение ФЗ из множества S.

\section{Нормальные формы, основанные на функциональных зависимостях}

\textbf{Определение 1.} Переменная отношения находится в первой нормальной форме (1НФ) тогда и только тогда, когда в любом допустимом значении отношения каждый его кортеж содержит только одно значение для каждого из атрибутов.

\textbf{Определение 2.} Функциональная зависимость R.X -> R.Y называется полной, если атрибут Y не зависит функционально от любого точного подмножества X.

\textbf{Определение 3.} Функциональная зависимость R.X -> R.Y называется транзитивной, если существует такой атрибут Z, что имеются функциональные зависимости R.X -> R.Z и R.Z -> R.Y и отсутствует функциональная зависимость R.Z -> R.X. 
(При отсутствии последнего требования мы имели бы <<неинтересные>> транзитивные зависимости в любом отношении, обладающем несколькими ключами.)

\textbf{Определение 4.} Неключевым атрибутом называется любой атрибут отношения, не входящий в состав потенциального ключа (в частности, первичного).

\textbf{Определение 5.} Два или более атрибута взаимно независимы, если ни один из этих атрибутов не является функционально зависимым от других.

\subsection*{Вторая нормальная форма}

\textbf{Определение 6.} (В этом определении предполагается, что единственным ключом отношения является первичный ключ.) Отношение R находится во второй нормальной форме (2НФ) в том и только в том случае, когда оно находится в 1НФ, и каждый неключевой атрибут полностью зависит от первичного ключа.

Если допустить наличие нескольких ключей, то определение 6 примет следующий вид: 

\textbf{Определение 6a.} Отношение R находится во второй нормальной форме (2NF) в том и только в том случае, когда оно находится в 1НФ, и каждый неключевой атрибут полностью зависит от каждого ключа R.

\subsection*{Третья нормальная форма}

\textbf{Определение 7.} (Снова определение дается в предположении существования единственного ключа.) Отношение R находится в третьей нормальной форме (3НФ) в том и только в том случае, если оно находится в 2НФ и каждый неключевой атрибут нетранзитивно зависит от первичного ключа.

Если отказаться от того ограничения, что отношение обладает единственным ключом, то определение 3NF примет следующую форму: 

\textbf{Определение 7a.} Отношение R находится в третьей нормальной форме (3НФ) в том и только в том случае, если оно находится в 2НФ, и каждый неключевой атрибут не является транзитивно зависимым от какого-либо ключа R.

\subsection*{Нормальная форма Бойса-Кодда}

Определение 3НФ неадекватно при выполнении следующих условий:

\begin{enumerate}[label={\arabic*)}]
	\item  переменная-отношение имеет два (или более) потенциальных ключа;
	\item  эти потенциальные ключи являются составными;
	\item  два или более потенциальных ключей перекрываются (т. е. имеют по крайней мере один общий атрибут).
\end{enumerate}

Поэтому впоследствии исходное определение 3НФ было заменено более строгим определением нормальной формы Бойса-Кодда (НФБК).

\textbf{Определение 8.} Детерминант - любой атрибут (или группа атрибутов), от которого полностью функционально зависит некоторый другой атрибут.

\textbf{Определение 9.} Отношение R находится в нормальной форме Бойса-Кодда (НФБК) в том и только в том случае, если каждый детерминант является потенциальным ключом.

\subsection*{Многозначные зависимости и четвертая нормальная форма}

\textbf{Формальное определение МЗ.} 
Пусть A, B и C являются произвольными подмножествами множества атрибутов переменной-отношения R. 
Тогда подмножество B \textbf{многозначно зависит} от подмножества A, что символически выражается записью A -> > B $\Leftrightarrow$ множество значений B, соответствующее заданной паре (значение A, значение С) переменной-отношения R, зависит от A, но не зависит от С.

\textbf{Определение 4НФ.}  Переменная-отношение R находится в четвертой нормальной форме (4НФ) тогда и только тогда, когда в случае существования таких подмножеств А и B атрибутов этой переменной-отношения R, для которых выполняется нетривиальная МЗ А -> > B, все атрибуты переменной-отношения R также функционально зависят от атрибута А.

\subsection*{Зависимости соединения и пятая нормальная форма}

\textbf{Определение *зависимости соединения.} 
Пусть R является переменной-отношением, а А, В, $\ldots$, Z - произвольными под- множествами множества ее атрибутов. Переменная-отношение R удовлетворяет зависимости соединения *{ А, В, $\ldots$, Z} (читается <<звездочка А, В, $\ldots$, Z>>) тогда и только тогда, когда любое допустимое значение переменной-отношения R эквивалентно соединению ее проекций по подмножествам атрибутов А, В, $\ldots$, Z.

\textbf{Определение 5НФ.} Переменная-отношение R находится в пятой нормальной форме (5НФ), которую иногда иначе называют проекционно-соединительной нормальной формой (ПСНФ), тогда и только тогда, когда каждая нетривиальная ЗС в переменной-отношении R подразумевается ее потенциальными ключами.