\chapter*{ЗАКЛЮЧЕНИЕ}
\addcontentsline{toc}{chapter}{ЗАКЛЮЧЕНИЕ}

В ходе выполнения лабораторной работы были решены следующие задачи:

\begin{enumerate}[label={\arabic*)}]
	\item описаны три алгоритма сортировки: блочной, слиянием и поразрядной;
	\item создано программное обеспечение, реализующее следующие алгоритмы:
	\begin{itemize}[label=--]
		\item алгоритм блочной сортировки;
		\item алгоритм сортировки слиянием;
		\item алгоритм поразрядной сортировки;
	\end{itemize}
	
	\item проведен анализ эффективности реализаций алгоритмов по памяти и по времени;
	\item проведена оценка трудоемкости алгоритмов сортировки;
	\item подготовлен отчет по лабораторной работе;
\end{enumerate}

Цель данной лабораторной работы, а именно исследование трех алгоритмов сортировки: блочной сортировки, сортировки слиянием и поразрядной сортировки, также была достигнута.

Согласно теоретическим расчетам трудоемкости алгоритмов сортировки наименее трудоемким на равномерно распределенных данных оказался алгоритм блочной сортировки, наиболее трудоемким -- алгоритм поразрядной сортировки.

Для равномерно распределенных данных лучше всего использовать алгоритмы блочной и поразрядной сортировки, так как среднее отклонение во времени работы порядка 5 мкс, и разница в расходе используемой памяти не превышает 52 байт.

Для данных, упорядоченных по возрастанию лучше всего использовать алгоритм блочной сортировки, поскольку этот алгоритм показал наименьшее время работы и наименьший расход используемой памяти среди всех трех алгоритмов.

Для данных, упорядоченных по убыванию лучше всего использовать алгоритмы блочной и поразрядной сортировки, так как среднее отклонение во времени работы порядка 5 мкс, и разница в расходе используемой памяти не превышает 52 байт.

Для массива данных, состоящего из одинаковых элементов, лучше всего использовать алгоритм блочной сортировки, поскольку этот алгоритм показал наименьшее время работы (в $1.5$ раза быстрее алгоритма сортировки слиянием и в $1.26$ раза быстрее алгоритма поразрядной сортировки) и наименьший расход используемой памяти среди всех трех алгоритмов.

Результаты проведенного исследования практически подтвердили теоретические расчеты трудоемкости: наиболее эффективным по времени работы и по используемой памяти является алгоритм блочной сортировки, но наиболее трудоемким оказался алгоритм сортировки слиянием.

