\chapter{Технологический раздел}

В данном разделе будут перечислены требования к программному обеспечению, средства реализации, листинги кода и функциональные тесты.

\section{Требования к программному обеспечению}

К программе предъявляется ряд требований:

\begin{itemize} [label=--]
	\item на вход подаётся вектор элементов;
	\item все элементы вектора - целые неотрицательные числа (это необходимо для возможности сравнения сортировок между собой);
	\item на выходе в том же векторе находятся отсортированные по возрастанию элементы исходного.
\end{itemize}

\section{Средства реализации}

В качестве языка программирования для этой лабораторной работы был выбран $C++$ \cite{pl} по следующим причинам:

\begin{itemize}[label=--]
	\item в $C++$ есть встроенный модуль $ctime$, предоставляющий необходимый функционал для замеров процессорного времени;
	\item в стандартной библиотеке $C++$ есть оператор $sizeof$, позволяющий получить размер переданного объекта в байтах. Следовательно, $C++$ предоставляет возможности для проведения точных оценок по используемой памяти.
\end{itemize}

В качестве функции, которая будет осуществлять замеры процессорного времени, будет использована функция $clock\_gettime$ из встроенного модуля $ctime$ \cite{cpu_time_func}.

\section{Сведения о модулях программы}

Программа состоит из шести модулей: 

\begin{enumerate}[label={\arabic*)}]
	\item \texttt{algorithms.cpp} --- модуль, хранящий реализации алгоритмов сортировки;
	\item \texttt{processTime.cpp} --- модуль, содержащий функцию для замера процессорного времени;
	\item \texttt{memoryMeasurements.cpp} --- модуль, содержащий функции, позволяющие провести сравнительный анализ использования памяти в реализациях алгоритмов сортировки;
	\item \texttt{timeMeasurements.cpp} --- модуль, содержащий функции, позволяющие провести сравнительный анализ использования времени в реализациях алгоритмов сортировки;
	\item \texttt{main.cpp} --- файл, содержащий точку входа в программу;
	\item \texttt{task7} --- модуль, содержащий набор скриптов для проведения замеров программы по времени и памяти и построения графиков по полученным данным.
\end{enumerate}

\clearpage

\section{Реализации алгоритмов}

В листингах \ref{lst:bucketSort.txt} -- \ref{lst:radixSort.txt} представлены реализации трех алгоритмов сортировки: блочной, сортировки слиянием и поразрядной.

\includelisting
{bucketSort.txt} % Имя файла с расширением (файл должен быть расположен в директории inc/lst/)
{Реализация алгоритма блочной сортировки} % Подпись листинга

\clearpage

\includelisting
{mergeSortPart1.txt} % Имя файла с расширением (файл должен быть расположен в директории inc/lst/)
{Реализация алгоритма сортировки слиянием (начало)} % Подпись листинга

\clearpage

\includelisting
{mergeSortPart2.txt} % Имя файла с расширением (файл должен быть расположен в директории inc/lst/)
{Реализация алгоритма сортировки слиянием (конец)} % Подпись листинга

\clearpage

\includelisting
{radixSort.txt} % Имя файла с расширением (файл должен быть расположен в директории inc/lst/)
{Реализация алгоритма поразрядной сортировки (конец)} % Подпись листинга

%\includelisting
%{classicMatrixMul.txt} % Имя файла с расширением (файл должен быть расположен в директории inc/lst/)
%{Реализация классического алгоритма умножения двух матриц} % Подпись листинга


\clearpage

\section{Функциональные тесты}

В таблице \ref{tbl:func_tests} приведены тестовые данные, на которых было протестированно разработанное ПО. Все тесты были успешно пройдены.

\begin{table}[ht]
	\begin{center}
		\begin{threeparttable}
			\caption{Функциональные тесты}
			\label{tbl:func_tests}
			\begin{tabular}{|c|c|c|c|}
				\hline
				\bfseries Массив & \bfseries Блочная & \bfseries Слиянием & \bfseries Поразрядная \\
				\hline
				1 2 3 4 5 6 & 1 2 3 4 5 6 & 1 2 3 4 5 6 & 1 2 3 4 5 6 \\
				\hline
				6 5 4 3 2 1 & 1 2 3 4 5 6 & 1 2 3 4 5 6 & 1 2 3 4 5 6 \\
				\hline
				41 56 67 10 34 2 & 2 10 34 41 56 67  & 2 10 34 41 56 67  & 2 10 34 41 56 67  \\
				\hline
				54 33 0 55 33 7 14 & 0 7 14 33 33 54 55  & 0 7 14 33 33 54 55  & 0 7 14 33 33 54 55  \\
				\hline
				4 4 4 4 4 4 & 4 4 4 4 4 4  & 4 4 4 4 4 4  & 4 4 4 4 4 4  \\
				\hline
				10 & 10  & 10  & 10  \\
				\hline
				\{\} & \makecell{Сообщение  \\ об ошибке}  & \makecell{Сообщение  \\ об ошибке}  & \makecell{Сообщение  \\ об ошибке}  \\
				\hline
			\end{tabular}
		\end{threeparttable}
	\end{center}
\end{table}


\section*{Вывод}

В данном разделе были реализованы и протестированы 3 алгоритма сортировки:
алгоритм блочной сортировки, алгоритм сортировки слиянием и алгоритм поразрядной сортировки.

    