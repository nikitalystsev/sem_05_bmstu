\documentclass{bmstu}

\usepackage{threeparttable}

\usepackage{svg}
\svgpath{./inc/img/}

% Настройка полей
\usepackage{geometry}
\geometry{left=30mm}
\geometry{right=10mm}
\geometry{top=20mm}
\geometry{bottom=20mm}

% Настройка заголовков
\usepackage{titlesec}
\makeatletter
\renewcommand\LARGE{\@setfontsize\LARGE{22pt}{20}}
\renewcommand\Large{\@setfontsize\Large{20pt}{20}}
\renewcommand\large{\@setfontsize\large{16pt}{20}}
\makeatother
\titleformat{\chapter}{\large\bfseries}{\thechapter}{0.5em}{\large\bfseries}
\titleformat{name=\chapter,numberless}[block]{\hspace{\parindent}}{}{0pt}{\large\bfseries\centering}
\titleformat{\section}{\large\bfseries}{\thesection}{0.5em}{\large\bfseries}
\titleformat{\subsection}{\large\bfseries}{\thesubsection}{0.5em}{\large\bfseries}
\titleformat{\subsubsection}{\large\bfseries}{\thesubsection}{0.5em}{\large\bfseries}
\titlespacing{\chapter}{12.5mm}{-22pt}{10pt}
\titlespacing{\section}{12.5mm}{10pt}{10pt}
\titlespacing{\subsection}{12.5mm}{10pt}{10pt}
\titlespacing{\subsubsection}{12.5mm}{10pt}{10pt}

%21.10 - 11.11 22:00
%
%Крайне рекомендуется сдать (скинуть) пдф два раза - в первый раз черновик, во второй раз прочитанный спустя пару дней и отредактированный черновик. Вы сами удивитесь, каким Вам покажется текст спустя некоторое время.
%
%Вёрстка по ширине с переносами, 14 кегль. Допускается и рекомендуется вместо картинок поместить плейсхолдеры, а вместо выносных формул - текстовые плейсхолдеры с указанием, какие величины в формуле описаны. К моменту чистовой работы над текстом половина картинок и формул может быть уже удалена, не тратьте время на тщательное оформление черновой работы.
%
%Источники и правильные ссылки на источники обязательны.
%
%Ссылки на картинки и формулы обязательны.
%
%Отдельной последней страницей иметь список использованных источников.

\bibliography{biblio}

\begin{document}

%\makeresearchtitle
%{Информатика и системы управления} % Название факультета
%{Программное обеспечение ЭВМ и информационные технологии} % Название кафедры
%{Методы статистического машинного перевода} % Тема работы
%{ИУ7-53Б} % Номер группы
%{Лысцев~Н.~Д.} % ФИО студента
%{Кострицкий~А.~С.} % ФИО научного руководителя
%{}
%{}

%\maketableofcontents

%\include{01-definitions}
%\include{02-abbreviations}
	
\chapter*{ВВЕДЕНИЕ}
\addcontentsline{toc}{chapter}{ВВЕДЕНИЕ}

Сортировка -- процесс перестановки предметов, при котором они располагаются в порядке возрастания или убывания \cite{whatIsSort}.

Целью данной лабораторной работы является исследование трех алгоритмов сортировки: блочной сортировки, сортировки слиянием и поразрядной сортировки.

Для достижения поставленной цели необходимо решить следующие задачи:

\begin{enumerate}[label={\arabic*)}]
	\item описать три алгоритма сортировки: блочной, слиянием и поразрядной;
	\item создать программное обеспечение, реализующее следующие алгоритмы:
	\begin{itemize}[label=--]
		\item алгоритм блочной сортировки;
		\item алгоритм сортировки слиянием;
		\item алгоритм поразрядной сортировки;
	\end{itemize}
	\item провести анализ эффективности реализаций алгоритмов по памяти и по времени;
	\item провести оценку трудоемкости алгоритмов сортировки;
	\item обосновать полученные результаты в отчете к выполненной лабораторной работе.
\end{enumerate}

%\chapter{Комплект №1. Случайные события}

\includeimage
{1.1} % Имя файла без расширения (файл должен быть расположен в директории inc/img/)
{f} % Обтекание (без обтекания)
{h} % Положение рисунка (см. figure из пакета float)
{1\textwidth} % Ширина рисунка
{Решение задачи 1.1} % Подпись рисунка

\includeimage
{1.2} % Имя файла без расширения (файл должен быть расположен в директории inc/img/)
{f} % Обтекание (без обтекания)
{h} % Положение рисунка (см. figure из пакета float)
{1\textwidth} % Ширина рисунка
{Решение задачи 1.2} % Подпись рисунка

\includeimage
{1.3} % Имя файла без расширения (файл должен быть расположен в директории inc/img/)
{f} % Обтекание (без обтекания)
{h} % Положение рисунка (см. figure из пакета float)
{1\textwidth} % Ширина рисунка
{Решение задачи 1.3} % Подпись рисунка

\includeimage
{1.4} % Имя файла без расширения (файл должен быть расположен в директории inc/img/)
{f} % Обтекание (без обтекания)
{h} % Положение рисунка (см. figure из пакета float)
{1\textwidth} % Ширина рисунка
{Решение задачи 1.4} % Подпись рисунка

\includeimage
{1.5} % Имя файла без расширения (файл должен быть расположен в директории inc/img/)
{f} % Обтекание (без обтекания)
{h} % Положение рисунка (см. figure из пакета float)
{1\textwidth} % Ширина рисунка
{Решение задачи 1.5} % Подпись рисунка

\includeimage
{1.6} % Имя файла без расширения (файл должен быть расположен в директории inc/img/)
{f} % Обтекание (без обтекания)
{h} % Положение рисунка (см. figure из пакета float)
{1\textwidth} % Ширина рисунка
{Решение задачи 1.6} % Подпись рисунка

\includeimage
{1.7} % Имя файла без расширения (файл должен быть расположен в директории inc/img/)
{f} % Обтекание (без обтекания)
{h} % Положение рисунка (см. figure из пакета float)
{1\textwidth} % Ширина рисунка
{Решение задачи 1.7} % Подпись рисунка

\includeimage
{1.8} % Имя файла без расширения (файл должен быть расположен в директории inc/img/)
{f} % Обтекание (без обтекания)
{h} % Положение рисунка (см. figure из пакета float)
{1\textwidth} % Ширина рисунка
{Решение задачи 1.8} % Подпись рисунка

\includeimage
{1.9} % Имя файла без расширения (файл должен быть расположен в директории inc/img/)
{f} % Обтекание (без обтекания)
{h} % Положение рисунка (см. figure из пакета float)
{1\textwidth} % Ширина рисунка
{Решение задачи 1.9} % Подпись рисунка

\includeimage
{1.10} % Имя файла без расширения (файл должен быть расположен в директории inc/img/)
{f} % Обтекание (без обтекания)
{h} % Положение рисунка (см. figure из пакета float)
{1\textwidth} % Ширина рисунка
{Решение задачи 1.10} % Подпись рисунка

\includeimage
{1.11} % Имя файла без расширения (файл должен быть расположен в директории inc/img/)
{f} % Обтекание (без обтекания)
{h} % Положение рисунка (см. figure из пакета float)
{1\textwidth} % Ширина рисунка
{Решение задачи 1.11} % Подпись рисунка

\includeimage
{1.13} % Имя файла без расширения (файл должен быть расположен в директории inc/img/)
{f} % Обтекание (без обтекания)
{h} % Положение рисунка (см. figure из пакета float)
{1\textwidth} % Ширина рисунка
{Решение задачи 1.13} % Подпись рисунка

\includeimage
{1.14} % Имя файла без расширения (файл должен быть расположен в директории inc/img/)
{f} % Обтекание (без обтекания)
{h} % Положение рисунка (см. figure из пакета float)
{1\textwidth} % Ширина рисунка
{Решение задачи 1.14} % Подпись рисунка

\includeimage
{1.15} % Имя файла без расширения (файл должен быть расположен в директории inc/img/)
{f} % Обтекание (без обтекания)
{h} % Положение рисунка (см. figure из пакета float)
{1\textwidth} % Ширина рисунка
{Решение задачи 1.15} % Подпись рисунка

\includeimage
{1.16} % Имя файла без расширения (файл должен быть расположен в директории inc/img/)
{f} % Обтекание (без обтекания)
{h} % Положение рисунка (см. figure из пакета float)
{1\textwidth} % Ширина рисунка
{Решение задачи 1.16} % Подпись рисунка

\includeimage
{1.17} % Имя файла без расширения (файл должен быть расположен в директории inc/img/)
{f} % Обтекание (без обтекания)
{h} % Положение рисунка (см. figure из пакета float)
{1\textwidth} % Ширина рисунка
{Решение задачи 1.17} % Подпись рисунка
%\chapter{Конструкторский раздел}

В данном разделе будут разработаны алгоритмы классического умножения матриц, умножения матриц с использованием алгоритма Винограда и его оптимизированной версии, а так же алгоритма Штрассена для умножения матриц и его оптимизированной версии и приведены схемы алгоритмов их реализации. Также будет приведена оценка трудоемкости данных алгоритмов.

\section{Разработка алгоритма классического умножения матриц}

На рисунке \ref{img:classicMul} приведена схема алгоритма классического умножения двух матриц.

\includesvgimage
{classicMul} % Имя файла без расширения (файл должен быть расположен в директории inc/img/)
{f} % Обтекание (без обтекания)
{h} % Положение рисунка (см. figure из пакета float)
{0.5\textwidth} % Ширина рисунка
{Схема алгоритма классического умножения двух матриц} % Подпись рисунка
 
\clearpage

\section{Разработка алгоритма Винограда для умножения двух матриц}

На рисунке \ref{img:vinogradMul} приведена схема алгоритма Винограда для умножения двух матриц.

\includesvgimage
{vinogradMul} % Имя файла без расширения (файл должен быть расположен в директории inc/img/)
{f} % Обтекание (без обтекания)
{h} % Положение рисунка (см. figure из пакета float)
{0.46\textwidth} % Ширина рисунка
{Схема алгоритма Винограда для умножения двух матриц} % Подпись рисунка

\clearpage

На рисунке \ref{img:vinogradMulCalcRowFactor} приведена схема подпрограммы вычисления cумм произведений пар соседних элементов строк матрицы.

\includesvgimage
{vinogradMulCalcRowFactor} % Имя файла без расширения (файл должен быть расположен в директории inc/img/)
{f} % Обтекание (без обтекания)
{h} % Положение рисунка (см. figure из пакета float)
{0.8\textwidth} % Ширина рисунка
{Схема подпрограммы вычисления cумм произведений пар соседних элементов строк матрицы} % Подпись рисунка

\clearpage

На рисунке \ref{img:vinogradMulCalcColFactor} приведена схема подпрограммы вычисления cумм произведений пар соседних элементов строк матрицы.

\includesvgimage
{vinogradMulCalcColFactor} % Имя файла без расширения (файл должен быть расположен в директории inc/img/)
{f} % Обтекание (без обтекания)
{h} % Положение рисунка (см. figure из пакета float)
{0.8\textwidth} % Ширина рисунка
{Схема подпрограммы вычисления cумм произведений пар соседних элементов столбцов матрицы} % Подпись рисунка

\clearpage

\section{Разработка алгоритма Штрассена для умножения двух матриц}

На рисунке \ref{img:strassenMul} приведена схема алгоритма Штрассена для умножения двух матриц.

\includesvgimage
{strassenMul} % Имя файла без расширения (файл должен быть расположен в директории inc/img/)
{f} % Обтекание (без обтекания)
{h} % Положение рисунка (см. figure из пакета float)
{0.6\textwidth} % Ширина рисунка
{Схема алгоритма Штрассена умножения двух матриц} % Подпись рисунка

\clearpage

\section{Оценка трудоемкости алгоритмов}

\subsection{Модель вычислений для проведения оценки трудоемкости алгоритмов}

Была введена модель вычислений для определения трудоемкости каждого отдельного взятого алгоритма сортировки.

\begin{enumerate}[label={\arabic*)}]
	\item Трудоемкость базовых операций имеет:
	\begin{itemize}[label=---]
		\item равную 1:
		\begin{equation}
			\label{for:operations_1}
			\begin{gathered}
				+, -, =, +=, -=, ==, !=, <, >, <=, >=, [], ++, {-}-,\\
				\&\&, >>, <<, ||, \&, |
			\end{gathered}
		\end{equation}
		\item равную 2:
		\begin{equation}
			\label{for:operations_2}
			*, /, \%, *=, /=, \%=
		\end{equation}
	\end{itemize}
	\item Трудоемкость условного оператора:
	\begin{equation}
		\label{for:if}
		f_{if} = f_{\text{условия}} + 
		\begin{cases}
			min(f_1, f_2), & \text{лучший случай}\\
			max(f_1, f_2), & \text{худший случай}
		\end{cases}
	\end{equation}
	\item Трудоемкость цикла:
	\begin{equation}
		\label{for:for}
		\begin{gathered}
			f_{for} = f_{\text{инициализация}} + f_{\text{сравнения}} + M_{\text{итераций}} \cdot (f_{\text{тело}} +\\
			+ f_{\text{инкремент}} + f_{\text{сравнения}})
		\end{gathered}
	\end{equation}
	\item Трудоемкость передачи параметра в функции и возврат из функции равны 0.
\end{enumerate}

\clearpage

\subsection{Трудоемкость классического алгоритма умножения двух матриц}

Для стандартного алгоритма умножения матриц трудоемкость будет слагаться из:

\begin{itemize}[label=---]
	\item внешнего цикла по $i \in [1 \ldots N]$ , трудоёмкость которого: $f = 2 + N \cdot (2 + f_{body})$;
	\item цикла по $j \in [1 \ldots P]$ , трудоёмкость которого: $f = 2 + 2 + P \cdot (2 + f_{body})$;
	\item цикла по $k \in [1 \ldots M]$ , трудоёмкость которого: $f = 2 + 2 + 14M$;
\end{itemize}

Поскольку трудоемкость стандартного алгоритма равна трудоемкости внешнего цикла, то:
\begin{equation}
	\label{eq:classic}
	\begin{gathered}
		f_{standart} = 2 + N \cdot (2 + 2 + P \cdot (2 + 2 + M \cdot (2 + 8 + 1 + 1 + 2)))= \\
		= 2 + 4N + 4NP + 14NMP \approx 14NMP = O(N^3)
	\end{gathered}
\end{equation}

\subsection{Трудоемкость алгоритма Винограда для умножения двух матриц}

При вычислении трудоемкости алгоритма Винограда учитывается следующее:

\begin{itemize}[label=---]
	\item создание и инициализация массивов $rowFactor$ и $colFactor$, трудоёмкость которых указана в формуле~(\ref{eq:v_init});
	\begin{equation}
		\label{eq:v_init}
		f_{init} = N + M
	\end{equation}
	\item заполнение массива $rowFactor$, трудоёмкость которого указана в формуле~(\ref{eq:v_rowF});
	\begin{equation}
		\label{eq:v_rowF}
		\begin{gathered}
			f_{rowFactor} = 2 + N \cdot (4 + \frac{M}{2} \cdot (4 + 6 + 1 + 2 + 3 \cdot 2)) = \\
			= 2 + 4N + \frac{19NM}{2} = 2 + 4N + 9,5NM
		\end{gathered} 
	\end{equation}
	\item заполнение массива $colFactor$, трудоёмкость которого указана в формуле~(\ref{eq:v_colF});
	\begin{equation}
		\label{eq:v_colF}
		\begin{gathered}
			f_{colFactor} = 2 + P \cdot (4 + \frac{M}{2} \cdot (4 + 6 + 1 + 2 + 3 \cdot 2)) = \\
			= 2 + 4P + \frac{19PM}{2} = 2 + 4P + 9,5PM
		\end{gathered}  
	\end{equation}
	\item цикл заполнения для чётных размеров, трудоёмкость которого указана в формуле~(\ref{eq:v_cycle});
	\begin{equation}
		\label{eq:v_cycle}
		\begin{gathered}
			f_{cycle} = 2 + N \cdot (4 + P \cdot (2 + 7 + 4 + \frac{M}{2} \cdot (4 + 28))) = \\
			= 2 + 4N + 13NP + \frac{32NPM}{2}  = 2 + 4N + 13NP + 16NPM 
		\end{gathered}
	\end{equation}
	\item цикла, который дополнительно нужен для подсчёта значений при нечётном размере матрицы, трудоемкость которого указана в формуле~(\ref{eq:v_check});
	\begin{equation}
		\label{eq:v_check}
		\begin{gathered}
			f_{check} = 3 + 
			\begin{cases}
				0, & \text{чётная} \\
				2 + M \cdot (4 + P \cdot (2 + 14)), & \text{иначе}
			\end{cases}
		\end{gathered}  
	\end{equation}
\end{itemize}

Тогда для худшего случая (нечётный общий размер матриц) имеем:

\begin{equation}
	\label{eq:vinograd_worst}
	\begin{gathered}
		f_{worst} = f_{init} + f_{rowFactor} + f_{colFactor} + f_{cycle} + f_{check} \approx 16NMP = O(N^3)
	\end{gathered}
\end{equation}

Для лучшего случая (чётный общий размер матриц) имеем:

\begin{equation}
	\label{eq:vinograd_best}
	\begin{gathered}
		f_{best} = f_{init} + f_{rowFactor} + f_{colFactor} + f_{cycle} + f_{check} \approx 16NMP = O(N^3)
	\end{gathered}
\end{equation}

\clearpage

\subsection{Трудоемкость оптимизированного алгоритма Винограда для умножения двух матриц}

Трудоемкость оптимизированного алгоритма Винограда состоит из:

\begin{itemize}[label=---]
	\item кэширования значения $\frac{M}{2}$ в циклах, которое равно 3;
	\item создания и инициализации массивов $rowFactor$ и $colFactor$ (\ref{eq:v_init});
	\item заполнения массива $rowFactor$, трудоёмкость которого (\ref{eq:v_rowF});
	\item заполнения массива $colFactor$, трудоёмкость которого (\ref{eq:v_colF});
	\item цикла заполнения для чётных размеров, трудоёмкость которого указана в формуле (\ref{сomplexity:v_opt_cycle});
	\begin{equation}
		\label{сomplexity:v_opt_cycle}
		\begin{aligned}
			f_{cycle} = 2 + N \cdot (4 + P \cdot (4 + 7 + \frac{M}{2} \cdot (2 + 10 + 5 + 2 + 4))) = \\
			= 2 + 4N + 11NP + \frac{23NPM}{2}  = 2 + 4N + 11NP + 11,5 \cdot NPM 
		\end{aligned}
	\end{equation}
	\item условия, которое нужно для дополнительных вычислений при нечётном размере матрицы, трудоемкость которого указана в формуле~(\ref{сomplexity:v_opt_check});
	\begin{equation}
		\label{сomplexity:v_opt_check}
		\begin{aligned}
			f_{check} = 3 + 
			\begin{cases}
				0, & \text{чётная} \\
				2 + N \cdot (4 + P \cdot (2 + 10)), & \text{иначе}
			\end{cases}
		\end{aligned}  
	\end{equation}
\end{itemize}

Тогда для худшего случая (нечётный общий размер матриц) имеем:
\begin{equation}
	\label{сomplexity:vinograd_opt_worst}
	\begin{aligned}
		f_{worst} = 3 + f_{init} + f_{atmp} + f_{btmp} + f_{cycle} + f_{check} \approx 11NMP = O(N^3)
	\end{aligned}
\end{equation}

Для лучшего случая (чётный общий размер матриц) имеем:
\begin{equation}
	\label{сomplexity:vinograd_opt_best}
	\begin{aligned}
		f_{best} = 3 + f_{init} + f_{rowFactor} + f_{colFactor} \\
		 + f_{cycle} + f_{check} \approx 11NMP = O(N^3)
	\end{aligned}
\end{equation}

\clearpage

\subsection{Трудоемкость алгоритма Штрассена для умножения двух матриц}

Пусть 
\begin{itemize}[label=---]
	\item $REC$ -- трудоемкость рекурсивного алгоритма;
	\item $DIR$ -- трудоемкость прямого решения;
	\item $DIV$ -- трудоемкость разбиения ввода ($N$) на несколько частей;
	\item $COM$ -- трудоемкость объединения решений.
\end{itemize}

Тогда трудоемкость рекурсивного алгоритма считается по следующей формуле:

\begin{equation}
	\label{eq:rec}
	REC(N) =
	\begin{cases}
		DIR(N), & N \leq N_0\\
		DIV(N) + \displaystyle\sum_{i=1}^{n} REC(F[i]) + COM(N), & N > N_0
	\end{cases}
\end{equation}

где $N$ -- число входных элементов, $N_0$ -- наибольшее число, определяющее тривиальный случай (прямое решение), $n$ -- число рекурсивных вызовов для данного $N$, $F[i]$ -- число входных элементов для данного $i$.

Для расчета трудоемкости алгоритма Штрассена предположим, что размеры переданных матриц -- степени двойки.

Тогда трудоемкость алгоритма Штрассена определяется следующим образом:

\begin{itemize}[label=---]
	\item Для матрицы, размером $N \leq 2$ трудоемкость определяется как и в случае классического алгоритма умножения матриц, то есть согласно формуле \ref{eq:classic}
	\item Для матриц размером $N > 2$ определяется так:
	\begin{enumerate}[label={\arabic*)}]
		\item Трудоемкость разбиения ввода ($N$) на части. Каждый следующий вызов берется размерность матрицы в 2 раза меньше предыдущей, и происходит создание
		соответствующих подматриц и заполнение их значениями.
		\begin{equation}
			\label{eq:div}
			\begin{gathered}
			DIV(N) = 1 + 8 \cdot (3 + \frac{N}{2} \cdot ((3 + \frac{N}{2} \cdot (5 + 2 + 1)) + 2 + 1) = \\ 16 \cdot N^2 + 24 \cdot N + 25
			\end{gathered}
		\end{equation}
		\item Трудоемкость вычисления матриц $M_i, \hspace{0.25cm} i = \overline{1, 7}$ (обозначим ее буквой $G = G(N)$):
		\begin{equation}
			\label{eq:G}
			\begin{gathered}
				G(N) = 10 \cdot (2 + \frac{N}{2} \cdot (2 + \frac{N}{2} \cdot (8 + 1 + 1) + 1 + 1)) + \\
				+ 7 \cdot REC(\frac{N}{2})
			\end{gathered}
		\end{equation}
	
		где, так как $N = 2^k$ и согласно с \ref{eq:recmul}
		
		\begin{equation}
			\begin{gathered}
			REC(\frac{N}{2}) = REC(2^{k-1}) = 7 \cdot M(2^{k-2}) = \ldots 7^{i-1} M(2^{k-i}) = \ldots \\
			7^{k-1} M(2^{k-k}) = 7^{k-1}
			\end{gathered}
		\end{equation}
		
		подставляя $k = \log_2(N)$ получаем, что 
		
		\begin{equation}
			\begin{gathered}
				REC(\frac{N}{2}) = \frac{N^{\log_2(7)}}{7}
			\end{gathered}
		\end{equation}
		
		Таким образом, трудоемкость вычисления матриц $M_i, \hspace{0.25cm} i = \overline{1, 7}$ определяется следующей формулой:
		
		\begin{equation}
			\label{eq:Gfinish}
			\begin{gathered}
				G(N) = 10 \cdot (10 \cdot (\frac{N}{2})^2 + 4 \cdot \frac{N}{2} + 2) + N^{\log_2(7)} = \\
				25 \cdot N^2 + 20 \cdot N + 20 + N^{\log_2(7)}
			\end{gathered}
		\end{equation}
		
		\item Трудоемкость объединения решений, а именно формирование результирующей матрицы из вычисленных матриц $M_i, \hspace{0.25cm} i = \overline{1, 7}$
		
		\begin{equation}
			\label{eq:com}
			\begin{gathered}
				COM(N) = 8 \cdot (2 + \frac{N}{2} \cdot (2 + \frac{N}{2} \cdot (8 + 1 + 1) + 1 + 1)) + \\
				4 \cdot (3 + \frac{N}{2} \cdot ((3 + \frac{N}{2} \cdot (5 + 2 + 1)) + 2 + 1) = \\
				28 \cdot N^2 + 28 \cdot N + 28
			\end{gathered}
		\end{equation}	
	\end{enumerate}
	
	Таким образом, для матриц размером $N > 2$ трудоемкость алгоритма Штрассена согласно \ref{eq:rec} определяется так:
	
	\begin{equation}
		\label{eq:com}
		\begin{gathered}
			f_{strassen}(N) = DIV(N) + G(N) + COM(N) = \\ 16 \cdot N^2 + 24 \cdot N + 25 + 25 \cdot N^2 + 20 \cdot N + 20 + N^{\log_2(7)} + \\
		    28 \cdot N^2 + 28 \cdot N + 28 = \\
		    N^{\log_2(7)} + 69 \cdot N^2 + 72 \cdot N + 73 \approx N^{\log_2(7)} = O(N^{\log_2(7)})
		\end{gathered}
	\end{equation}
\end{itemize}

\subsection{Трудоемкость оптимизированного алгоритма Штрассена для умножения двух матриц}

При программной реализации алгоритма Штрассена не нашлось мест для применения предложенных по варианту оптимизаций, поэтому трудоемкость алгоритма Штрассена осталасть такой же, как и в предыдущем пункте.

\section*{Вывод}

В данном разделе были построены схемы алгоритмов классического умножения матриц, умножения матриц с использованием алгоритма Винограда и алгоритма Штрассена. Также были приведены оценки трудоемкости этих алгоритмов.

Согласно расчетам трудоемкости, наиболее эффективным оказался алгоритм Штрассена. Трудоемкость оптимизированной версии алгоритма Винограда в 1.5 раза меньше, чем у его неоптимизированной версии и в 1.27 раз маньше, чем у классического алгоритма. 


%\chapter{Технологический раздел}

В данном разделе будут перечислены средства реализации, листинги кода и функциональные тесты.

\section{Средства реализации}

В качестве языка программирования для этой лабораторной работы был выбран $C++$ \cite{pl} по следующим причинам:

\begin{itemize}[label=--]
	\item в $C++$ есть встроенный модуль $ctime$, предоставляющий необходимый функционал для замеров процессорного времени;
	\item в стандартной библиотеке $C++$ есть оператор $sizeof$, позволяющий получить размер переданного объекта в байтах. Следовательно, $C++$ предоставляет возможности для проведения точных оценок по используемой памяти;
	\item в $C++$ есть тип данных $std::wstring$, который позволяет хранить и использовать как кириллические, так и латинские символы.
\end{itemize}

В качестве функции, которая будет осуществлять замеры процессорного времени, будет использована функция $clock\_gettime$ из встроенного модуля $ctime$ \cite{cpu_time_func}.

\section{Сведения о модулях программы}

Программа состоит из семи модулей: 

\begin{enumerate}[label={\arabic*)}]
	\item \texttt{algorithms.cpp} --- модуль, хранящий реализации алгоритмов поиска расстояний Левенштейна и Дамерау-Левенштейна;
	\item \texttt{processTime.cpp} --- модуль, содержащий функцию для замера процессорного времени;
	\item \texttt{memoryMeasurements.cpp} --- модуль, содержащий функции, позволяющие провести сравнительный анализ использования памяти в реализациях алгоритмов поиска расстояний Левенштейна и Дамерау-Левенштейна;
	\item \texttt{timeMeasurements.cpp} --- модуль, содержащий функции, позволяющие провести сравнительный анализ использования времени в реализациях алгоритмов поиска расстояний Левенштейна и Дамерау-Левенштейна;
	\item \texttt{matrix.cpp} --- модуль, содержащий набор функций для работы с Си-матрицей.
	
	\item \texttt{main.cpp} --- файл, содержащий точку входа в программу, из которой происходит вызов алгоритмов по разработанному интерфейсу;
	\item \texttt{task7} --- модуль, содержащий набор скриптов для проведения замеров программы по времени и памяти и построения графиков по полученным данным.
\end{enumerate}

\section{Реализации алгоритмов}


В листингах \ref{lst:classicMatrixMul.txt}, \ref{lst:vinogradMatrixMul.txt}, \ref{lst:vinogradMatrixMulWithOpt.txt} приведены реализации алгоритмов нахождения расстояния Левенштейна и Дамерау-Левенштейна.

\includelisting
{classicMatrixMul.txt} % Имя файла с расширением (файл должен быть расположен в директории inc/lst/)
{Реализация классического алгоритма умножения двух матриц} % Подпись листинга

\clearpage

\includelisting
{vinogradMatrixMul.txt} % Имя файла с расширением (файл должен быть расположен в директории inc/lst/)
{Реализация алгоритма Винограда для умножения двух матриц} % Подпись листинга

\clearpage

\includelisting
{vinogradMatrixMulWithOpt.txt} % Имя файла с расширением (файл должен быть расположен в директории inc/lst/)
{Реализация оптимизированного алгоритма Винограда для умножения двух матриц} % Подпись листинга

\clearpage

%\includelisting
%{damLevRecurWithCaching.txt} % Имя файла с расширением (файл должен быть расположен в директории inc/lst/)
%{Функция нахохжения расстояния Дамерау-Левенштейна рекурсивным способом c кешированием} % Подпись листинга

\clearpage

\section{Функциональные тесты}

В таблице \ref{tbl:func_tests_std} и \ref{tbl:func_tests_vin} приведены функциональные тесты для разработанных алгоритмов умножения матриц. Все тесты пройдены успешно.

\begin{table}[ht]
	\small
	\begin{center}
		\begin{threeparttable}
			\caption{Функциональные тесты для классического алгоритма умножения матриц}
			\label{tbl:func_tests_std}
			\begin{tabular}{|c|c|c|c|c|}
				\hline
				\multicolumn{2}{|c|}{\bfseries Входные данные}
				& \multicolumn{2}{c|}{\bfseries Результат для классического алгоритма} \\
				\hline 
				\bfseries Матрица 1
				& \bfseries Матрица 2
				& \bfseries Ожидаемый результат
				& \bfseries Фактический результат \\
				\hline
				$\begin{pmatrix}
					1 & 5 & 7\\
					2 & 6 & 8\\
					3 & 7 & 9
				\end{pmatrix}$ 
				&  
				$\begin{pmatrix}
					&
				\end{pmatrix}$
				&
				\text{Сообщение об ошибке}
				&
				\text{Сообщение об ошибке} \\ 
				\hline
				$\begin{pmatrix}
					1 & 5 & 7\\
				\end{pmatrix}$ 
				&  
				$\begin{pmatrix}
					1 & 2 & 3\\
				\end{pmatrix}$
				&
				\text{Сообщение об ошибке}
				&
				\text{Сообщение об ошибке} \\ 
				\hline
				$\begin{pmatrix}
					1 & 2 & 3\\
					4 & 5 & 6 \\
					7 & 8 & 9 \\
				\end{pmatrix}$ 
				&  
				$\begin{pmatrix}
					1 & 0 & 0\\
					0 & 1 & 0 \\
					0 & 0 & 1 \\
				\end{pmatrix}$
				&
				$\begin{pmatrix}
					1 & 0 & 0\\
					0 & 1 & 0 \\
					0 & 0 & 1 \\
				\end{pmatrix}$
				&
				$\begin{pmatrix}
					1 & 2 & 3\\
					4 & 5 & 6 \\
					7 & 8 & 9 \\
				\end{pmatrix}$ \\ 
				\hline
				$\begin{pmatrix}
					3 & 5\\
					2 & 1\\
					9 & 7\\
				\end{pmatrix}$
				&
				$\begin{pmatrix}
					1 & 2 & 3\\
					4 & 5 & 6 \\
				\end{pmatrix}$
				&
				$\begin{pmatrix}
					23 & 31 & 39 \\
					6 & 9 & 12
				\end{pmatrix}$ 
				&
				$\begin{pmatrix}
					23 & 31 & 39 \\
					6 & 9 & 12
				\end{pmatrix}$ \\ 
				\hline
				$\begin{pmatrix}
					10
				\end{pmatrix}$
				&
				$\begin{pmatrix}
					35
				\end{pmatrix}$
				&
				$\begin{pmatrix}
					350
				\end{pmatrix}$ 
				&
				$\begin{pmatrix}
					350
				\end{pmatrix}$ \\ 
				\hline
			\end{tabular}
		\end{threeparttable}
	\end{center}
\end{table}

\begin{table}[ht]
	\small
	\begin{center}
		\begin{threeparttable}
			\caption{Функциональные тесты для умножения матриц по алгоритму Винограда}
			\label{tbl:func_tests_vin}
			\begin{tabular}{|c|c|c|c|c|}
				\hline
				\multicolumn{2}{|c|}{\bfseries Входные данные}
				& \multicolumn{2}{c|}{\bfseries Результат для алгоритма Винограда} \\
				\hline 
				\bfseries Матрица 1
				& \bfseries Матрица 2
				& \bfseries Ожидаемый результат
				& \bfseries Фактический результат \\
				\hline
				$\begin{pmatrix}
					1 & 5 & 7\\
					2 & 6 & 8\\
					3 & 7 & 9
				\end{pmatrix}$ 
				&  
				$\begin{pmatrix}
					&
				\end{pmatrix}$
				&
				\text{Сообщение об ошибке}
				&
				\text{Сообщение об ошибке} \\ 
				\hline
				$\begin{pmatrix}
					1 & 5 & 7\\
				\end{pmatrix}$ 
				&  
				$\begin{pmatrix}
					1 & 2 & 3\\
				\end{pmatrix}$
				&
				\text{Сообщение об ошибке}
				&
				\text{Сообщение об ошибке} \\ 
				\hline
				$\begin{pmatrix}
					1 & 2 & 3\\
					4 & 5 & 6 \\
					7 & 8 & 9 \\
				\end{pmatrix}$ 
				&  
				$\begin{pmatrix}
					1 & 0 & 0\\
					0 & 1 & 0 \\
					0 & 0 & 1 \\
				\end{pmatrix}$
				&
				$\begin{pmatrix}
					1 & 0 & 0\\
					0 & 1 & 0 \\
					0 & 0 & 1 \\
				\end{pmatrix}$
				&
				$\begin{pmatrix}
					1 & 2 & 3\\
					4 & 5 & 6 \\
					7 & 8 & 9 \\
				\end{pmatrix}$ \\ 
				\hline
				$\begin{pmatrix}
					3 & 5\\
					2 & 1\\
					9 & 7\\
				\end{pmatrix}$
				&
				$\begin{pmatrix}
					1 & 2 & 3\\
					4 & 5 & 6 \\
				\end{pmatrix}$
				&
				\text{Сообщение об ошибке} 
				&
				\text{Сообщение об ошибке} \\ 
				\hline
				$\begin{pmatrix}
					10
				\end{pmatrix}$
				&
				$\begin{pmatrix}
					35
				\end{pmatrix}$
				&
				$\begin{pmatrix}
					350
				\end{pmatrix}$ 
				&
				$\begin{pmatrix}
					350
				\end{pmatrix}$ \\ 
				\hline
			\end{tabular}
		\end{threeparttable}
	\end{center}
\end{table}

\section*{Вывод}

В данном разделе были реализованы и протестированы три алгоритма:
классический алгоритм умножения, алгоритм Винограда умножения двух матриц, оптимизированный алгоритм Винограда умножения двух матриц.

    
%\chapter{Исследовательский раздел}

В данном разделе будут проведены сравнения реализаций алгоритмов умножения матриц 
по времени работы и по затрачиваемой памяти.

\section{Технические характеристики}

Технические характеристики устройства, на котором проводились исследования: 

\begin{itemize}[label=--]
	\item операционная система: Ubuntu 22.04.3 LTS x86\_64 \cite{os};
	\item оперативная память: 16 Гб;
	\item процессор: 11th Gen Intel® Core™ i7-1185G7 @ 3.00GHz × 8.
\end{itemize}

\section{Время выполнения алгоритмов}

Время работы алгоритмов измерялось с использованием функции $clock\_gettime$ из встроенного модуля $ctime$. 

Замеры времени для каждого размера матрицы проводились 1000 раз. На вход подавались случайно сгенерированные матрицы заданного размера. 

Результаты замеров реализаций алгоритмов по времени представлены в таблице \ref{tbl:time}. Их графическое представление показано на рисунке \ref{img:linear_graph_time}.

\begin{table}[ht]
	\small
	\begin{center}
		\begin{threeparttable}
			\caption{Замер памяти для матриц размером от 10 до 100}
			\label{tbl:time}
			\begin{tabular}{|c|c|c|c|}
				\hline
				& \multicolumn{3}{c|}{\bfseries Время, нс} \\ \cline{2-4}
				\bfseries Линейный размер, штуки & \bfseries Классический & \bfseries Виноград & \bfseries Виноград (опт)  \\ \cline{2-4}
				\hline
				10 & 3416.896 & 3180.420 & 3134.480  \\
				\hline
				20 & 29545.140 & 22744.160 & 21871.700  \\
				\hline
				30 & 94943.940 & 74144.360 & 73816.280  \\
				\hline
				40 & 218554.000 & 177503.000 & 169667.400 \\
				\hline
				50 & 433830.200 & 345092.400 & 332352.600 \\
				\hline
				60 &  742386.600 & 588659.800 & 574165.200  \\
				\hline
				70 & 1150328.000 & 930386.000 & 915796.200  \\
				\hline
				80 & 1718110.000 & 1381020.000 & 1344386.000  \\
				\hline
				90 & 2463566.000 & 1984904.000 & 1919098.000  \\
				\hline
				100 & 3386088.000 & 2715484.000 & 2632198.000  \\
				\hline
			\end{tabular}	
		\end{threeparttable}
	\end{center}
\end{table}

\clearpage

\begin{figure}[h]
	\centering
	\includesvg[width=0.9\textwidth]{linear_graph_time}
	\caption{Результаты замеров времени работы алгоритмов для матриц размеров от 10 до 100}
	\label{img:linear_graph_time}
\end{figure}




\clearpage


\section{Использование памяти}

Введем следующие обозначения:
\begin{itemize}[label=--]
	\item$l_1$ --- длина строки $S_{1}$;
	\item$l_2$ --- длина строки $S_{2}$;
	\item$sizeof()$ --- функция вычисляющая размер в байтах;
	\item $wstring$ --- строковый тип;
	\item $int$ --- целочисленный тип;
	\item $size\_t$ --- беззнаковый целочисленный тип.
\end{itemize}

Максимальная глубина стека вызовов при рекурсивной реализации нахождения расстояния Дамерау-Левенштейна равна сумме входящих строк, а на каждый вызов требуется 2 дополнительные переменные, соответственно, максимальный расход памяти равен:

\begin{equation}
	\label{eq:dl_rec_memory}
	(l_1 + l_2) \cdot ( 2 \cdot sizeof(size\_t)  + 2 \cdot sizeof(int)) + 2 * sizeof(wstring),
\end{equation}

где:
\begin{itemize}[label=--]
	\item $2 \cdot sizeof(size\_t)$ --- хранение размеров строк;
	\item $2 \cdot sizeof(int)$ --- дополнительные переменные;
	\item $2 \cdot sizeof(wstring)$ --- хранение двух строк.
\end{itemize}

Расчет используемой памяти для рекурсивного алгоритма c кешированием поиска расстояния Дамерау-Левенштейна будет теоретически схож с расчетом в формуле \ref{eq:dl_rec_memory}, но также учитывается матрица, соответственно, максимальный расход памяти равен:

\begin{equation}
	\label{eq:dl_hash_memory}
	\begin{aligned}
		(l_1 + 1) \cdot (l_2 + 1) \cdot sizeof(int) + sizeof(int **) + (l_1 + 1) \cdot sizeof(int *) + \\
		(l_1 + l_2) \cdot (2 \cdot sizeof(size\_t) + sizeof(int)) + 2 \cdot sizeof(wstring)
	\end{aligned}
\end{equation}

где 
\begin{itemize} [label=--]
	\item $(l_1 + 1) \cdot (l_2 + 1) \cdot sizeof(int)$ --- хранение матрицы;
	\item $sizeof(int **)$ -- указатель на матрицу;
	\item $(l_1 + 1) \cdot sizeof(int *)$ -- указатель на строки матрицы;
	\item $2 \cdot size(size\_t)$ --- хранение размеров строк;
	\item $sizeof(int)$ --- дополнительная переменная;
	\item $2 * sizeof(wstring)$ --- хранение двух строк.
\end{itemize}

Расчет использования памяти при итеративной реализации алгоритма поиска расстояния Левенштейна теоретически равен:

\begin{equation}
	\label{eq:lev_mtr_memory}
	\begin{aligned}
		2 \cdot (l_2 + 1) \cdot sizeof(int) +  2 \cdot sizeof(wstring) + \\
		 2 \cdot sizeof(size\_t) + sizeof(int),
	\end{aligned}
\end{equation}

где 
\begin{itemize}[label=--]
	\item $2 \cdot (l_2 + 1) \cdot sizeof(int)$ --- хранение двух массивов;
	\item $2 \cdot size(wstring)$ --- хранение двух строк;
	\item $2 \cdot sizeof(size\_t)$ --- хранение размеров строк;
	\item $sizeof(int)$ --- дополнительная переменная.
\end{itemize}

Расчет использования памяти при итеративной реализации алгоритма поиска расстояния Дамерау-Левенштейна теоретически равен:

\begin{equation}
	\label{eq:dl_mtr_memory}
	\begin{aligned}
		(l_1 + 1) \cdot (l_2 + 1) \cdot sizeof(int) + sizeof(int **) + (l_1 + 1) \cdot sizeof(int *) +  \\
		2 \cdot sizeof(wstring) + 2 \cdot sizeof(size\_t) + sizeof(int),
	\end{aligned}
\end{equation}

где 
\begin{itemize} [label=--]
	\item $(l_1 + 1) \cdot (l_2 + 1) \cdot sizeof(int)$ --- хранение матрицы;
	\item $sizeof(int **)$ -- указатель на матрицу;
	\item $(l_1 + 1) \cdot sizeof(int *)$ -- указатель на строки матрицы;
	\item $2 * sizeof(wstring)$ --- хранение двух строк;
	\item $2 \cdot size(size\_t)$ --- хранение размеров матрицы;
	\item $sizeof(int)$ --- дополнительная переменная.
\end{itemize}

\begin{table}[ht]
	\small
	\begin{center}
		\begin{threeparttable}
			\caption{Замер памяти для строк, размером от 10 до 200}
			\label{tbl:memory}
			\begin{tabular}{|c|c|c|c|c|}
				\hline
				& \multicolumn{4}{c|}{\bfseries Размер, байты} \\ \cline{2-5}
				& \multicolumn{1}{c|}{\bfseries Левенштейн}
				& \multicolumn{3}{c|}{\bfseries Дамерау-Левенштейн} \\ \cline{2-5}
				\bfseries Длина, символ & \bfseries Итеративный & \bfseries Итеративный & \multicolumn{2}{c|}{\bfseries Рекурсивный} \\ \cline{4-5}
				& & & \bfseries Без кеша & \bfseries С кешом \\
				\hline
				10 & 252 & 748 & 624 & 1128 \\
				\hline
				20 & 412 & 2188 & 1184 & 2968 \\
				\hline
				30 & 572 & 4428 & 1744 & 5608 \\
				\hline
				40 & 732 & 7468 & 2304 & 9048 \\
				\hline
				50 & 892 & 11308 & 2864 & 13288 \\
				\hline
				60 &  1052 & 15948 & 3424 & 18328 \\
				\hline
				70 & 1212 & 21388 & 3984 & 24168 \\
				\hline
				80 &1372 & 27628 & 4544 & 30808 \\
				\hline
				90 & 1532 & 34668 & 5104 &  38248 \\
				\hline
				100 & 1692 & 42508 & 5664 & 46488 \\
				\hline
			\end{tabular}	
		\end{threeparttable}
	\end{center}
\end{table}

%\begin{figure}[h]
%	\centering
%	\includesvg[width=0.9\textwidth]{linear_graph_mem}
%	\caption{Результаты вычислений используемой памяти реализаций алгоритмов для строк с длиной от 10 до 100}
%	\label{img:linear_graph_mem}
%\end{figure}


\clearpage

Анализируя таблицу \ref{tbl:memory}, сравним нерекурсивные реализации алгоритмов поиска расстояний Левенштейна и Дамерау-Левенштейна. При любой длине строки от 10 до 100 символов итеративный алгоритм поиска расстояния Левенштейна использует меньше памяти: при длине строки в 10 символов - в 3 раза меньше, а при длине строки в 100 символов уже в 25 раз. Такие результаты объясняются тем, что нерекурсивная реализация алгоритма поиска расстояния Дамерау-Левенштейна использует матрицу для сохранения ранее вычисленных значений, в то время как нерекурсивной реализации алгоритма поиска расстояния Левенштейна необходимы лишь текущая и предыдущая строки матрицы.

Сравнивая рекурсивную и рекурсивную с кешированием реализации алгоритмов поиска расстояний Левенштейна, можно увидеть, что использования матрицы в качестве кеша также приводит к быстрому росту используемой памяти в зависимости от длины входных строк.

При рассмотрении нерекурсивной и рекурсивной реализаций алгоритма поиска расстояний Дамерау-Левенштейна видно, что последняя с ростом длины строки использует в несколько раз меньше памяти, чем нерекурсивная: при длине строки в 10 символов -- в 1.2 раза меньше, а при длине строки в 100 символов -- в 7.5 раз меньше. Это связано с тем, что нерекурсивная реализация использует матрицу, в то время как рекурсивная использует только память, выделенную под локальные переменные при каждом рекурсивном вызове функции.

\section*{Вывод}

В данном разделе были проведены замеры времени работы а также расчеты используемой памяти реализаций алгоритмов поиска расстояний Левенштейна и Дамерау-Левенштейна. 

Итеративные реализации алгоритмов поиска расстояний Левенштейна и Дамерау-Левенштейна работают быстрее рекурсивных, поскольку при итеративных реализациях не происходит многократного расчета одних и тех же промежуточных значений в ходе работы алгоритма.

Однако, рекурсивные алгоритмы более эффективные при использовании памяти, поскольку при использовании рекурсивной реализации происходит выделение памяти только под локальные переменные при каждом рекурсивном вызове.

Использование матрицы в качестве кеша в рекурсивной реализации алгоритма Дамерау-Левенштейна позволило сократить время работы алгоритма, но увеличило количество используемой памяти.


%\chapter*{ЗАКЛЮЧЕНИЕ}
\addcontentsline{toc}{chapter}{ЗАКЛЮЧЕНИЕ}

В ходе выполнения лабораторной работы были решены следующие задачи:

\begin{enumerate}[label={\arabic*)}]
	\item описаны три алгоритма сортировки: блочной, слиянием и поразрядной;
	\item создано программное обеспечение, реализующее следующие алгоритмы:
	\begin{itemize}[label=--]
		\item алгоритм блочной сортировки;
		\item алгоритм сортировки слиянием;
		\item алгоритм поразрядной сортировки;
	\end{itemize}
	
	\item проведен анализ эффективности реализаций алгоритмов по памяти и по времени;
	\item проведена оценка трудоемкости алгоритмов сортировки;
	\item подготовлен отчет по лабораторной работе;
\end{enumerate}

Цель данной лабораторной работы, а именно исследование трех алгоритмов сортировки: блочной сортировки, сортировки слиянием и поразрядной сортировки, также была достигнута.

Согласно теоретическим расчетам трудоемкости алгоритмов сортировки наименее трудоемким на равномерно распределенных данных оказался алгоритм блочной сортировки, наиболее трудоемким -- алгоритм поразрядной сортировки.

Для равномерно распределенных данных лучше всего использовать алгоритмы блочной и поразрядной сортировки, так как среднее отклонение во времени работы порядка 5 мкс, и разница в расходе используемой памяти не превышает 52 байт.

Для данных, упорядоченных по возрастанию лучше всего использовать алгоритм блочной сортировки, поскольку этот алгоритм показал наименьшее время работы и наименьший расход используемой памяти среди всех трех алгоритмов.

Для данных, упорядоченных по убыванию лучше всего использовать алгоритмы блочной и поразрядной сортировки, так как среднее отклонение во времени работы порядка 5 мкс, и разница в расходе используемой памяти не превышает 52 байт.

Для массива данных, состоящего из одинаковых элементов, лучше всего использовать алгоритм блочной сортировки, поскольку этот алгоритм показал наименьшее время работы (в $1.5$ раза быстрее алгоритма сортировки слиянием и в $1.26$ раза быстрее алгоритма поразрядной сортировки) и наименьший расход используемой памяти среди всех трех алгоритмов.

Результаты проведенного исследования практически подтвердили теоретические расчеты трудоемкости: наиболее эффективным по времени работы и по используемой памяти является алгоритм блочной сортировки, но наиболее трудоемким оказался алгоритм сортировки слиянием.



\makebibliography

%\include{09-appendix}

\end{document}