ИУ7-53Б, 11\_KNY, Лысцев Никита

\section*{Методы статистического машинного перевода}

% решаемая МП задача

В современном мире существует большое количество статей, технической документации, книг, написанных на разных языках. Перевод таких текстов с одного языка на другой человеком вручную требует значительных временных и финансовых затрат, а также наличия соответствующих навыков. Возникает проблема перевода. Решением данной проблемы может стать максимальная автоматизация этого процесса. С этим помогают системы машинного перевода (МП).

% определение СМП и языковых пар

Статистический машинный перевод (СМП) -- разновидность МП, которая основана на поиске наиболее вероятного перевода предложения с использованием данных, полученных из двуязычной совокупности текстов (параллельного корпуса) в результате  обучения (по языковым парам)\cite{smt}.

Языковые пары -- тексты, содержащие предложения, написанные на одном языке и соответствующие им предложения на другом языке. Чем большим количеством языковых пар располагает система, тем лучше получается текст на выходе. Анализируя языковые пары, система определяет наиболее вероятный вариант перевода.

\subsection*{Метод перевода, основанный на словах}

Введем некоторые обозначения:

\begin{itemize}[label*=--]
	\item $f$ -- слово на языке источника, с которого нужно переводить;
	\item $e$ -- слово на языке приемника, на который нужно переводить.
\end{itemize}

Понятие статистического машинного перевода подразумевает использование статистики. На основе анализа языковых пар для каждого слова $f$ в предложении выполняется подсчет количества переводов для каждого возможного варианта перевода $e$ на язык приемника. Далее, на основе этих подсчетов, производится оценка распределения вероятности лексического перевода. 

\clearpage

Формально, находится функция \cite{smt_book}

\begin{equation}
	\label{eq:prob}
	p_f : e \rightarrow p_f(e)
\end{equation}

что для слова $f$ возвращает вероятность для каждого выбора перевода $e$, которая указывает, насколько вероятен этот перевод.

Функция \ref{eq:prob} должна  обладать следующим свойством:

\begin{equation}
	\label{eq:prop1_prob}
	\displaystyle\sum_e p_f(e) = 1
\end{equation}

После получения вероятностного распределения для каждого слова $f$ в предложении выбирается наиболее вероятный перевод $e$. Сопоставление слова $f$ и его перевода $e$ в предложении на языках источника и приемника определяются функцией выравнивания:

\begin{equation}
	\label{eq:alig}
	a : j \rightarrow i
\end{equation}

которая отображает каждое выходное слово в позиции $i$ к входному слову в позиции $j$.

Вероятности лексического перевода и понятие выравнивания позволяют нам определить модель, которая генерирует несколько различных переводов предложения, каждый из которых имеет разную вероятность. Эта модель называется IBM Модель 1.

\clearpage

При статистическом подходе используется такое понятие как «канал   помехами»\cite{smt}: рассматривая перевод с английского на русский, предполагается, что английское предложение на самом деле – русское предложение, но искаженное неким шумом. Для корректного перевода или же «расшифровки шума» нам нужно знать, что в рассматриваемом случае говорят русскоязычные и как рассматриваемый текст искажается, что превращается в текст на английском языке. Перевод осуществляется после поиска такого русского предложения, которое максимизирует вероятность «встречи» английского предложения и соответствующего ему русского предложения. В основе такого поиска лежит теорема Байеса:

\begin{equation}
	\label{eq:L}
	P(e|f) = P(f|e) \times P(e)
\end{equation}
где e – русское предложение, f – английское предложение.

Таким образом, в основе перевода, то есть вероятности того, что
английское предложение e будет переведено русским предложением f, лежат
модель языка, то есть P(e), и модель перевода – P(f|e). Задача, которая
ставится перед системой статистического машинного перевода – не перевод,
как таковой, а декодирование. То есть, в данном случае, перевод – не что
иное, как расшифровка исходного текста. 

Алгоритм перевода можно изобразить следующей схемой.

\includeimage
{alg} % Имя файла без расширения (файл должен быть расположен в директории inc/img/)
{f} % Обтекание (без обтекания)
{h} % Положение рисунка (см. figure из пакета float)
{0.4\textwidth} % Ширина рисунка
{Алгоритм статистического перевода \cite{smt2}} % Подпись рисунка


\clearpage

Итак, весь алгоритм основывается на выравнивании, создании моделей
и самом переводе. Рассмотрим данные составляющие более подробно

Выравнивание – процесс, в результате которого на выходе мы
получаем проработанный текст, в котором за каждым предложением на
языке оригинала идет соответствующее предложение на языке перевода. Для выравнивания текстов система использует инструментальное средство
автоматического выравнивания параллельных текстов (alignment tool). С
помощью таких инструментов система может проанализировать и привести в
соответствие тексты оригинала и перевода \cite{smt2}. 

Подготовленные таким образом тексты используются для обучения модели перевода.

Существуют 5 основных моделей статистического перевода: IBM 1, IBM 2, IBM 3, IBM 4, IBM 5 \cite{smt}. В первой модели для перевода предложения с одного языка на другой достаточно перевести все слова, а за их расстановку в предложении отвечает модель языка. Такая модель оперирует таблицей вероятностей попарно переводных соответствий двух языков. Вторая модель призвана сохранить порядок слов при переводе. Помимо таблицы вероятностей попарно переводных соответствий, используется таблица вероятностей обратных смещений. Третья модель устраняла недостаток второй модели, связанный с тем, что одному слову оригинала не могло соответствовать сразу несколько слов перевода. В моделях 4 и 5 появляются включения грамматики. Результатом формирования модели перевода является таблица вероятностей слов и фраз. 

Модель языка оценивает фразы целевого языка и дает им соответствующую вероятность. В качестве модели языка в статистическом переводе используются n-граммные модели (в частности, триграммные), определяющие выбор слова на основе предшествующих слов.

На этапе перевода предложения в составе текста делятся на фразы. Для каждой фразы производится поиск наиболее вероятного перевода в таблице фраз, который максимизирует произведение вероятностей текста оригинала и текста перевода согласно теореме Байеса. 